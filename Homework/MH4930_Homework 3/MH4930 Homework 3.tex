\documentclass{article}

%%%%%%%%%%%%%%%%%%%%%%%%%%%%%%%%%%
%%%%%%%%%%%% Packages %%%%%%%%%%%%
%%%%%%%%%%%%%%%%%%%%%%%%%%%%%%%%%%
\usepackage{graphicx}
\usepackage{tikz, quiver}

% font package
\usepackage{amsfonts, dsfont}
\usepackage[T1]{fontenc}

% math packages
\usepackage{amsmath, amsthm, amssymb, amsfonts}

% for better cross-reference workflow
%\usepackage{showlabels}

% for better list and enumeration
\usepackage[shortlabels]{enumitem}

% to-do notes
\usepackage{todonotes}

% Removes paragraph indent and adds spacing between paragraphs
\usepackage[parfill]{parskip} 
\usepackage{setspace}

% Darkmode
\usepackage{darkmode}
%\enabledarkmode

%%%%%%%%%%%%%%%%%%%%%%%%%%%%%%%%%%%%%
%%%%%%%%%%%% Environment %%%%%%%%%%%%
%%%%%%%%%%%%%%%%%%%%%%%%%%%%%%%%%%%%%

\newtheorem{thm}{Theorem}
\newtheorem{cor}[thm]{Corollary}
\newtheorem{lem}[thm]{Lemma}
\newtheorem{pro}[thm]{Proposition}

\theoremstyle{definition}
\newtheorem{defn}[thm]{Definition}
\newtheorem{re}[thm]{Remark}
\newtheorem{ex}[thm]{Exercise}
\newtheorem{qn}[thm]{Question}
\newtheorem{pr}[thm]{Problem}

\newenvironment{answer}{\proof[Answer]}{\endproof}

%%%%%%%%%%%%%%%%%%%%%%%%%%%%%%%%
%%%%%%%%%%%% Symbol %%%%%%%%%%%%
%%%%%%%%%%%%%%%%%%%%%%%%%%%%%%%%
\newcommand{\op}{\operatorname{op}}
\newcommand{\ann}{\operatorname{Ann}}
\newcommand{\im}{\operatorname{im}}
\newcommand{\h}{\operatorname{Hom}}
\newcommand{\id}{\operatorname{id}}
\newcommand{\obj}{\operatorname{Obj}}
\newcommand{\mor}{\operatorname{Mor}}
\newcommand{\f}{\mathcal F}
\newcommand{\cat}{\mathcal C}
\newcommand{\ep}{\varepsilon}
\newcommand{\ext}{\operatorname{Ext}}
\newcommand{\tor}{\operatorname{Tor}}

\newcommand{\xto}[1]{\xrightarrow{#1}}
\newcommand{\ot}{\leftarrow}
\newcommand{\xot}[1]{\xleftarrow{#1}}

\newcommand{\pd}{+\dots+}
\newcommand{\many}{,\dots, }
\newcommand{\ten}{\otimes}

%%%%%%%%%%%%%%%%%%%%%%%%%%%%%%%%%%%%%
%%%%%%%%%%%% New Command %%%%%%%%%%%%
%%%%%%%%%%%%%%%%%%%%%%%%%%%%%%%%%%%%%

% Rounded brackets
\newcommand{\br}[1]{\left(#1\right)}

% Curly Brackets
\newcommand{\sbr}[1]{\left\{#1\right\}}

% Functions
\newcommand{\func}[2]{#1 \br {#2}}

% Inverse functions
\newcommand{\inv}[1]{^{#1}}

% Compact list
\setlist[itemize]{itemsep=3pt, topsep=2pt}
\setlist[enumerate]{itemsep=3pt, topsep=2pt}

%%%%%%%%%%%%%%%%%%%%%%%%%%%%%%
%%%%%%%%%%%% Size %%%%%%%%%%%%
%%%%%%%%%%%%%%%%%%%%%%%%%%%%%%
\usepackage{geometry}
\geometry{a4paper, margin=1in}

%%%%%%%%%%%%%%%%%%%%%%%%%%%%%%
%%%%%%%%%% Spacing %%%%%%%%%%%
%%%%%%%%%%%%%%%%%%%%%%%%%%%%%%
\setlength{\parskip}{0.5\baselineskip} 
\setlength{\parindent}{0pt}

\begin{document}

Name: Loo Wee Lun

\medskip

\textbf{Exercise 61} Thie exericse defines the connecting homomorphism $\delta_n$ in Theorem 17.2 (The Long Exact Sequence in Cohomology). Let $0\to X\xto{\alpha}Y\xto{\beta} Z \to 0 $ be a SES of cochain complexes. Let $a\in H^n(Z)$ and $a=z + \im d_n$ where $z\in \ker d_{n+1} :Z^n \to Z^{n+1}$.
\begin{enumerate}
    \item Show that there exists $y\in Y^n$ such that $\beta_n(y) = z$ and a unique $x\in \ker d_{n+2}\subseteq X^{n+1}$ such that $\alpha(x) = d(y)$.
    \item Let $z + \im d_n = z' + \im d_n$, and $y,y', x, x'$ such that $\beta(y)=z,\ \beta(y')=z',\ \alpha(x) = d(y)$ and $\alpha(x')=d(y')$. Show that $x + \im d_{n+1} = x' + \im d_{n+1}$.
    \item Conclude that we have a map $\delta_n:H^n(Z) \to H^{n+1}(X)$ defined by $\delta_n(z+\im d_n) = x + \im d_{n+1}$.
    \item Prove that the connecting homomorphism $\delta_n$ is a group homomorphism.
\end{enumerate}

\begin{proof}
    \hfill

    1. Since we have SES of cochain complex, so $0 \to X^n \xto{\alpha_n} Y^n \xto{\beta_n} Z^n \to 0$ is a SES. By definition $\im \beta_n = \ker 0 = Z$, implying that $\beta_n$ is surjective. Thus given $z\in \ker d_{n+1} \subseteq Z^n$, there exists $y\in Y^n$ such that $\beta_n(y)=z$.

    Send $y\in Y^n$ along two paths we have 
    \[\beta_{n+1}(d_{n+1}^Y(y)) = d_{n+1}^Z(\beta_n(y))=d_{n+1}^Z(z)=0\]
    since $z\in \ker d_{n+1}$. So $d_{n+1}^Y(y) \in \ker \beta_{n+1} = \im \alpha_{n+1}$. Note $\alpha$ is injective, so there exists a unique $x\in X^{n+1}$ such that $\alpha_{n+1}(x)=d_{n+1}^Y(y)$. We now check that $x\in \ker d_{n+2}^X$. Send $x$ along two paths we have
    \[\alpha_{n+2}(d_{n+2}^X(x)) = d_{n+2}^Y(\alpha_{n+1}(x) = d_{n+2}^Y(d_{n+1}^Y)(y)) = 0\]
    Note $\alpha$ is injective, so $\ker \alpha$ is trivial, we thus have $d_{n+2}^X(x)= 0$, so $x\in \ker d_{n+2}^X$.

    2. Since $z+\im d_n = z' + \im d_n$, so $z-z'\in \im d_n$. Let $w\in Z^{n-1}$ such that $d_n^Z(w)=z-z'$. Since $\beta$ is surjective, there exists $y_w\in Y^{n-1}$ such that $\beta_{n-1}(y_w)=w$. Send $y_w$ along two paths we have 
    \[\beta_{n}(d_n^Y(y_z)) = d^Z_{n}(\beta_{n-1}(y_w))= d^Z_{n}(w) = z-z' = \beta_n(y)-\beta_n(y') = \beta_n(y-y')\]
    Rearranging we see $\beta_n(d_n^Y(y_z) - (y-y'))=0$, so $d_n^Y(y_z) - (y-y')\in \ker \beta_n = \im \alpha_n$. Let $x_z\in X^n$ such that 
    \[\alpha_n(x_z) = d_n^Y(y_z) - (y-y')\]
    Lastly, send $x_z$ along two paths we have
    \begin{align*}
        \alpha_{n+1}(d_{n+1}^X(x_z)) 
        &= d_{n+1}^Y(\alpha_n(x_z))\\ 
        &=d_{n+1}^Y(d_n^Y(y_z) - (y-y')) \\
        &= 0 -d_{n+1}^Y(y) + d_{n+1}^Y(y')\\
        &= -\alpha_{n+1}(x) + \alpha_{n+1}(x')\\
        &= \alpha_{n+1}(x'-x)
    \end{align*}
    Since $\alpha$ is injective, we have that $x'-x = d_{n+1}^X(x_z)$, implying that $x'-x \in \im d_{n+1}^X$, so $x + \im d_{n+1} = x' + \im d_{n+1}$.

    3. As suggested, consider the map $\delta_n:H^n(Z)\to H^{n+1}(X)$ defined by $\delta_n(z+\im d_n) = x+\im d_{n+1}$ where $x$ is given as follow:
    \begin{itemize}
        \item In part 1, we have shown that by fixing $z$, there correspond some $y$ (need not be unique). Also, there exists a unique $x$ such that $\alpha(x) = d(y)$. Let this $x$ be such that $\delta_n(z+\im d_n) = x+\im d_{n+1}$
        \item The elements of $H^n(Z)$ do take the form $z+\im d_n$, and the elements of $H^{n+1}(X)$ do take the form $x + \im d_{n+1}$, by the definition of cohomology.
        \item We have to check well-definedness, which is already settled in part 2.
        \item Moreover, we claim that $\delta_n(z+\im d_n)$ is independent of the choice of the middleman $y$ where $\beta_n(y) = z$. Fix $z$, if we take two different pre-image of $z$, say $y$ and $y'$ such that $\beta_n(y) = z = \beta_n(y')$ and $\alpha(x)=d(y)$, $\alpha(x')=d(y')$, since we have $z + \im d_n = z + \im d_n$, applying part 2 tells us that $x+ \im d_{n+1} = x' + \im d_{n+1}$, therefore no problem occurs.
    \end{itemize}

    4. Lastly, we show that the defined map $\delta_n$ is a group homomorphism. Let $z_1 + \im d_n, z_2 + \im d_n \in H^n(Z)$. Futher suppose that $\beta(y_1)=z_1, \beta(y_2)=z_2, \alpha(x_1)=d(y_1), \alpha(x_2)=d(y_2)$, where $x_1, x_2 \in \ker d_{n+2}$. Thus by definition we have 
    \[\delta_n(z_1+\im d_n) = x_1+\im d_{n+1} \quad \text{and} \quad \delta_n(z_2+\im d_n) = x_2+\im d_{n+1}\]
    Note $\beta(y_1+y_2) = z_1+z_2$, and $\alpha(x_1+x_2) = d(y_1 + y_2)$. Also $x_1 + x_2 \in \ker d_{n+2}$ since kernel is a subgroup. By definition, we see
    \begin{align*}
        \delta_n((z_1+\im d_n) + (z_2+\im d_n))=\delta_n((z_1+z_2)+\im d_n) 
        &= (x_1 + x_2) + \im d_{n+1}\\
        &= (x_1 + \im d_{n+1}) + (x_2 + \im d_{n+1})\\
        &= \delta_n(z_1+\im d_n) + \delta_n(z_2+\im d_n)
    \end{align*}
    This shows that $\delta_n$ is indeed a group homomorphism.
\end{proof}


\end{document}
