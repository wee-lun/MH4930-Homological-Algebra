\documentclass{article}

%%%%%%%%%%%%%%%%%%%%%%%%%%%%%%%%%%
%%%%%%%%%%%% Packages %%%%%%%%%%%%
%%%%%%%%%%%%%%%%%%%%%%%%%%%%%%%%%%
\usepackage{graphicx}
\usepackage{tikz}
\usepackage{quiver}

% math packages
\usepackage{amsmath}
\usepackage{amsthm}
\usepackage{amsfonts, dsfont}
\usepackage[T1]{fontenc}

% for better cross-reference workflow
%\usepackage{showlabels}
\usepackage[bookmarks=True]{hyperref}
\usepackage{bookmark}

% for better list and enumeration
\usepackage[shortlabels]{enumitem}

% to-do notes
\usepackage{todonotes}

% Removes paragraph indent and adds spacing between paragraphs
\usepackage[parfill]{parskip} 
\usepackage{setspace}

% darkmode
\usepackage{darkmode}
%\enabledarkmode

%%%%%%%%%%%%%%%%%%%%%%%%%%%%%%%%%%%%%
%%%%%%%%%%%% Environment %%%%%%%%%%%%
%%%%%%%%%%%%%%%%%%%%%%%%%%%%%%%%%%%%%

\newtheorem{thm}{Theorem}[subsection]
\newtheorem{cor}[thm]{Corollary}
\newtheorem{lem}[thm]{Lemma}
\newtheorem{pro}[thm]{Proposition}

\theoremstyle{definition}
\newtheorem{re}[thm]{Remark}
\newtheorem{defn}[thm]{Definition}
\newtheorem{ex}[thm]{Example}

%%%%%%%%%%%%%%%%%%%%%%%%%%%%%%%%
%%%%%%%%%%%% Symbol %%%%%%%%%%%%
%%%%%%%%%%%%%%%%%%%%%%%%%%%%%%%%
\newcommand{\Z}{{\mathbb Z}}
\newcommand{\C}{{\mathbb C}}
\newcommand{\R}{{\mathbb R}}
\newcommand{\N}{{\mathbb N}}
\newcommand{\Q}{{\mathbb Q}}

\newcommand{\op}{\operatorname{op}}
\newcommand{\ann}{\operatorname{Ann}}
\newcommand{\im}{\operatorname{im}}
\newcommand{\h}{\operatorname{Hom}}
\newcommand{\id}{\operatorname{id}}
\newcommand{\obj}{\operatorname{Obj}}
\newcommand{\mor}{\operatorname{Mor}}
\newcommand{\f}{\mathcal F}
\newcommand{\cat}{\mathcal C}
\newcommand{\ep}{\varepsilon}
\newcommand{\coker}{\operatorname{coker}}
\newcommand{\ext}{\operatorname{Ext}}
\newcommand{\tor}{\operatorname{Tor}}
\newcommand{\aut}[1]{\operatorname{Aut}\br{#1}}
\newcommand{\gl}{\operatorname{GL}}
\newcommand{\gal}{\operatorname{Gal}}

\newcommand{\xto}[1]{\xrightarrow{#1}}
\newcommand{\ot}{\leftarrow}
\newcommand{\xot}[1]{\xleftarrow{#1}}

\newcommand{\pd}{+\dots+}
\newcommand{\many}{,\dots, }
\newcommand{\ten}{\otimes}

%%%%%%%%%%%%%%%%%%%%%%%%%%%%%%%%%%%%%
%%%%%%%%%%%% New Command %%%%%%%%%%%%
%%%%%%%%%%%%%%%%%%%%%%%%%%%%%%%%%%%%%

% Rounded brackets
\newcommand{\br}[1]{\left(#1\right)}

% Curly Brackets
\newcommand{\sbr}[1]{\left\{#1\right\}}

% Functions
\newcommand{\func}[2]{#1 \br {#2}}

% Inverse functions
\newcommand{\invfunc}[3]{#1^{-#2} \br {#3}}

% Compact list
\setlist[itemize]{itemsep=3pt, topsep=2pt}
\setlist[enumerate]{itemsep=3pt, topsep=2pt}

%%%%%%%%%%%%%%%%%%%%%%%%%%%%%%
%%%%%%%%%%%% Size %%%%%%%%%%%%
%%%%%%%%%%%%%%%%%%%%%%%%%%%%%%
\usepackage{geometry}
\geometry{a4paper, margin=1in}

%%%%%%%%%%%%%%%%%%%%%%%%%%%%%%
%%%%%%%%%% Spacing %%%%%%%%%%%
%%%%%%%%%%%%%%%%%%%%%%%%%%%%%%
\setlength{\parskip}{0.5\baselineskip} 
\setlength{\parindent}{0pt}

\begin{document}

\begin{titlepage}
        \centering
        \hfill\par
        \vspace{4\baselineskip}
        {\Huge 
        MH4930: Special Topics in Mathematics \par Homological Algebra}
        \vspace{4\baselineskip}
        
        {by \Large\textsc{Loo Wee Lun}\par}
        \vfill
        \em Taken during AY 25/26 at\par
        {\em School of Physical and Mathematical Sciences} \par
        {\em Nanyang Technological University}
\end{titlepage}

\tableofcontents

\newpage
\setcounter{section}{-1}
\section{Preface and Disclaimer}
This note is written by me, a student who taken the course MH4930: Special Topics in Mathematics under Dr. Lim Kay Jin during academic year 25/26 at the School of Physical and Mathematical Sciences of Nanyang Technological University in Singapore. 

I hereby disclaim that contents this note is only for my own use, and the note are not originally produced by me, except for the presentation of the written proofs. I do not take responsible for any grammatical and mathematical errors in the note. The note might or might not be complete, and, once completing the course, I will not update the note due to any reason.


\newpage
\section{Module Theory}
\subsection{Definition and Basic Theory of Modules}
\begin{defn} [Module]
    Let $R$ be a ring (might be unital or not). A left $R$-module $M$ is an abelian group $\br{M,+}$ with binary operation $\cdot:R\times M\to M, \br{r,m}\mapsto r\cdot m$ such that for every $r,s\in R$ and $m,n\in M$ we have
    \begin{enumerate}
        \item $\br{r+s}\cdot m = r\cdot m + s\cdot m$
        \item $r\cdot \br{s\cdot m} = \br{rs}\cdot m$
        \item $r\cdot \br{m+n} = r\cdot m + r\cdot n$
    \end{enumerate}
    Additionally, if $R$ is unital, then we want $1\cdot m = m$.
\end{defn}
One can think of module analogous to "ring action". 

\medskip

\begin{re}
    One can define a right $R$-module in a similar manner. However, the existence of a left module does not necessarily imply the existence of a corresponding right module, where the usual obstruction is the second criteria. Despite that, we describe a general procedure in constructing a right module from a left module. 

    Suppose $\br{R,+,\star}$ is a ring. Let $(R^{\op},+,*)$ be a ring where $R^{\op}$ is the same set as $R$ but the operation $*$ is defined as $a*b := b\star a$. Then any left $R$-module $(M,\cdot)$ is a right $R^{\op}$ module with operation $\cdot_{\op}$ defined as $m\cdot_{\op} r:= r\cdot m$ and vice versa.
\end{re}

\medskip

\begin{re}
    If $R$ is a commutative ring, then any left $R$-module is a right $R$-module via the the binary operation $m*r := r\cdot m$.
\end{re}

\medskip

\textbf{From now onward, all mentioned module is a left module unless otherwise specified.}

\medskip

\begin{defn} [Sub-module]
    Let $M$ be an $R$-module and $N\subseteq M$. We say that $N$ is a sub-module of $M$ if $N$ is also an $R$-module under the same action.
\end{defn}

\medskip

\begin{re}
    If $R$ is a field, then an $R$-module is a vector space over $F$. Naturally, a sub-module over a field is a subspace. Thus, the idea of module can be interpreted as a generalization of the theory of vector spaces.
\end{re}

\medskip

\begin{pro}
    Let $M$ be a group. We have $M$ is abelian if and only if $M$ is a $\Z$-module.
\end{pro}
\begin{proof}
    $\br{\impliedby}$. This is trivial, since by definition a module must be abelian.

    $\br{\implies}$. For any $n\in \Z$ and $m\in M$, define the operation where
    \[n\cdot m 
    \begin{cases}
        \underbrace{m+\dots +m}_{n \text{ times}} &, n\geq 0\\
        \underbrace{(-m)+\dots +(-m)}_{-n \text{ times}} &, n< 0
    \end{cases}\]
    Then by verifying the axioms (which are omitted here), one can show that $M$ is indeed a $\Z$-module.
\end{proof}

\medskip

\begin{ex}
    Here, we provide some examples of modules.
    \begin{enumerate}
        \item For any ring $R$, the trivial module is defined to be $M:=\sbr{0}$ where $r\cdot 0:= 0$ for any $r\in R$
        \item For any ring $R$, the regular module is defined to be $M=R$ where $r\cdot m:= rm$. To distinguish $R$ as a ring and as a module, we use $_RR$ to denote the ring $R$ being a regular module.
        \item For any unital ring $R$, the free $R$-module is defined to be $M:= R^n$ where $r\cdot \br{v_1, \dots, v_n} := \br{rv_1, \dots, rv_n}$
        \item Let $M:= \R$, then
        \begin{enumerate}
            \item if $R=\R$, then it is a regular module.
            \item if $R=\Q$, then it is a infinite dimensional vector space.
            \item if $R=\Z$, then it is viewed as an abelian group.
        \end{enumerate}
        \item (Restriction of scalars). Let $\varphi:R\to S$ be a ring homomorphism and $\br{M,\cdot}$ be an $S$-module. Then $M$ is an $R$-module via $r\star m := \varphi\br{r}\cdot m$.
        \item Let $\br{N, \cdot}$ be an $R$-module. The annihilator of $N$ is defined to be
        \[\ann_R\br{N}=\sbr{r\in R:r\cdot n =0\  \forall n\in N }\]
        Suppose that $\pi:R\to S$ is a ring epimorphism such that $\ker \pi\subseteq \ann_R\br{N}$. Then $N$ is an $S$-module via $s\star n= r\cdot n$ where $\pi\br{r}=s$.
        \item Let $R= \mathbb M_{n\times n}\br{F}$ where $F$ is some field and $V = \mathbb M_{n\times 1}\br{F}$. Then $V$ is a $R$-module via left multiplication as the binary operation. We say $V$ is the natural module over the matrix ring $R$.
    \end{enumerate}
\end{ex}

\medskip

\begin{re}
    The sub-module of a regular module corresponds to left ideal.
\end{re}

\medskip

\textbf{From now onward, all rings are unital unless otherwise specified.}

\begin{pro} \label{pro: 0 and -1 act}
    Let $M$ be an $R$-module, and $x\in N$, $r\in R$. Then we have
    \begin{itemize}
        \item $0_R\cdot x=0_M$
        \item $-1_R\cdot x = -x $
    \end{itemize}
\end{pro}
\begin{proof}
    For the first statement, note that
    \[r\cdot m = (0_R+r)\cdot m = 0_R\cdot m + r\cdot m\implies 0_R\cdot m = r\cdot m - r\cdot m = 0_M\]
    For the second statement, note that
    \[0_M = 0_R\cdot x = (1_R + (-1_R))\cdot x = 1_R\cdot x + (-1_R\cdot x) = x+(-1_R\cdot x)\]
    The statement follows by moving $x$ from LHS to RHS.
\end{proof}

\begin{pro} [Sub-module criterion] \label{pro: submod cri}
    Let $M$ be a $R$-module and $N\subseteq M$. We have that $N$ is a sub-module of $M$ if and only if $N$ is non-empty and $x+r\cdot y\in N$ for any $x,y\in N$ and $r\in R$.
\end{pro}
\begin{proof}
    $\br{\implies}$. If $N$ is a sub-module of $M$, then $N$ must not be empty since it must contain the identity element. Moreover, since $N$ is a sub-module, thus for any $y\in N$ and $r\in R$ we must have $r\cdot y\in N$ by closure. It is then obvious that $x+r\cdot y\in N$ for any $x,y\in N$ and $r\in R$.

    $\br{\impliedby}$. Suppose that $N\subseteq M$ is non-empty $x+r\cdot y\in N$ for any $x,y\in N$ and $r\in R$. First note by taking $r=-1$ we see that for any $x,y\in N$ we have $x-y\in N$, thus by the subgroup criterion we see that $N$ is a subgroup of $M$. Next, since $N$ is non-empty, let $n\in N$ and take $r=-1$. By Proposition \ref{pro: 0 and -1 act} we see that 
    \[0_M = n -n = n+(-1)\cdot n \in R\]
    Finally, by taking $r=0_R$, we $x=0_M$ we see that for any $r\in R$ and $y\in N$ we have $r\cdot y\in N$, which establish the closure. This completes the proof.
\end{proof}
\begin{re}
    Note that Proposition \ref{pro: submod cri} can only be used for unital rings. For non-unital rings, we can only prove sub-module via showing that the axioms are true.
\end{re}

\newpage

\subsection{Algebras and Module Homomorphisms}

\begin{defn} [$R$-Algebra]
    Let $R$ be a commutative (unital) ring. An $R$-algebra $A$ is a (unital) ring with ring homomorphism $\varphi:R\to A$ such that $\varphi\br{1_R}=1_A$ and $\varphi\br{R}\subset Z\br{A}$, where $Z\br{A}$ is the center of the multiplicative group of $A$.
\end{defn}

\medskip

\begin{ex}
    Let $A=R[X]$ where $R$ is any ID or even a field. Define the ring homomorphism $\varphi:R\to R[X],\ r\mapsto r$. Then $R[X]$ is an $R$-algebra.
\end{ex}

\medskip

\begin{pro}
    $R$-algebra $A$ is a $R$-module via the binary operation $r\cdot a:= \varphi\br{r}a$ where $\varphi$ is the ring homomorphism that embeds $R$ to the center of $A$.
\end{pro}
\begin{proof}
    We just have to verify the axioms of modules: for any $r,s\in R$ and $a,b\in A$ we have
    \begin{enumerate}
        \item $\br{r+s}\cdot a = \varphi\br{r+s}a = \br{\varphi\br{r}+\varphi\br{s}}a = \varphi\br{r}a + \varphi\br{s}a = r\cdot a + s\cdot a$.
        \item $r\cdot \br{s\cdot a} = \varphi(r)\varphi\br{s}a = \varphi\br{rs}a = \br{rs}\cdot a$
        \item $r\cdot \br{a+b} = \varphi\br{r}\br{a+b}=\varphi\br{r}a + \varphi\br{r}b = r\cdot a + r\cdot b$
        \item $1_R\cdot a = \varphi\br{1_R}a = 1_Aa = a$.
    \end{enumerate}
    This completes the proof.
\end{proof}

\medskip

\begin{defn} [Algebra homomorphism]
    Let $\br{A,\cdot}$ and $\br{B,\star}$ be $R$-algebra and $\varphi:A\to B$ be a ring homomorphism. We then say $\varphi$ is an algebra homomorphism from $A$ to $B$ if $\varphi\br{1_A}=1_B$ and $\varphi\br{r\cdot a}=r\star \varphi(a)$. An algebra isomorphism is then a bijective algebra homomorphism.
\end{defn}

\medskip

\begin{ex} [Group algebra] \label{ex: group algebra}
    Let $G$ be a group and $R$ be a commutative ring. Define $RG$ (or sometimes $R[G]$) to be the set
    \[RG:= \sbr{\text{formal finite sum of the form }\sum_{g\in G}r_g g \text{ where }r_g\in R}\]
    together with the operating rules
    \[\sum r_g g + \sum s_g g = \sum\br{r_g+s_g}g\]
    and 
    \[\br{\sum r_g g}\br{\sum s_g g} = \sum t_g g,\ t_g := \sum_{h\in G}r_{gh}s_{h^{-1}}\]
    Then $RG$ is an $R$-algebra and is called the group algebra of $G$ over $R$.
\end{ex}

\medskip

\begin{defn} [Module homomorphism]
    Let $V,W$ be $R$-module. The map $\varphi:V\to W$ is a $R$-module homomorphism if it is a group homomorphism and satisfies $\varphi\br{r\cdot v} = r\cdot \varphi\br{v}$ for all $r\in R$ and $v\in V$. An $R$-module isomorphism is then an $R$-module homomorphism which is also a group isomorphism.
\end{defn}

\medskip

\begin{ex} [Kernel and image]
    Let $\varphi:V\to W$ be $R$-module homomorphism. Then we define
    \[\ker\varphi := \sbr{v\in V:\varphi\br{v}=0}\]
    \[\im \varphi := \sbr{\varphi\br{v}\in W:v\in V}\]
    Then $\ker \varphi$ and $\im\varphi$ is a sub-module of $V$ and $W$ respectively.
\end{ex}

\medskip


\begin{ex}
    Let $V,W$ be $R$-module. The hom set from $V$ to $W$ over $R$ is defined to be
    \[\h_R\sbr{V,W}:=\sbr{\text{$R$-module homomorphism from $V$ to $W$}}\]
\end{ex}

\medskip


\begin{ex}
    \hfill
    
    \begin{enumerate}
        \item Let $R:= \R[x]$ and define $\varphi:R\to R$ where $\sum a_ix^i\mapsto \sum a_ix^{2i}$. Then $\varphi$ is a ring homomorphism but not $R$-module homomorphism. If not, then it must satisfy $\varphi\br{x\cdot 1}=x$ but $\varphi\br{x\cdot 1}=\varphi\br{x}=x^2$.
        \item Let $\pi_i:R^n \to R$ where $\br{r_1,\dots,r_n}\mapsto r_i$. Then $\pi_i$ is an $R$-module homomorphism since
        \[\pi_i\br{r\cdot\br{r_1,\dots, r_n}}=\pi_i\br{rr_1, \dots, rr_n} = rr_i = r\cdot \pi_i\br{r_1, \dots, r_n}\]
        A partial converse is as follow: let $\tau_i:R\to R^n$ where $x\mapsto \br{0,\dots, 0,x, 0,\dots, 0}$ where $x$ is at the $i$-th position. Then $\tau$ is an $R$-module homomorphism.
        \item For $V,W$ are $R$-modules, the trivial map is defined to be the $R$-module homomorphism $\varphi:V\to W$ where $v\mapsto 0_W$.
        \item If $R$ is a field, then $R$-module homomorphism is equivalent to linear transformation.
        \item If $R=\Z$, then $R$-module homomorphism is equivalent to abelian group homomorphism.
    \end{enumerate}
\end{ex}

\medskip

\begin{pro}
    Let $U,V,W$ be $R$-module. We have the following
    \begin{enumerate}
        \item $\varphi:U\to V$ is an $R$-module homomorphism $\iff$ $\varphi\br{rx+y}=r\varphi\br{x}+\varphi\br{y}$ for all $r\in R$ and $x,y\in U$
        \item $\h_R\br{U,V}$ is an abelian group where for $\varphi,\psi\in \h_R\br{U,V}$ we define $\br{\varphi+\psi}\br{u}=\varphi\br{u}+\psi\br{u}$ for all $u\in U$. Moreover, if $R$ is commutative, then $\h_R\br{U,V}$ is an $R$-module with $\br{r\cdot \varphi}\br{u}=\varphi\br{ru}$.
        \item If $\varphi\in \h_R\br{U,V}$ and $\psi\in \h_R\br{V,W}$, then $\psi\circ\varphi\in \h_R\br{U,W}$.
        \item $\operatorname{End}_R\br{U}:= \h_R\br{U,U}$ is a unital ring with multiplicative operation defined to be the composition, i.e. $\varphi\circ \psi$. Moreover, if $R$ is commutative, then $\operatorname{End}_R\br{U}$ is an $R$-algebra.
    \end{enumerate}
\end{pro}
\begin{proof}
    \hfill
    \begin{enumerate}
        \item The forward direction simply follows from the definition, thus omitted. For the backward direction, take $r=1$ we obtain $\varphi\br{x+y}=\varphi(x) + \varphi(y)$, showing that it is a homomorphism. Take $y=0$ we get $\varphi\br{rx}=r\varphi\br{x}$, showing that it a $R$-module homomorphism.
        \item Tutorial question.
        \item $\br{\psi\circ\varphi}\br{ru} = \func{\psi}{\func{\varphi}{ru}}=\func{\psi}{r\func{\varphi}{u}}=r\func{\psi}{\func{\varphi}{u}}=r\func{\br{\psi\circ\varphi}}{u}$
        \item We have to prove that it is a unital ring. Let $\varphi,\alpha,\beta,\gamma\in \operatorname{End}_R\br{U}$,
        \begin{itemize}
            \item Define map $\mathds 1:U\to U$ be the identity map. Clearly $\varphi\circ \mathds 1 = \mathds 1\circ \varphi = \varphi$.
            \item $\br{\alpha+\beta}\circ\gamma(u) = \alpha\br{\gamma(u)}+\beta\br{\gamma(u)} = \br{\alpha\circ\gamma} (u) + \br{\beta\circ\gamma}(u)$. This shows that $\br{\alpha + \beta}\circ \gamma = \alpha\circ \gamma + \beta \circ \gamma$
            \item Similarly for $\alpha\circ\br{\beta +\gamma} = \alpha \circ \beta + \alpha\circ \gamma$
        \end{itemize}
        To show that it is an $R$-algebra when $R$ is commutative, define $f:R\to \operatorname{End}_R\br{U}$ where $r\mapsto r\mathds 1$ where $r\mathds 1:u\mapsto ru$. It is clear that $r\mathds1 = r\cdot \mathds 1$. We now show that it is an algebra:
        \begin{itemize}
            \item $(r\mathds1)\cdot \func{\varphi}{u} = (r\mathds1)(\varphi(u)) = r\varphi(u) = \varphi(ru) = \varphi\circ(r\mathds1)(u)$. This shows that $(r\mathds1)\circ\varphi = \varphi \circ (r\mathds1)$
            \item $\func{f}{r+s} = (r+s)\mathds1 = r\mathds1 + s\mathds1 = \func{f}{r} + \func{f}{s}$
        \end{itemize}
    \end{enumerate}
    This completes the proof.
\end{proof}

\medskip

\begin{re}
    Let $R,S$ be rings. An $\br{R,S}$-bimodule $_RM_S=\br{_RM,M_S}$ where $\br{rm}s = r\br{ms}$. Suppose we have $_RM_S$ and $_RN$. Then $\h_R\br{M,N}$ is an $S$-module where $\br{s\cdot \varphi}\br{m}:=\varphi\br{ms}$.
\end{re}

\medskip

\begin{ex}
    Let $G$ be a group and $F$ be a field. Consider $R=FG$ and let $M$ be a left $R$-module. Then the right action defined to be $m*g := g^{-1}m$ makes it a right module.
\end{ex}

\medskip

\begin{pro}
    Let $N\subseteq M$ be $R$-modules. Then $M/N$ is an $R$-module where $r\cdot \br{m+N}:= rm+N$. We have a canonical surjective $R$-module homomorphism $\pi:M\to M/N$.
\end{pro}

\medskip 

\begin{ex}
    Let $V_1,\dots, V_m$ be submodules of $V$.
    \begin{enumerate}
        \item $V_1+\dots + V_m = \sbr{v_1\pd v_m :v_i\in V_i}$ is a submodule of $V$.
        \item Let $A\subseteq V$. We define
        \[\langle A\rangle := RA = \sbr{r_1a_1 \pd r_na_n:r_1,\dots ,r_n \in R,\ a_1, \dots, a_n \in A,n\in \Z^+}\]
        Then we say $\langle A \rangle$ is the submodule generated by $A$, and it is the smallest sub-module of $V$ containing $A$.

        \medskip
        
        It is clear that $R\emptyset = \sbr{0}$. Also, if $A=U$ is a submodule of $V$, then $RU = U$.
    \end{enumerate}
\end{ex}

\medskip

\begin{defn} [Finitely generated]
    Let $U$ be a submodule of $V$. We say that $U$ is finitely generated as an $R$-module if there exists a finite set $A\subseteq U$ such that $U=RA$.
\end{defn}

\medskip

\begin{defn} [Cyclic]
    Let $U$ be a submodule of $V$. We say that $U$ is cyclic if $U=RA$ where $A=\sbr{a}\subseteq U$ only contains one element.
\end{defn}

\medskip

\begin{defn}[Minimal generating set]
    Let $V$ be a finitely generated module. By definition there exists a non-negative integer $d$ such that $V=RA$ with $|A|=d$. Then we say that $A$ is the minimal generating set of $V$.
\end{defn}

\medskip

\begin{re}
    In linear algebra, every vector space has a unique dimension. This means that any basis of the fixed vector space has the same cardinality. However, this is not true for the case of module. Therefore, there exists finitely generated module such that its two generating set has different cardinality.
\end{re}

\medskip

\begin{ex}
    \hfill
    \begin{enumerate}
        \item The generating set of $\Z$-module is equivalent to the generating set as abelian group.
        \item The cyclic sub-module of a regular module $_RR$ is equivalent to a principal left ideal of $R$. Moreover, if $R$ is a PID, then we have the following chain of equivalence:
        \[\text{submodules of $_RR$} \equiv \text{cyclic submodules} \equiv \text{ideals} \equiv \text{principal ideals}\]
        \item The submodule of a finitely generated module need not be finitely generated. For example, let $F$ be a field and define $R=F[X_1,X_2,\dots]$ and $U=\sbr{f\in R:\deg f\ge 1}$. Note that $_RR=R1$ is finitely generated as a regular module. However $U$ is not finitely generated. If not, then there exists $A\subseteq U$ such that $U=RA$ where $|A|<\infty$. But the finiteness of $A$ implies that only finite number of polynomial are chosen, and thus only finite number of variables are involve, contradicting to the fact that there are infinitely many variables in $U$, since $\sbr{X_1,X_2, \dots}\in U$.
        \item Let $V=R^n$, then $\Omega:= \sbr{e_i=\br{0,\dots,0,1,0,\dots,0}:1\leq i\leq n}$ is a generating set of $V$. Additionally, if $R$ is commutative, then $\Omega$ is the minimal generating set.
    \end{enumerate}
\end{ex}

\medskip

\begin{defn} [Invariant basis number property]
    Let $R$ be a ring. We say $R$ has the invariant basis number (IBN) property if every finitely generated $R$-module has a well-defined rank, i.e. any generating set of a finitely generated $R$-module has the same cardinality.
\end{defn}

\medskip

\begin{thm} [Isomorphism Theorems of Modules]
    \hfill
    \begin{enumerate}
        \item Let $\varphi:M\to N$ be $R$-module homomorphisms. Then $M/\ker\varphi \cong \im\varphi$
        \item Let $U,V$ be submodules of $W$. Then $(U+V)/V \cong U/(U\cap V)$.
        \item Let $U\subseteq V\subseteq W$ be $R$-modules. Then $W/V = (W/U) /(V/U)$.
        \item Let $U\subseteq V$ be $R$-modules. Then there exists an one-to-one correspondence between the following sets:
        \[\sbr{\text{Submodules of $V$ containing U}}\longleftrightarrow \sbr{\text{submodules of $V/U$}}\]
        The correspondence is given by $W\mapsto \func{\pi}{W}$ where $\pi$ is the canonical map.
    \end{enumerate}
\end{thm}

\medskip
\begin{pro} \label{pro: indi sum}
    Let $N_1, \dots, N_m$ be sub-modules of an $R$-module $M$. TFAE:
    \begin{enumerate}
        \item The map $\pi:N_1 \times \dots \times N_m \to N_1 \pd N_m$ where $\br{x_1, \dots, x_m}\mapsto x_1 \pd x_m$ is an isomorphism of $R$-module.
        \item For every $j\in \sbr{1,\dots, m}$ we have 
        \[N_j \cap \sum_{i\neq j}N_i = \sbr{0}\]
        \item For any $x\in N_1\pd N_m$, there exists a unique $x_i\in N_i$ for every $i$ such that $x=x_1 \pd x_m$.
    \end{enumerate}
\end{pro}
\begin{proof}
    $[1\implies 2]$. Suppose not, say there exists non-zero element $x_j\in N_j \cap \sum_{i\neq j}N_i$. So we can express 
    \[x_j=x_1 \pd x_{j-1}+x_{j+1}\pd x_m \neq 0\]
    where $x_i\in N_i$ for all $i=1\many j-1,\ j+1\many m$. This implies that 
    \[\pi\br{\mathbf{0}, x_j, \mathbf{0}} = \pi\br{x_1 \many x_{j-1},\ x_{j+1}\many x_m}\implies \pi\br{x_1 \many x_{j-1}, -x_j,\ x_{j+1}\many x_m}=0\]
    So $\br{x_1 \many x_{j-1}, -x_j,\ x_{j+1}\many x_m}\in \ker \pi$. By assumption $\pi$ is isomorphism, so it is injective and has trivial kernel, indicating that $\br{x_1 \many x_{j-1}, -x_j,\ x_{j+1}\many x_m}=(0\many 0)$ and thus $x_j=0$, which is a contradiction.

    $[2\implies 3]$. Let $x_i\in N_i$ and $y_i\in N_i$ be chosen for all $i$ such that
    \[\sum_{i=1}^m x_i = \sum_{i=1}^m y_i\]
    Note that for any $j$ we have
    \[N_j \ni x_j - y_j = \sum_{i\neq j}\br{y_i-x_i}\in \sum_{i\neq j}N_i\]
    And so we see that $x_j-y_j$ is a common element of 
    \[N_j \cap \sum_{i\neq j}N_i = \sbr{0}\]
    and thus implying that $x_j-y_j = 0$, so $x_j = y_j$. Since $j$ is chosen to be arbitrary, we can repeat the procedure and thus showing that there is a unique representation.

    $[3 \implies 1]$. Define $\pi: N_1 \times \dots \times N_m\to N_1 \pd N_m$ such that $\br{x_1, \dots, x_m}\mapsto x_1 \pd x_m$. It is easy to verify that $\pi$ is a $R$-module homomorphism and is surjective. For injectivity, suppose that 
    \[\pi\br{x_1, \dots, x_m}=\pi\br{y_1, \dots, y_m}\]
    and so $x_1\pd x_m = y_1 \pd y_m$. By assumption, the representation is unique, and thus we have $x_i = y_i$ for all $i=1, 2, \dots, m$. 
\end{proof}

\begin{re} [Internal direct sum]
    In any cases in Proposition \ref{pro: indi sum}, we say that $\sum_{i=1}^m N_i$ is a (internal) direct sum and we denote it by 
    \[\bigoplus_{i=1}^m N_i\]
\end{re}

\newpage
\subsection{Free Module and Tensor Product}
\begin{defn} [Free on a subset]
    Let $F$ be an $R$-module. We say that $F$ is free on a subset $A$ of $F$ if for every $x\in F$ there exist unique non-zero elements $r_1, \dots, r_m\in R$ and unique choice of $a_1, \dots, a_m$ such that 
    \[x= r_1a_1 \pd r_m a_m\]
    If so, we say that $A$ is a (free) basis of $F$. Also, when $R$ is unital and commutative, the cardinality $|A|$ of $A$ is well-defined, and we define the rank of $F$ be $|A|$.
\end{defn}

\medskip

\begin{ex}
    \hfill
    \begin{enumerate}
        \item $_RR$ is free on $\sbr{1}$ since for every $r\in R$ we have $r=r\cdot 1$.
        \item $\bigoplus_RR$ is free on the 'standard free basis' $\sbr{\mathbf{e}_i = \br{\mathbf{0}, 1_i, \mathbf{0}}}_i$.
        \item Let $R=\Z$. Then the $\Z$-module $\Z/2\Z$ is not free on $\sbr{\bar{1}}$ since $\bar{0}=1\cdot \bar{0}=2\cdot \bar{0}=4\cdot \bar{0}$.
    \end{enumerate}
\end{ex}

\medskip

\begin{defn} [Free on a set]
    Let $A$ be a set and $F$ be an $R$-module. We say that $F$ is free on $A$ if there exists an injective map $\iota: A\to F$ such that for any $R$-module $M$ and map of set $\varphi:A\to M$, there exists a unique $R$-module homomorphism $\Phi:F\to M$ such that the the following diagram commutes:
    \[\begin{tikzcd}
	   A && F \\
	   \\
	   && M
	   \arrow["\iota", hook, from=1-1, to=1-3]
	   \arrow["\varphi"', from=1-1, to=3-3]
	   \arrow["{\exists! \Phi}", dashed, from=1-3, to=3-3]
    \end{tikzcd}\]
\end{defn}

\medskip

\begin{lem} [Universal property of free modules]
    If $F$ is a free $R$-module on $A$ a subset of $F$, then $F$ is free on set $A$ where the map $\iota:A \hookrightarrow F$ is the inclusion map.
\end{lem}
\begin{proof}
    Let $M$ be an $R$-module and $\varphi:A\to M$ is a given map of set. Suppose that $F$ is free on a subset $A\subseteq F$, which means that any $x\in F$ has unique representation with respect to $A$. Write $x= r_1a_1 \pd r_m a_m$ to denote its unique representation.

    We define a map $\Phi:F\to M$ such that $x=r_1a_1 \pd r_ma_m \mapsto r_1\varphi(a_1) \pd r_m\varphi(a_m)$. We claim that $\Phi$ is an $R$-module homomorphism. Let $y=r'_1a'_1 \pd r'_ma'_m$ be an element of $F$. So $x+y = r_1a_1 \pd r_ma_m + r'_1a'_1 \pd r'_ma'_m$ and thus
    \[\Phi(x+y) = r_1\varphi(a_1) \pd r_m\varphi(a_m) + r'_1\varphi(a'_1) \pd r'_m\varphi(a'_m) = \Phi(x) + \Phi(y)\]
    Also, let $r\in R$, and so $rx = rr_1 a_1 \pd rr_ma_m$, we have
    \[\Phi(rx) = rr_1 \varphi(a_1)\pd rr_m \varphi(a_m) = r(r_1 \varphi(a_1)\pd r_m \varphi(a_m)) = r\varphi(x)\]
    This shows that $\Phi$ is indeed an $R$-module homomorphism. 
    
    We now check if commutativity holds, i.e. $\Phi\circ \iota = \varphi$. Let $a\in A\subseteq F$, by definition $\iota(a) = a\in F$. Since $F$ is free on the subset $A$, so the unique representation of $a$ is $a$. Thus $(\Phi\circ\iota)(a) = \Phi(a) = \varphi(a)$, thus commutativity holds.

    To check uniqueness, suppose that $\Psi:F\to M$ is an $R$-module homomorphism such that $\Psi\circ \iota = \varphi$. But we know that $\Psi\circ \iota = \varphi$, so $\Psi\circ \iota = \Phi \circ \iota$, implying that $\Psi = \Phi$ on $A\subseteq F$. But $F$ is free on the subset $A$, so the equality of the $R$-module homomorphisms $\Psi$ and $\Phi$ can be extended to the whole $F$. Thus $\Psi = \Phi$, showing that $\Phi$ is indeed unique. This completes the proof.
\end{proof}

\begin{thm}
    Let $A$ be a set and $R$ be a ring. Define
    \[F(A):= \sbr{f:A\to R \mid  f(a)\neq 0 \text{ for finitely many }a\in A}\]
    Then $F(A)$ is free on set $A$ with group operation
    \[\br{f+g}\br{a} = \func{f}{a} + \func{g}{a}\]
    and ring action
    \[\func{\br{r\cdot f}}{a}:= \func{rf}{a}\]
    where $\iota:A\to \func{F}{A}$ is defined to be $a\mapsto \varepsilon_a$ where
    \[\varepsilon_a:b\mapsto 
    \begin{cases}
        0 & ,b\neq a\\
        1 & , b=a
    \end{cases}\]
\end{thm}
\begin{proof}
    It is clear that $F(A)$ with the defined group operation and ring action is an $R$-module. Before the proof, we claim that $F(A)$ is free on the subset $\iota(A)$. To see this, for any $f\in F(A)$ we claim that the unique representation of $f$ over $\iota(A)$ is 
    \[f(x) =\sum_{a\in A} f(a)\varepsilon_a(x)\]
    \begin{enumerate}
        \item We first show that the declared linear combination is true. Let $x=b\in A$, then 
        \[\sum_{a\in A}f(a) \varepsilon_a(b) = \sum_{\substack{a\in A\\ a\neq b}}f(a) \varepsilon_a(b) + f(b) \varepsilon_b(b) = 0 + f(b)=f(b)\]
        since $\varepsilon_a(b)=0$ for all $a\neq b$ and $\varepsilon_b(b) = 1$. This shows that the declared identity holds.
        \item We show that it is indeed unique. Suppose we can write $f$ into 
        \[f(x) = \sum_{a\in A}r_a \varepsilon_a(x)\]
        Then we have 
        \[ f(b) = \sum_{a\in A} r_a \varepsilon_a(b) = r_b\varepsilon_b(b) = r_b\]
        for any $b\in A$. Thus the declared representation is unique.
    \end{enumerate}
    This shows that $F(A)$ is free on the subset $\iota(A)$. 

    Next, suppose given $M$ is an $R$-module and $\varphi:A\to M$ is a map of sets. Define $\varphi':\iota(A)\to M$ where $\varepsilon_a\mapsto \varphi(a)$, then we have the following commutative diagram:
    \[\begin{tikzcd}
	   A && \iota(A) \\
	   \\
	   && M
	   \arrow["\iota", hook, from=1-1, to=1-3]
	   \arrow["\varphi"', from=1-1, to=3-3]
	   \arrow["{\varphi'}", from=1-3, to=3-3]
    \end{tikzcd}\]
    On the other hand, since $F(A)$ is free on the subset $\iota(A)$, so we have the following commutative diagram:
    \[\begin{tikzcd}
	   \iota(A) && F \\
	   \\
	   && M
	   \arrow["j", hook, from=1-1, to=1-3]
	   \arrow["\varphi'"', from=1-1, to=3-3]
	   \arrow["{\exists! g}", dashed, from=1-3, to=3-3]
    \end{tikzcd}\]
    where $j$ is the inclusion map, and the existence of $g$ is ensured by the universal property of free modules. Glueing the two obtained commutative diagram together we get
    \[\begin{tikzcd}
	   A && \iota(A) && F\\
	   \\
	   && M
	   \arrow["\iota", hook, from=1-1, to=1-3]
	   \arrow["\varphi"', from=1-1, to=3-3]
	   \arrow["{\varphi'}", from=1-3, to=3-3]
       \arrow["j", from=1-3, to=1-5]
       \arrow["g", dashed, from=1-5, to=3-3]
    \end{tikzcd}\]
    Since $\varphi = \varphi'\circ \iota$ and $\varphi' = g\circ j$, altogether we get $\varphi = g \circ (j\circ \iota)$. This shows that $F(A)$ is free on set $A$.
\end{proof}

\begin{cor}
    \hfill
    \begin{enumerate}
        \item Let $F_1$ and $F_2$ be $R$-modules free on a set $A$ with inclusion maps $\iota:A\to F_1$ and $j:A\to F_2$. Then there exists a unique isomorphism $\Phi:F_1\to F_2$ such that $\Phi\circ\iota = j$.
        \item If $F$ is an $R$-module free on $A$, then $F\cong F(A)$.
    \end{enumerate}
\end{cor}
\begin{proof}
    Let $F_1$ and $F_2$ be $R$-modules free on a set $A$ with inclusion maps $\iota:A\to F_1$ and $j:A\to F_2$. Consider the following commutative diagram 
    \[\begin{tikzcd}
	   A && F_1 \\
	   \\
	   && F_2
	   \arrow["\iota", hook, from=1-1, to=1-3]
	   \arrow["j"', from=1-1, to=3-3]
	   \arrow["{\exists! \Phi}", dashed, from=1-3, to=3-3]
    \end{tikzcd}\]
    where the existence of $\Phi$ is ensured by the universal property of free module. Similarly we have
    \[\begin{tikzcd}
	   A && F_2 \\
	   \\
	   && F_1
	   \arrow["j", hook, from=1-1, to=1-3]
	   \arrow["\iota"', from=1-1, to=3-3]
	   \arrow["{\exists! \Psi}", dashed, from=1-3, to=3-3]
    \end{tikzcd}\]
    Note that $j = \Phi \circ \iota$ and $\iota = \Psi \circ j$. We claim that $\Psi$ and $\Phi$ are isomorphisms pair of $F_1$ and $F_2$, that is, we show that $\Psi \circ \Phi = \id_{F_1}$ and $\Phi \circ \Psi = \id_{F_2}$. Simply note that 
    \[\Phi \circ \iota = j \implies \Psi\circ (\Phi \circ \iota) = \Psi \circ j = \iota \implies (\Psi \circ \Phi)(\iota(a)) =\iota(a)\quad  \forall a\in A\]
    This implies $\Psi\circ \Phi$ fixes $\iota(A)\subseteq F_1$. But since $F_1$ is free on set $A$, so it can be extend into whole $F_1$, thus $\Psi \circ \Phi = \id_{F_1}$. We can use the similar argument to show $\Phi \circ \Psi = \id_{F_2}$, and is thus omitted.

    We have proven that $F(A)$ is an $R$-module that is free on set $A$. If $F$ is an $R$-module free on $A$, it follows directly from the first statement that $F\cong F(A)$. This completes the proof.
\end{proof}

The following definition extends the notion of linear map from linear algebra into the realm of module:
\begin{defn} [$R$-balanced map]
    Let $_RN, M_R, L$ be abelian groups. Let $\beta:M\times N\to L$ be a map. We say that $\beta$ is an $R$-balanced map if it satisfies all the following for any $m, m'\in M$, $n, n'\in N$ and $r\in R$
    \begin{enumerate}
        \item $\beta\br{m+m',n}=\func{\beta}{m,n}+ \func{\beta}{m',n}$
        \item $\func{\beta}{m,n+n'} = \func{\beta}{m,n} + \func{\beta}{m,n'}$
        \item $\func{\beta}{mr,n} = \func{\beta}{m,rn}$
    \end{enumerate}
\end{defn}

\medskip

\begin{defn} [Tensor product]
    Let $M_R$ and $_RN$ be $R$-module. Let $\func{F}{M\times N}$ be the free $\Z$-module on the set $M\times N$. Let $H$ be a subgroup of $\func{F}{M\times N}$ generated by elements of the form:
    \begin{itemize}
        \item $\br{m+m', n}-\br{m,n}-\br{m', n}$
        \item $\br{m, n+n'}-\br{m,n}-\br{m, n'}$
        \item $\br{m\cdot r, n}-\br{m, r\cdot n}$
    \end{itemize}
    for all $m,m'\in M$, $n,n'\in N$ and $r\in R$.

    The tensor product of $M$ and $N$ with respect to $R$ is defined to be the quotient group
    \[M\otimes _R N:= \func{F}{M\times N}/H\]
    where the tensor product of two elements $m\in M$ and $n\in N$ is defined to be
    \[m\otimes n := \br{m,n}+H\]
    Then the map $\iota:M\times N \to M\otimes_R N$ where $\br{m,n}\mapsto m\otimes n$ is an $R$-balanced map.
\end{defn}

\medskip

\begin{thm} [Universal property of tensor product] \label{thm: uni prop tens}
    Let $M_R$ and $_RN$ be $R$-modules, and consider their tensor product $M\ten_R N$ with the map $\iota:M\times N\to M\ten_RN$ where $\br{m,n}\mapsto m\ten n$. Then
    \begin{enumerate}
        \item For every abelian group $L$ and every $R$-balanced map $\beta:M\times N\to L$, there exists a unique group homomorphism $\Phi: M\ten_R N\to L$ such that the following diagram commutes:
            \[\begin{tikzcd}
	           M\times N && M\ten_R N \\
	           \\
	           && L
	           \arrow["\iota", hook, from=1-1, to=1-3]
	           \arrow["\beta"', from=1-1, to=3-3]
	           \arrow["{\exists! \Phi}", dashed, from=1-3, to=3-3]
            \end{tikzcd}\]
        \item Conversely, for every abelian group homomorphism $\Phi:M\ten_R N\to L$, then the map $\beta:M\times N\to L$ where $\beta:= \Phi\circ \iota$ is $R$-balanced. In particular, we have a bijection between the following sets:
        \[\sbr{R \text{-balanced map where }\beta: M\times N\to L}\leftrightarrow \sbr{\text{group homomorphism } \Phi:M\ten_R N \to L}\]
    \end{enumerate}
\end{thm}
\begin{proof}
    Let $j:M\times N \to F(M\times N)$ be the inclusion map. Then the universal property of free modules implies the existence of $\zeta:F(M\times N)\to L$ such that $\zeta\circ j = \beta$:
    \[\begin{tikzcd} [sep=small]
	           M\times N && F(M\times N) \\
	           \\
	           && L
	           \arrow["j", hook, from=1-1, to=1-3]
	           \arrow["\beta"', from=1-1, to=3-3]
	           \arrow["{\exists! \zeta}", dashed, from=1-3, to=3-3]
    \end{tikzcd}\]
    Let $H$ be the subgroup of $F(M\times N)$ as defined in the definition of tensor product. We claim that $H\subseteq \ker \zeta$, specifically we show that all the generators are mapped to $0$ by $\zeta$:
    \begin{align*}
        \zeta((m+m', n)-(m,n)-(m',n))
        &= (\zeta\circ j)((m+m', n)-(m,n)-(m',n))\\
        &= \beta((m+m', n)-(m,n)-(m',n))\\
        &= 0
    \end{align*}
    where the last equality is because $\beta$ is an $R$-balanced maps. Similarly we can show for the other two forms of generators, and thus is omitted here. This shows that $H\subseteq \ker\zeta$.

    Thus $\zeta$ induces a group homomorphism $\Phi: F(M\times N)/H \to L$ such that $\Phi(m\ten n) := \zeta(m,n)$.
    \[\begin{tikzcd} [sep=small]
	           M\times N && F(M\times N)/H \\
	           \\
	           && L
	           \arrow["\iota", hook, from=1-1, to=1-3]
	           \arrow["\beta"', from=1-1, to=3-3]
	           \arrow["{\exists! \Phi}", dashed, from=1-3, to=3-3]
    \end{tikzcd}\]
    We check the commutativity:
    \[(\Phi\circ \iota)((m,n)) = \Phi(m\ten n) = \zeta((m,n)) = \zeta(j(m,n)) = (\zeta\circ j)(m,n) = \beta((m,n)) \]
    This shows that $\Phi\circ \iota = \beta$. Next we show that $\Phi$ is uniquely determined. Note that every element of $M\ten_R N$ takes the form $\sum(m_i\ten n_i)$. Then 
    \[\Phi\br{\sum (m_i \ten n_i)} = \sum\br{\Phi(m_i \ten n_i)} = \sum \Phi(\iota((m_i,n_i))) = \sum(\Phi\circ \iota)((m_i,n_i)) = \sum\beta((m_i,n_i))\]
    This shows that $\Phi$ is determined by $\beta$. If we have another group homomorphism $\Psi$ such that $\Psi\circ \iota = \beta$, then again we have
    \[\Psi\br{\sum(m_i\ten n_i)} = \sum\beta((m_i, n_i))\]
    which implies that $\Phi = \Psi$, showing $\Phi$ is indeed uniquely defined.

    For the second statement, suppose given an abelian group homomorphism $\Phi:M\ten_R N\to L$, we check that $\beta:= \Phi\circ \iota$ is $R$-balanced (which is omitted here, since it simply follows from $R$-balanceness of tensor product). For the correspondence, we claim that the declared map $\beta$ is a one-to-one correspondence to $\Phi$. Suppose not, then there exists $\Psi$ such that $\Psi \circ \iota = \beta$, but then
    \[\Psi \circ \iota = \beta = \Phi \circ \iota \implies \Psi(m\ten n) = \Phi(m\ten n) \forall (m,n)\in M\times N\]
    This shows that $\Psi = \Phi$.
\end{proof}

\begin{re}
    Note that the tensor product $M\ten_R N$ is defined as a quotient group, and thus any defined map on the tensor product must be examined to be well-defined. This is practically infeasible due to the complicated structure of the quotient group. This is where Theorem \ref{thm: uni prop tens} can come useful, since everything is already settled in the proof, and the only job remain for us to do is to prove that $\beta$ is an $R$-balanced map in order to apply this statement.
\end{re}

\medskip

\begin{defn} [Bimodule]
    Let $R$ and $S$ be rings. An $\br{R,S}$-bimodule $M$ is both $_RM$ and $M_S$ satisfying $(rm)s = r(ms)$, where the ring action notation is dropped for the sake of readability, for every $r,s\in R$ and $m\in M$. In this case, we denote $M$ as $_RM_S$.
\end{defn}

\medskip

\begin{ex}
    \hfill
    \begin{enumerate}
        \item Let $S, T$ be sub-rings of $R$. Then $_SR_T$ is a bimodule.
        \item Let $I$ be an ideal of $R$. Then $_{R/I}(R/I)_R$ is a bimodule.
        \item for every $R$-module $M$ where $R$ is commutative, the induced right action $m*r:= r\cdot m$ gives rise to bimodule $_RM_R$.
        \item Consider modules $_RY_S$ and $Z_S$. We have seen that $M:=\h_S\br{Y_S, Z_S}$ is an abelian group where $\func{\br{\alpha + \beta}}{y}:= \func{\alpha}{y} + \func{\beta}{y}$. Then $M$ is a $R$-module with ring action $\func{(\alpha\cdot r)}{y}:= \func{\alpha}{r\cdot y}$ 
        \item Consider $_RM$ be a $R$-module. If $S$ is contained in the center $\func{Z}{R}$, then we have bimodule $_RM_S$.
    \end{enumerate}
\end{ex}

\medskip

\begin{pro}
    If we have modules $_SM_R$ and $_RN$, then $M\ten_R N$ is a left $S$-module.
\end{pro}
\begin{proof}
    Define $\func{s}{m,n} = \br{sm,n}$ to be an action of $S$ on $\func{F}{M\times N}$. We need to show that $H\subseteq \func{F}{M\times N}$ is an $S$-submodule, which we will be omitted here. After showing that $H$ is an $S$-submodule, then by definition
    \[M \ten _R N = \func{F}{M\times N}/H\]
    and so $M\ten_R N$ is an $S$-module.
\end{proof}

\medskip
\begin{defn} [Extension of scalars]
    Suppose that $M$ is an $R$-module and $R$ is a subring of $S$. The extension of scalar from $R$ to $S$ on $M$ is defined to be $S\ten_R M$, which is an $S$-module by previous proposition.
\end{defn}

\medskip

\begin{thm} [Universal property of extension of scalar] \label{thm: uni prop ext sca}
     Let $\varphi: R\to S$ be a unital ring homomorphism and consider $_RN$. Define $j:N\to S\ten_R N$ where $n\mapsto 1\ten n$. For any $_SL$ and $R$-module homomorphism $\gamma:N\to L$, there exists a unique $S$-module homomorphism $\Phi:S\ten_R N\to L$ such that the following diagram commutes:
    \[\begin{tikzcd}
	N && {S\ten_R N} \\
	\\
	&& L
	\arrow["j", from=1-1, to=1-3]
	\arrow["\gamma"', from=1-1, to=3-3]
	\arrow["\Phi", dashed, from=1-3, to=3-3]
    \end{tikzcd}\]
\end{thm}
\begin{proof}
	We define:
	\begin{itemize}
		\item $\iota:S\times N \to S\ten_R N$ where $(s,n)\mapsto s\ten n$
		\item $\beta:S\times N\to L$ where $(s,n)\mapsto s\gamma(n)$.
	\end{itemize}
	We claim that $\beta$ is $R$-balanced: First
	\[\beta(s,n+n') = s\gamma(n+n') = s\gamma(n) + s\gamma(n') = \beta(s,n) + \beta(s,n')\]
	Then:
	\[\beta(s+s',n) = (s+s')\gamma(n) = s\gamma(n) + s'\gamma(n) = \beta(s,n) + \beta(s',N)\]
	Next, note that since $R$ is a subring of $S$, we see $S$ as an $R$-module by $s\cdot r:= s\varphi(r)$. We have
	\[\beta(s\cdot r, n) = (s\cdot r)\gamma(n) = (s\varphi(r))\gamma(n)=s(\varphi(r)\gamma(n)) = s(r*\gamma(n)) = \beta(s, r*\gamma(n))\]
	So $\beta$ is $R$-balanced. By the universal property of tensor product there exists a unique group homomorphism $\Phi:S\ten_R N\to L$ where $\Phi\circ \iota = \beta$.
	\[\begin{tikzcd} [sep=small]
	S\times N && {S\ten_R N} \\
	\\
	&& L
	\arrow["\iota", from=1-1, to=1-3]
	\arrow["\beta"', from=1-1, to=3-3]
	\arrow["\Phi", dashed, from=1-3, to=3-3]
    \end{tikzcd}\]
	We claim that $\Phi$ is an $S$-module homomorphism:
	\[\Phi(s(s'\ten n)) = \Phi(ss'\ten n) = ss'\gamma(n) = s(s'\gamma(n)) = s\Phi(s'\ten n)\]
	so $\Phi$ is indeed an $S$-module homomorphism. Finally, we claim that $\Phi$ is the required $S$-module homomorphism such that $\Phi\circ j = \gamma$:
	\[(\Phi\circ j)(n) = \Phi(j(n)) = \Phi(1\ten n) = 1\cdot \gamma(n) = \gamma(n)\]
	Thus $\Phi\circ j = \gamma$, this completes the proof.
\end{proof}
The following corollary implies that the kernel is the obstruction for $N$ to be embedded in an $S$-module:
\begin{cor}
    Let $\varphi:R\to S$ be an inclusion map. Let $j:N\to S \ten_R N$ where $n\mapsto 1\ten n$. Then $N/\ker j$ is the unique largest quotient of $N$ such that it can be embedded into an $S$-module. In particular, $N$ can be embedded into an $S$-module if $\ker j$ is trivial.
\end{cor}
\begin{proof}
    We first show that $j$ is an $R$-module homomorphism: firstly
	\[j(n+n') = 1\ten (n+n') = 1\ten n + 1\ten n' = j(n) + j(n')\]
	Secondly:
	\[j(rn) = 1\ten (rn) = (1\cdot r)\ten n = \varphi(r) \ten n = r(1\ten n) = r j(n)\]
	So $j$ is an $R$-module homomorphism. Next, consider $K\subseteq N$ such that $\gamma:N/K \to L$ is an injective $R$-module homomorphism, that is, we are embedding quotient of $N$ into an $R$-module. Define
	\begin{itemize}
		\item $\pi:N\to N/K$ be the canonical surjection.
		\item $\beta:N\to L$ defined by $\beta = \gamma\circ \pi$. Note that $\ker \beta = K$.
	\end{itemize}
	Since $\gamma$ and $\pi$ are $R$-module homomorphism, so is $\beta$. By the universal property of extension of scalar, there exists a unique $S$-module homomorphism $\Phi:S\ten_R N\to L$ such that $\Phi \circ j = \beta$:
	\[\begin{tikzcd} [sep=small]
	 N && {S\ten_R N} \\
	\\
	&& L
	\arrow["j", from=1-1, to=1-3]
	\arrow["\beta"', from=1-1, to=3-3]
	\arrow["\Phi", dashed, from=1-3, to=3-3]
    \end{tikzcd}\]
	Now let $x\in \ker j$, so $j(x) = 0$, and note that 
	\[\beta(x) = (\Phi \circ j)(x) = \Phi(j(x)) = \Phi(0) = 0\]
	and so $x\in \ker \beta$. This shows that $\ker j\subseteq \ker \beta = K$, thus $N/K \subseteq N/\ker \beta \subseteq N/\ker j$. This shows that $N/\ker j$ is the largest possible quotient of $N$ that can be embedded into an $S$-module. And since $j$ is given, so it must be unique. This completes the proof.
\end{proof}

\begin{ex}
    \hfill
    \begin{enumerate}
        \item Consider $_RN$ and we claim that $R\ten_R N \cong N$. To see this, let $\varphi:R\to R$ where $r\mapsto r$ and define $\iota:N\to R\ten_R N$ where $n\mapsto 1\ten n$. It is clear that $\varphi$ is a $R$-module homomorphism. We thus have the following diagram
        \[\begin{tikzcd}[sep=small]
	n & N && {R\ten_R N} & {1\ten n} \\
	\\
	&&& N
	\arrow["{\in }"{description}, draw=none, from=1-1, to=1-2]
	\arrow[curve={height=-24pt}, maps to, from=1-1, to=1-5]
	\arrow["\iota", from=1-2, to=1-4]
	\arrow["\varphi"', from=1-2, to=3-4]
	\arrow["\ni"{description}, draw=none, from=1-4, to=1-5]
	\arrow["\Phi", dashed, from=1-4, to=3-4]
\end{tikzcd}\]
    where by Theorem \ref{thm: uni prop ext sca} $\Phi \circ \iota = \id_N = \varphi$.

    We claim that $\iota$ is an $R$-module isomorphism. Firstly $\iota$ is injective since $\Phi\iota = \id_N$. Next $\iota$ is surjective since $r\ten n = 1\ten \br{rn} = \iota\br{rn}$. Thus the claim is proved.
    \item Let $N$ be a finite abelian group, and so $N$ is a $\Z$-module. We claim that $\Q\ten_\Z N = 0$. To see this, let $\varphi : \Z\to \Q$ and denote $n:= |N|$. Note that
    \[\frac{r}{s}\ten x=\frac{r}{sn}\cdot n \ten x = \frac{r}{sn}\ten nx = \frac{r}{sn}\ten 0 = 0 \]
    This means that to extend the scalar of $N$ from $\Z$ to $\Q$, the only possible embedding is the zero map. In other words, any quotient of $N$ that can be embedded into a $\Q$-module is the zero quotient.
    \item We claim that $\Q\ten_\Z \Z\cong \Q$. Similarly to the method in first bullet we have the following commutative diagram:
    \[\begin{tikzcd}[sep=small]
	n & \Z && {\Q\ten_\Z \Z} & {1\ten n} \\
	&& \\
	&&& \Q \\
	&&& {\frac{n}{1}}
	\arrow["{\in }"{description}, draw=none, from=1-1, to=1-2]
	\arrow[curve={height=-24pt}, maps to, from=1-1, to=1-5]
	\arrow[from=1-1, to=4-4]
	\arrow["\iota", from=1-2, to=1-4]
	\arrow["\beta"', from=1-2, to=3-4]
	\arrow["\ni"{description}, draw=none, from=1-4, to=1-5]
	\arrow["\Phi", dashed, from=1-4, to=3-4]
	\arrow["{\in }"{marking, allow upside down}, draw=none, from=4-4, to=3-4]
\end{tikzcd}\]
    We claim that $\Phi$ is isomorphism. First to see surjectivity:
    \[\frac{n}{m} = \frac{1}{m}\cdot \frac{n}{1} = \frac{1}{m}\func{\beta}{n} = \frac{1}{m}\Phi(1\ten n) = \func{\Phi}{\frac{1}{m}\ten n}\]
    To see injectivity, note that an element in $\Q \ten_\Z \Z$ takes the form $\sum \br{q_i\ten n_i}$ where $q_i\in \Q$ and $n_i\in \Z$. We can rewrite such element as the following:
    \[\sum\br{q_i\ten n_i} = \sum\br{q_in_i\ten 1} = \br{\sum\br{q_in_i}}\ten 1 = q\ten 1\]
    where the last equality is to rename the chunky sum into some element $q\in \Q$. We now prove injectivity by showing that the $\ker \Phi$ is trivial: let $q=a/b$ where $q\ten 1\in \ker \Phi$, then
    \begin{align*}
        \func{\Phi}{q\ten 1} = 0 
        &\implies b\cdot\Phi \br{q\ten 1} = b\cdot0 \\
		&\implies \Phi(a\ten 1) = 0\\
		&\implies a\Phi(1\ten 1) = 0\\
        &\implies a\Phi\br{\iota(1)} = 0\\
        &\implies a \beta(1) = 0\\
        &\implies a \cdot \frac{1}{1} = 0 \\
        &\implies a = 0
    \end{align*}
    and thus $q\ten 1 = 0 \ten 1 = 0$. This shows that $\ker \Phi$ is trivial, so $\Phi$ is injective.
    \end{enumerate}
\end{ex}

The following definition, as suggested in its name, is the generalization of bi-linearity in linear algebra:
\begin{defn} [$R$-bilinear]
    Let $R$ be a commutative ring and let $L, M, N$ be $R$-modules. A map $\beta:M\times N\to L$ is said to be $R$-bilinear if all the following holds:
    \begin{enumerate}
        \item $\beta(rm+r'm', n) = r\beta(m,n) + r'\beta(m', n)$
        \item $\beta(m, rn+r'n') = r\beta(m,n) + r'\beta(m,n')$
    \end{enumerate}
\end{defn}
It is immediate from the definition that $R$-bilinear implies $R$-balanced. Conversely, if an $R$-balanced map is, in a sense, 'two-sided $R$-balanced', then the map is $R$-bilinear.

\medskip

\begin{cor}
    Let $R$ be a commutative ring, $M$ and $N$ be $R$-modules. Then $M\ten _R N$ is an $R$-module and $\iota:M\times N \to M\ten_R N$ where $\br{m,n}\mapsto m\ten n$ is bilinear. Furthermore, if $L$ is an $R$-module, we have a bijection between the sets:
    \[\sbr{R\text{-bilinear maps }\beta:M\times N \to L} \longleftrightarrow \sbr{R\text{-module homomorphisms } \Phi: M\ten_R N \to L}\]
    where the bijection is given by the relation $\Phi \circ \iota = \beta$.
\end{cor}
\begin{proof}
    It has been shown that $M\ten_R N$ is indeed an $R$-module by some previous statement. Also, we have proven that $\iota$ is a left $R$-balanced map. Since $R$ is commutative, the same can be concluded that $\iota$ is a right $R$-balanced map. It remains to show that $\Phi:M\ten_R N \to L$ defined by the relation $\Phi\circ \iota = \beta$ is indeed a $R$-module homomorphism.

    First, note that we have the following commutative diagram:
    \[\begin{tikzcd}[sep=small]
	{(m,n)} & {M\times N} && {M \ten_R N} & {m\ten n} \\
	\\
	&&& L
	\arrow["\in"{marking, allow upside down}, draw=none, from=1-1, to=1-2]
	\arrow[curve={height=-18pt}, maps to, from=1-1, to=1-5]
	\arrow["\iota", from=1-2, to=1-4]
	\arrow["\beta"', from=1-2, to=3-4]
	\arrow["\ni"{marking, allow upside down}, draw=none, from=1-4, to=1-5]
	\arrow["\Phi", dashed, from=1-4, to=3-4]
\end{tikzcd}\]
    By Theorem \ref{thm: uni prop tens}, the map $\Phi$ exists and it is a group homomorphism. We show that $\Phi$ respects the action of $R$:
    \begin{align*}
        \Phi(r(m\ten n)) 
        &= \Phi(rm \ten n)\\
        &= \Phi(\iota(rm,n))\\
        &= \beta(rm,n) \\
        &= r \beta(m,n)\\
        &= r \Phi(\iota(m,n) \\
        &= r \Phi(m\ten n)
    \end{align*}
    This completes the proof.
\end{proof}

\medskip

\begin{ex}
    Define commutative ring homomorphism $\varphi: R\to S$. We have seen that $S \ten_R R \cong S$ as a left $S$-module. In fact we have that $R\ten_R S\cong S$ as right $S$-module.
\end{ex}

\medskip

\begin{thm} [Tensor product of $R$-module homomorphisms]
    Let $M_R, M'_R, _RN, _RN'$ be $R$-modules. Let $\alpha: M\to M'$ and $\beta:N\to N'$ be $R$-module homomorphisms. Then we have the following:
    \begin{enumerate}
        \item There exists a unique group homomorphism $\alpha\ten \beta: M\ten_R N\to M' \ten_R N'$ where $\func{\br{\alpha\ten \beta}}{m\ten n} = \func{\alpha}{m}\ten \func{\beta}{n}$.
        \item If $M$ and $M'$ are $\br{S,R}$-bimodule, then $\alpha\ten \beta$ is a $S$-module homomorphism.
        \item Suppose further that we have $M''_R$ and $_RN''$ as $R$-modules. Let $\lambda:M'\to M''$ and $\mu:N'\to N''$ be $R$-module homomorphisms. Then we have $(\lambda\alpha)\ten(\mu\beta) = (\lambda\ten \mu) \circ (\alpha\ten \beta)$.
    \end{enumerate}
\end{thm}
\begin{proof}
    Let $\gamma:M\times N \to M'\ten_R N'$ such that $(m,n)\mapsto \alpha(m)\ten \beta(n)$. We first show that $\gamma$ is $R$-balanced. Note
    \begin{enumerate}
        \item $\gamma(mr,n)  = \alpha(mr)\ten\beta(n) = \alpha(m)r \ten \beta(n) = \alpha(m) \ten r\beta(n) = \alpha(m) \ten \beta(rn) = \gamma(m,rn)$
        \item $\gamma(m+m',n) = \alpha(m+m')\ten \beta(n) = \br{\alpha(m)+\alpha(m')} \ten \beta(n) = (\alpha(m) \ten\beta(n)) +(\alpha(m')\ten\beta(n)) = \gamma(m,n) + \gamma(m',n)$
        \item $\gamma(m,n+n') = \alpha(m)\ten \beta(n+n') = \alpha(m)\ten(\beta(n) + \beta(n'))= (\alpha(m) \ten\beta(n)) +(\alpha(m)\ten\beta(n')) = \gamma(m,n) + \gamma(m,n')$
    \end{enumerate}
    And thus we obtain the following commutative diagram:
    \[\begin{tikzcd}[sep=small]
	{(m,n)} & {M\times N} && {M\ten_R N} & {m\ten n} \\
	\\
	&&& {M' \ten_R N'} \\
	&&& {\alpha(m) \ten \beta(n)}
	\arrow["{\in }"{marking, allow upside down}, draw=none, from=1-1, to=1-2]
	\arrow[curve={height=-12pt}, maps to, from=1-1, to=1-5]
	\arrow[maps to, from=1-1, to=4-4]
	\arrow["\iota", from=1-2, to=1-4]
	\arrow["\gamma"', from=1-2, to=3-4]
	\arrow["{\exists! \Phi}", dashed, from=1-4, to=3-4]
	\arrow["\in"{marking, allow upside down}, draw=none, from=1-5, to=1-4]
	\arrow["\in"{marking, allow upside down}, draw=none, from=4-4, to=3-4]
    \end{tikzcd}\]
    where the existence of the group homomorphism $\Phi$ is ensured by Theorem \ref{thm: uni prop tens}. This proves the first statement.

    For the second statement, suppose that $M$ and $M'$ are $(S,R)$-bimodules. To show that $\Phi$ is an $S$-module homomorphism, we see that
    \begin{align*}
        \Phi(s(m\ten n)) 
        &= \Phi(sm\ten n )\\
        &= \Phi(\iota(sm,n)) \\
        &= \gamma(sm,n)\\
        &= \alpha(sm)\ten \beta(n) \\
        &= s\alpha(m) \ten \beta(n)\\
        &= s(\alpha(m) \ten \beta(n)) \\
        &= s\Phi(m\ten n)
    \end{align*}
    This proves the second statement.

    For the third statement, note that $\lambda\alpha:M\to M''$ is well-defined, where $m\mapsto \lambda(\alpha(m))$. Similarly we have that $\mu\beta:N\to N''$ is well-defined. Define $\gamma:M\times N\to M''\times N''$ such that $(m,n)\mapsto (\lambda\alpha)(m)\ten (\mu \beta)(n)$. We shall prove that $\gamma$ is $R$-balanced, but is omitted here for the sake of readability. Similarly we have the following commutative diagram:
    \[\begin{tikzcd}[sep=small]
	{(m,n)} & {M\times N} && {M\ten_R N} & {m\ten n} \\
	\\
	&&& {M'' \ten_R N''} \\
	&&& {(\lambda\alpha)(m) \ten (\mu\beta)(n)}
	\arrow["{\in }"{marking, allow upside down}, draw=none, from=1-1, to=1-2]
	\arrow[curve={height=-12pt}, maps to, from=1-1, to=1-5]
	\arrow[maps to, from=1-1, to=4-4]
	\arrow["\iota", from=1-2, to=1-4]
	\arrow["\gamma"', from=1-2, to=3-4]
	\arrow["{\exists! \Phi}", dashed, from=1-4, to=3-4]
	\arrow["\in"{marking, allow upside down}, draw=none, from=1-5, to=1-4]
	\arrow["\in"{marking, allow upside down}, draw=none, from=4-4, to=3-4]
\end{tikzcd}\]
    We will show that $(\lambda\alpha)\ten(\mu\beta)=(\lambda\ten\mu)\circ(\alpha\ten\beta)$ using the uniqueness of $\Phi$. First note that 
    \[\Phi(m\ten n) = \Phi(\iota(m,n)) = \gamma(m,n) = (\lambda\alpha)(m) \ten (\mu\beta)(n)\]
    Next, we see that
    \begin{align*}
       ((\lambda\ten\mu)\circ (\alpha\ten\beta))(\iota(m,n))
        &= ((\lambda\ten\mu)\circ (\alpha\ten\beta))(m\ten n)\\
        &= (\lambda\ten \mu)(\alpha(m)\ten \beta(n))\\
        &= \lambda(\alpha(m))\ten \mu(\beta(n)) \\
        &= (\lambda\alpha \ten \mu\beta)(m\ten n) \\
        &= (\lambda\alpha \ten \mu\beta)(\iota(m,n)) \\
        &= \Phi(\iota(m,n))
    \end{align*}
    Since $\Phi$ is unique, we see that $(\lambda\alpha)\ten(\mu\beta) = (\lambda\ten \mu) \circ (\alpha\ten \beta)$ must hold. This completes the proof.
\end{proof}

\begin{thm} [Associativity of tensor product] \label{thm: assoc ten}
    Consider the modules $M_R, _RN_S$ and $_SL$. Then there exists a unique module isomorphism such that 
    \[(M\ten_R N) \ten_S L \cong M\ten_R (N\ten_S L)\]
    where $\Phi\br{(m\ten_R n) \ten_S \ell} = m\ten_R(n\ten_S \ell)$. Furthermore, if $M$ is a $(T,R)$-bimodule, then $\Phi$ is a $T$-module isomorphism.
\end{thm}
\begin{proof}
    Fix $\ell\in L$. Define $\iota:M\times N\to M\ten_R N$ such that $(m,n)\mapsto m\ten n$. Also define $\beta:M\times N\to M\ten_R (N\ten_S L)$ where $(m,n)\mapsto m\ten (n\ten \ell)$. We prove that $\beta$ is $R$-balanced:
    \begin{enumerate}
        \item $\beta(mr, n) = mr\ten (n\ten \ell) = m\ten r(n\ten \ell) = m\ten (rn\ten \ell) = \beta(m,rn)$
        \item $\beta(m+m', n) = (m+m')\ten n = (m\ten n) + (m'\ten n) = \beta(m,n) + \beta(m',n)$
        \item $\beta(m,n+n') = m \ten (n+n') = (m\ten n) + (m\ten n') = \beta(m,n) + \beta(m,n')$
    \end{enumerate}
    And so we have the following commutative diagram:
    \[\begin{tikzcd}[sep=small]
	{(m,n)} & {M\times N} && {M\ten_R N} & {m\ten n} \\
	\\
	&&& {M\ten_R (N\ten_S L)} \\
	&&& {m\ten(n\ten \ell)}
	\arrow["{\in }"{marking, allow upside down}, draw=none, from=1-1, to=1-2]
	\arrow[curve={height=-20pt}, maps to, from=1-1, to=1-5]
	\arrow[maps to, from=1-1, to=4-4]
	\arrow["\iota", from=1-2, to=1-4]
	\arrow["\beta"', from=1-2, to=3-4]
	\arrow["{\exists! \Phi_\ell}", dashed, from=1-4, to=3-4]
	\arrow["\in"{marking, allow upside down}, draw=none, from=1-5, to=1-4]
	\arrow[curve={height=-20pt}, maps to, from=1-5, to=4-4]
	\arrow["\in"{marking, allow upside down}, draw=none, from=4-4, to=3-4]
\end{tikzcd}\]
    where the existence of the group homomorphism $\Phi_\ell$ is ensured by, again, Theorem \ref{thm: uni prop tens}. Note that since $\ell$ is fixed, so $\Phi_\ell$ is with respect to the choice of $\ell$. 

    Next, define $\iota':(M\ten_R N) \times L \to (M\ten_R N)\ten_S L$ where $(m\ten n, \ell) \mapsto (m\ten n)\ten \ell$. Also, define $\Phi:(M\ten_R N)\times L \to M\ten_R(N\ten_S L)$ where $(m\ten n, \ell)\mapsto \Phi_\ell(m\ten n)$. We claim that $\Phi$ is an $S$-homomorphism, since for $s\in S$ we have
    \[\func{\Phi}{(m\ten n)s, \ell} = \Phi(m\ten ns, \ell) = \Phi_\ell(m\ten ns) = m\ten (ns\ten \ell) = m\ten (n\ten s\ell) = \Phi_{s\ell}(m\ten n) = \Phi(m\ten n, s\ell)\]
    And thus we obtain the following commutative diagram:
    \[\begin{tikzcd}[sep=small]
	{(m\ten n, \ell)} & {(M\ten_R N)\times L} && {(M\ten_R N)\ten_S L} & {(m\ten n) \ten \ell} \\
	\\
	&&& {M\ten_R(N\ten_S L)} \\
	&&& {\Phi_\ell (m\ten n)}
	\arrow["\in"{description}, draw=none, from=1-1, to=1-2]
	\arrow[curve={height=-18pt}, maps to, from=1-1, to=1-5]
	\arrow[maps to, from=1-1, to=4-4]
	\arrow["{\iota'}", from=1-2, to=1-4]
	\arrow["\Phi"', from=1-2, to=3-4]
	\arrow["\ni"{description}, draw=none, from=1-4, to=1-5]
	\arrow["\exists !\Psi", dashed, from=1-4, to=3-4]
	\arrow[curve={height=-18pt}, maps to, from=1-5, to=4-4]
	\arrow["\in"{marking, allow upside down}, draw=none, from=4-4, to=3-4]
\end{tikzcd}\]
where the existence of the $S$-module homomorphism $\Psi$ is unique by Theorem \ref{thm: uni prop ext sca}. Since the diagram is commutative, we see that
\[\Psi((m\ten n), \ell) = \Phi_\ell(m\ten n) = m \ten (n\ten \ell)\]
The whole argument can be repeated again, first by fixing $m\in M$ to get $\tilde\Phi_m:M\times(N\ten_SL) \to M\ten_R (N\ten_S L)$ such that $(m,n\ten \ell) \mapsto m\ten(n\ten \ell)$, then to obtain an $S$-module homomorphism $\tilde\Psi:M\ten_R (N\ten_S L)\to (M\ten_R N) \ten_S L$ such that 
\[\tilde\Psi(m, n\ten \ell) = \tilde\Phi_m(n\ten \ell) = m\ten(n\ten \ell)\]
In other words, we now obtain two $S$-module homomorphisms such that
\[\begin{tikzcd}[sep=small]
	{\Psi:(M\ten_R N)\ten_S L} && {M\ten_R (N\ten_S L): \tilde\Psi}
	\arrow[shift left,  from=1-1, to=1-3]
	\arrow[shift left,  from=1-3, to=1-1]
\end{tikzcd}\]
To show that $(M\ten_R N)\ten_S L\cong M\ten_R (N\ten_SL)$, it is sufficient to prove that $\tilde\Psi\circ \Psi$ and $\Psi\circ \tilde\Psi$ are identity maps. We only show that first one. Note that
\[(\tilde\Psi\circ \Psi)(m\ten n, \ell) = \tilde\Psi(m\ten (n\ten \ell)) = (m\ten n)\ten\ell\]
Since $\tilde\Psi$ and $\Psi$ are both $S$-module homomorphism, it is immediate that $\tilde\Psi \circ \Psi$ is an $S$-module homomorphism as well. We thus obtain the following commutative diagram by the universal property of tensor product:
\[\begin{tikzcd}[sep=small]
	{(m\ten n, \ell)} & {(M\ten_R N)\times L} && {(M\ten_R N)\ten_S L} & {(m\ten n) \ten \ell} \\
	\\
	&&& {M\ten_R(N\ten_S L)} \\
	&&& {(m\ten n)\ten \ell}
	\arrow["\in"{description}, draw=none, from=1-1, to=1-2]
	\arrow[curve={height=-18pt}, maps to, from=1-1, to=1-5]
	\arrow[maps to, from=1-1, to=4-4]
	\arrow[from=1-2, to=1-4]
	\arrow["{\tilde\Psi \circ \Psi}"', from=1-2, to=3-4]
	\arrow["\ni"{description}, draw=none, from=1-4, to=1-5]
	\arrow["\id", dashed, from=1-4, to=3-4]
	\arrow[curve={height=-18pt}, maps to, from=1-5, to=4-4]
	\arrow["\in"{marking, allow upside down}, draw=none, from=4-4, to=3-4]
\end{tikzcd}\]
where it is clear that the map $\id$ is the only map that takes $(m\ten n)\ten \ell$ to itself. By the uniqueness statement in the universal property, we see that $\tilde\Psi\circ\Psi =\id$. Similar argument can be used to show that $\Psi\circ\tilde\Psi = \id$. This completes the proof.
\end{proof}
The following is an immediate corollary of the previous theorem:
\medskip

\begin{cor}
    Let $R$ be a commutative ring. Let $M,N,L$ be $R$-modules. Then $(M\ten_R N) \ten_R L \cong M\ten_R (N\ten_R L)$
\end{cor}

\medskip

\begin{thm} [Distributivity of tensor product]
    Let $M_R, M'_R, _RN, _RN'$ be $R$-modules. Then there exists a unique module isomorphism $\Phi$ such that 
    \[M\ten_R (N\oplus N') \cong (M\ten_R N) \oplus(M\ten_R N')\]
    where $\Phi(m\ten (n,n')) = (m\ten n, m \ten n')$. Similarly, there exists a unique module isomorphism $\Phi'$ such that 
    \[(M\oplus M') \ten_R N \cong (M\ten_R N) \oplus(M'\ten_R N)\]
    where $\Phi'((m,m')\ten n) = (m\ten n, m' \ten n)$. 
\end{thm}
\begin{proof}
    We only prove the second statement, where the arguments are very similar to Theorem \ref{thm: assoc ten}, thus some details are omitted. First, obtain the following three diagrams:

    Diagram 1:
\[\begin{tikzcd}[sep=small]
	{((m,m'),n)} & {(M\oplus M') \times N} && {(M\oplus M') \ten_R N} & {(m,m')\ten n} \\
	\\
	&&& {(M\ten _R N) \oplus (M'\ten_R N)} \\
	&&& {(m\ten n, m'\ten n)}
	\arrow["\in"{marking, allow upside down}, draw=none, from=1-1, to=1-2]
	\arrow[maps to, from=1-1, to=4-4]
	\arrow[from=1-2, to=1-4]
	\arrow["\beta"', from=1-2, to=3-4]
	\arrow["\ni"{marking, allow upside down}, draw=none, from=1-4, to=1-5]
	\arrow["\in"{marking, allow upside down}, draw=none, from=4-4, to=3-4]
\end{tikzcd}\]

Diagram 2:
\[\begin{tikzcd}[sep=small]
	{(m,n)} & {M\times N} && {M\ten_R N} \\
	\\
	&&& {(M\oplus M') \ten_R N} \\
	&&& {(m,0)\ten n}
	\arrow["\in"{marking, allow upside down}, draw=none, from=1-1, to=1-2]
	\arrow[maps to, from=1-1, to=4-4]
	\arrow[from=1-2, to=1-4]
	\arrow["\gamma"', from=1-2, to=3-4]
	\arrow["\in"{marking, allow upside down}, draw=none, from=4-4, to=3-4]
\end{tikzcd}\]

Diagram 3:
\[\begin{tikzcd}[sep=small]
	{(m',n)} & {M'\times N} && {M'\ten_R N} \\
	\\
	&&& {(M\oplus M') \ten_R N} \\
	&&& {(0,m')\ten n}
	\arrow["\in"{marking, allow upside down}, draw=none, from=1-1, to=1-2]
	\arrow[maps to, from=1-1, to=4-4]
	\arrow[from=1-2, to=1-4]
	\arrow["\delta"', from=1-2, to=3-4]
	\arrow["\in"{marking, allow upside down}, draw=none, from=4-4, to=3-4]
\end{tikzcd}\]
Next, prove that all maps $\beta, \gamma$, and $\delta$ are $R$-balanced, which is omitted here due to tedious work. By Theorem \ref{thm: uni prop tens}, there exists unique group homomorphism $\Phi, \varphi$, and $\varphi'$ respectively for diagram 1, 2, and 3 such that all are them are commutative:

Diagram 1:
\[\begin{tikzcd}[sep=small]
	{((m,m'),n)} & {(M\oplus M') \times N} && {(M\oplus M') \ten_R N} & {(m,m')\ten n} \\
	\\
	&&& {(M\ten _R N) \oplus (M'\ten_R N)} \\
	&&& {(m\ten n, m'\ten n)}
	\arrow["\in"{marking, allow upside down}, draw=none, from=1-1, to=1-2]
	\arrow[curve={height=-18pt}, maps to, from=1-1, to=1-5]
	\arrow[maps to, from=1-1, to=4-4]
	\arrow[from=1-2, to=1-4]
	\arrow["\beta"', from=1-2, to=3-4]
	\arrow["\ni"{marking, allow upside down}, draw=none, from=1-4, to=1-5]
	\arrow["{\exists ! \Phi}", dashed, from=1-4, to=3-4]
	\arrow[curve={height=-24pt}, maps to, from=1-5, to=4-4]
	\arrow["\in"{marking, allow upside down}, draw=none, from=4-4, to=3-4]
\end{tikzcd}\]

Diagram 2:
\[\begin{tikzcd}[sep=small]
	{(m,n)} & {M\times N} && {M\ten_R N} & {m\ten n} \\
	\\
	&&& {(M\oplus M') \ten_R N} \\
	&&& {(m,0)\ten n}
	\arrow["\in"{marking, allow upside down}, draw=none, from=1-1, to=1-2]
	\arrow[curve={height=-24pt}, maps to, from=1-1, to=1-5]
	\arrow[maps to, from=1-1, to=4-4]
	\arrow[from=1-2, to=1-4]
	\arrow["\gamma"', from=1-2, to=3-4]
	\arrow["{\exists ! \varphi}", dashed, from=1-4, to=3-4]
	\arrow["\ni"{description}, draw=none, from=1-5, to=1-4]
	\arrow[curve={height=-18pt}, maps to, from=1-5, to=4-4]
	\arrow["\in"{marking, allow upside down}, draw=none, from=4-4, to=3-4]
\end{tikzcd}\]

Diagram 3:
\[\begin{tikzcd}[sep=small]
	{(m',n)} & {M'\times N} && {M'\ten_R N} & {m' \ten n} \\
	\\
	&&& {(M\oplus M') \ten_R N} \\
	&&& {(0,m')\ten n}
	\arrow["\in"{marking, allow upside down}, draw=none, from=1-1, to=1-2]
	\arrow[curve={height=-18pt}, maps to, from=1-1, to=1-5]
	\arrow[maps to, from=1-1, to=4-4]
	\arrow[from=1-2, to=1-4]
	\arrow["\beta"', from=1-2, to=3-4]
	\arrow["{\exists ! \varphi'}", dashed, from=1-4, to=3-4]
	\arrow["\ni"{description}, draw=none, from=1-5, to=1-4]
	\arrow[curve={height=-18pt}, maps to, from=1-5, to=4-4]
	\arrow["\in"{marking, allow upside down}, draw=none, from=4-4, to=3-4]
\end{tikzcd}\]

We define the map $\Psi:(M\ten _R N)\oplus (M'\ten_R N) \to (M\oplus M')\ten_R N$ such that $((m\ten n), (m'\ten n'))\mapsto \func{\varphi}{m\ten n} + \func{\varphi'}{m'\ten n'}=(m,0)\ten n + (0,m')\ten n'$. Note both $\varphi$ and $\varphi'$ are group homomorphisms, and thus $\Psi$ is a group homomorphism. 

Finally, to show that $(M\ten _R N)\oplus (M'\ten_R N) \cong (M\oplus M')\ten_R N$, we need to show that $\Phi\circ  \Psi$ and $\Psi\circ \Phi$ are identity maps. We only show one, since another follows with the a similar argument:
\[\Psi(\func{\Phi}{(m,m')\ten n}) = \Psi(m\ten n, m'\ten n) = (m,0)\ten n + (0,m')\ten n = (m,m')\ten n\]
This (more or less) completes the proof.
\end{proof}

\begin{cor}
    Let $f:R\to S$ be a ring homomorphism. Then $S\ten_R R^m \cong S^m$ as a left $S$-module.
\end{cor}
\begin{proof}
    Note
    \[R^m = \bigoplus_{i=1}^mR\]
    and so 
    \[S\ten_R R^m =S\ten_R \bigoplus_{i=1}^m R\cong \bigoplus_{i=1}^m\br{S\ten R} \cong \bigoplus_{i=1}^m S= S^m\]
    This completes the proof.
\end{proof}

\begin{cor}
    Let $R$ be commutative. Then $R^m \ten_R R^n \cong R^{mn}$
\end{cor}
\begin{proof}
    Similar to the previous proof, we note
    \[R^m \ten_R R^n = \br{\bigoplus_m R} \ten \br{\bigoplus_n R} = \bigoplus_m \br{R\ten_R \bigoplus_n R} = \bigoplus_m \bigoplus_n (R\ten_R R) = \bigoplus_m \bigoplus_n R = R^{mn}\]
    This completes the proof.
\end{proof}
\newpage

\section{Injective, Projective, and Flat Modules}
\subsection{Short Exact Sequence and Splitting}

\begin{defn} [Exact sequence and complex]
    \hfill
    \begin{enumerate}
        \item A pair of $R$-module homomorphism
        \[X\xrightarrow{\alpha}Y\xrightarrow{\beta} Z\]
        is said to be exact at $Y$ if $\ker \beta = \im \alpha$, and we say that this sequence is exact.
        \item A complex is a chain 
        \[\dots \xrightarrow{d_{-2}}X_{-1}\xrightarrow{d_{=1}}X_0 \xrightarrow{d_0}X_1\xrightarrow{d_1}\dots\]
        where $X_I$ are $R$-modules, and $d_i$ are $R$-modules homomorphisms such that $d_{i+1}d_i=0$ for all $i$. In other words, we have that $\im d_i \subseteq \ker d_{i+1}$
        \item A complex is said to be exact if it is exact at every $X_i$.
    \end{enumerate}
\end{defn}

\medskip

\begin{re}
    From the above definition we see that an exact sequence can be made into an exact complex by adding zeroes and zero maps.
\end{re}

\medskip

\begin{pro}
    \hfill
    \begin{enumerate}
        \item The sequence of $R$-modules $0\to X\xrightarrow{\alpha} Y$ is exact if and only if $\alpha$ is injective.
        \item The sequence $Y\xrightarrow{\beta}Z\to 0$ is exact if and only if $\beta$ is surjective.
    \end{enumerate}
\end{pro}
\begin{proof}
    For the first statement, first suppose that the sequence is exact. Then $\ker\alpha = \im 0 = \sbr{0}$, and thus $\alpha$ is injective. For the converse statement, suppose that $\alpha$ is injective, then $\ker\alpha = 0$. On the other hand, the map from $0$ to $X$ is a zero map. Together we have $\ker\alpha = 0=\im 0$, so the sequence is exact.

    For the second statement, first suppose that the sequence is exact. Then $\im \beta = \ker 0$. However, the zero map $0:Z\to 0$ sends everything to $0$, so the kernel is $Z$. Together we have $\im\beta = Z$, thus $\beta$ is surjective. For the converse statement, suppose that $\beta$ is surjective. Then $\im\beta = Z$. On the other hand, the map from $Z$ to $0$ is a zero map, so $\ker 0 = Z$. Together we have $\im \beta =Z=\ker 0$, so the sequence is exact.
\end{proof}

\begin{cor}
    The sequence $0\to X\xrightarrow{\alpha}Y\xrightarrow{\beta}Z\to 0$ is exact if and only if $\alpha$ is injective and $\beta$ is injective and $\im \alpha = \ker \beta$. 
\end{cor}

\medskip

\begin{re}
    In this case, we called such sequence a short exact sequence (SES). Moreover, note that
    \[Y/\im \alpha \cong Z\]
    in the SES.
\end{re}

\medskip

\begin{ex}
    \hfill
    \begin{enumerate}
        \item Let $\varphi:M\to N$ be a $R$-module homomorphism. Then we have the SES:
        \[0\to \ker \varphi\to M\to \im\varphi \to 0\]
        \item Let $M$ be a $R$-module and $S$ be a generating set of $M$. Then there exists a surjective $R$-homomorphism $\pi$ such that $F(S)\xrightarrow{\pi}M$. This is due to the universal property:
        \[\begin{tikzcd}[sep=small]
	s & S && {F(S)} & \bullet \\
	\\
	&&& M \\
	&&& s
	\arrow["\in"{description}, draw=none, from=1-1, to=1-2]
	\arrow[curve={height=-18pt}, from=1-1, to=1-5]
	\arrow[maps to, from=1-1, to=4-4]
	\arrow[from=1-2, to=1-4]
	\arrow[from=1-2, to=3-4]
	\arrow[dashed, two heads, from=1-4, to=3-4]
	\arrow[curve={height=-24pt}, maps to, from=1-5, to=4-4]
	\arrow["\in"{marking, allow upside down}, draw=none, from=4-4, to=3-4]
\end{tikzcd}\]
        If $M$ is finitely generated, then we can choose $S$ such that $n:=|S|<\infty$, and we have
        \[\bigoplus_n {_RR}\cong F(S) \xrightarrow{\pi} M\]
        An $R$-module $M\neq0$ is simple/irreducible if $M$ has only $0$ and $M$ as submodule. Let $M$ be simple and $0\neq m\in M$. Then
        \[0\neq Rm = \sbr{rm:r\in R} = M\]
        due to simplicity of $M$. This says that $\sbr{m}$ generates $M$, and so by above we have that $R$ surjects to $M$. This tells that simple $R$-module is quotient of the regular module. In general, every $R$-module is a quotient of a free module. We then obtain the SES:
        \[0\to \ker\pi \to F(S) \to M\to 0\]
    \end{enumerate}
\end{ex}

\medskip

\begin{defn} [Complex homomorphisms]
    Let $0\to X\to Y\to Z\to 0$ and $0\to X'\to Y'\to Z'\to 0$ be SES of $R$-modules. 
    \begin{enumerate}
        \item A homomorphism between the SES's is a collection of $R$-module homomorphisms $\gamma_1, \gamma_2, \gamma_3$ such that the following is commutative:
        \[\begin{tikzcd}[sep=small]
	0 && X && Y && Z && 0 \\
	\\
	0 && {X'} && {Y'} && {Z'} && 0
	\arrow[from=1-1, to=1-3]
	\arrow[from=1-3, to=1-5]
	\arrow["{\gamma_1}"', from=1-3, to=3-3]
	\arrow[from=1-5, to=1-7]
	\arrow["{\gamma_2}"', from=1-5, to=3-5]
	\arrow[from=1-7, to=1-9]
	\arrow["{\gamma_3}"', from=1-7, to=3-7]
	\arrow[from=3-1, to=3-3]
	\arrow[from=3-3, to=3-5]
	\arrow[from=3-5, to=3-7]
	\arrow[from=3-7, to=3-9]
\end{tikzcd}\]
    and we say that the complex homomorphism is an isomorphism if the collection $\gamma_1, \gamma_2, \gamma_3$ are isomorphisms.
    \item The SES $0\to X\to Y\to Z\to 0$ and $0\to X\to Y'\to Z\to 0$ are said to be equivalent if there exists an $R$-module isomorphism $g:Y\to Y'$ such that the following commutes:
    \[\begin{tikzcd}[sep=small]
	0 && X && Y && Z && 0 \\
	\\
	0 && X && {Y'} && Z && 0
	\arrow[from=1-1, to=1-3]
	\arrow[from=1-3, to=1-5]
	\arrow["\id"', from=1-3, to=3-3]
	\arrow[from=1-5, to=1-7]
	\arrow["g"', from=1-5, to=3-5]
	\arrow[from=1-7, to=1-9]
	\arrow["\id"', from=1-7, to=3-7]
	\arrow[from=3-1, to=3-3]
	\arrow[from=3-3, to=3-5]
	\arrow[from=3-5, to=3-7]
	\arrow[from=3-7, to=3-9]
\end{tikzcd}\]
    \end{enumerate}
\end{defn}

\begin{ex}
    Since there is no isomorphism between $\Z$ and $\Z\oplus \Z/n\Z$, thus the following SES must not be equivalent:
    \[0\to \Z \xrightarrow{\times n}\Z\xrightarrow{\bmod n}\Z/n\Z \to 0 \]
    and 
    \[0\to \Z\to \Z\oplus \Z/n\Z\to \Z/n\Z \to 0\]
\end{ex}

\begin{pro} [Five Lemma]
Suppose we have the following commutative diagram and suppose that each of the following rows are exact:
    \[\begin{tikzcd}[sep=small]
	{M_1} && {M_2} && {M_3} && {M_4} && {M_5} \\
	\\
	{N_1} && {N_2} && {N_3} && {N_4} && {N_5}
	\arrow["{g_1}", from=1-1, to=1-3]
	\arrow["{f_1}", from=1-1, to=3-1]
	\arrow["{g_2}", from=1-3, to=1-5]
	\arrow["{f_2}", from=1-3, to=3-3]
	\arrow["{g_3}", from=1-5, to=1-7]
	\arrow["{f_3}", from=1-5, to=3-5]
	\arrow["{g_4}", from=1-7, to=1-9]
	\arrow["{f_4}", from=1-7, to=3-7]
	\arrow["{f_5}", from=1-9, to=3-9]
	\arrow["{h_1}", from=3-1, to=3-3]
	\arrow["{h_2}", from=3-3, to=3-5]
	\arrow["{h_3}", from=3-5, to=3-7]
	\arrow["{h_4}", from=3-7, to=3-9]
\end{tikzcd}\]
and $f_i$ are $R$-module homomorphisms. Then we have that:
\begin{enumerate}
    \item If $f_5$ is injective and $f_2, f_4$ are surjective, then $f_3$ is surjective.
    \item If $f_1$ is surjective and $f_2, f_4$ are injective, then $f_3$ is injective.
\end{enumerate}
\end{pro}
\begin{proof}
    We first show surjectivity. Suppose as assumed in the first statement. Let $n\in N_3$. Then $h_4(h_3(n))=0\in N_5$. Since $f_4$ is surjective, there exists $m\in M_4$ such that $f_4(m)=h_3(n)\in N_4$. Sending $m$ along two routes we have
    \[f_5(g_4(m)) = h_4(f_4(m)) = h_4(h_3(n)) = 0\in N_5 \]
    Since $f_5$ is injective, we have $g_4(m)=0\in M_5$, implying that $m\in \ker g_4$. Due to the exactness we have that $m\in \im g_3$, so there exists $a\in M_3$ such that $g_3(a)=m\in M_4$. Again, sending $a\in M_3$ along two routes we have
    \[h_3(f_3(a)) = f_4(g_3(a)) = f_4(m)=h_3(m)\]
    And thus $h_3(f_3(a)-m)=0$, implying that $f_3(a)-m\in \ker h_3 = \im h_2$, and so there is $b\in N_2$ such that $h_2(b)=f_3(a)-m\in N_3$. Note $f_2$ is surjective, so there exists $c\in M_2$ such that $f_2(c)=b\in N_2$. Sending $c\in M_2$ along two different routes we have
    \[f_3(h_2(c)) = h_2(f_2(c)) = h_2(b) = f_3(a)-m \in N_3\]
    Rearranging the equation we get $f_3(a-h_2(c))=m$. This proves that $f_3$ is surjective.

    We now show injectivity. Suppose as assumed in the second statement. It suffices to show that $\ker f_3$ is trivial. Let $m\in M_3$ such that $f_3(m)=0$. Sending $m\in M_3$ along two routes we get 
    \[f_4(g_3(m)) = h_3(f_3(m)) = h_3(0) = 0\in N_4\]
    Since $f_4$ is injective, so $g_3(m)=0\in M_4$. This implies that $m\in \ker g_3 = \im g_2$, so there exists $a\in M_2$ such that $g_2(a)=m\in M_3$. Sending $a\in M_2$ along two routes, we have
    \[h_2(f_2(a)) = f_3(g_2(a)) = f_3(m) = 0\]
    So $f_2(a)\in \ker h_2 = \im h_1$, implying that there exists $n\in N_1$ such that $h_1(n) = f_2(a)\in N_2$. Since $f_1$ is surjective, so there exists $b\in M_1$ such that $f_1(b) = n$. Sending $b\in M_1$ along two routes we have
    \[f_2(g_1(b)) = h_1(f_1(b)) = h_1(n) = f_2(a)\]
    Rearranging the equation we get $f_2(g_1(b)-a) = 0\in N_2$. Since $f_2$ is injective, so $g_1(b)-a = 0\in N_2$, which by rearranging we have $g_1(b)=a$. Lastly, send $b\in M_1$ to $M_3$ via compositing $g_1$ and $g_2$, which we get a zero map:
    \[g_2(g_1(b)) = 0\in M_3\]
    Since $g_1(b)=a$, we have that $g_2(a) = 0\in M_3$. Recall that $g_2(a)=m\in M_3$, so together we have that $m=0\in M_3$. This shows that $\ker f_3$ is trivial.
\end{proof}

\begin{defn} [Splitting sequence]
    A SES $0\to X\xrightarrow{\alpha}Y\xrightarrow{\beta}Z\to 0$ splits if there exists a submodule $Y'\subseteq Y$ such that $Y= Y'\oplus \alpha(X)$.
\end{defn}

\medskip

\begin{re}
    Note that if a SES splits, then we have that
    \[Y'\cong Y/\alpha(X)= Y/\im \alpha = Y/\ker\beta \cong Z\]
    Moreover, since $\alpha$ is injective, so $\alpha(X)\cong X$ and  we have $Y\cong X\oplus Z$. To conclude, a SES splits implies that $Y\cong X\oplus Z$. However, the converse is not true.
\end{re}

\medskip

\begin{pro} \label{pro: SES splits}
    Let $0\to X\xrightarrow{\alpha}Y\xrightarrow{\beta}Z\to 0$ be a SES. TFAE:
    \begin{enumerate}
        \item The SES splits.
        \item $\exists \gamma:Z\to Y$ is an $R$-module homomorphism such that $\beta\circ \gamma = \id_Z$
        \item $\exists \delta: Y\to X$ is an $R$-module homomorphism such that $\delta\circ \alpha = \id_X$
        \item $\exists \varphi:Y\to X\oplus Z$ such that $\varphi\circ \alpha = \iota:X\to X\oplus Z$ is the inclusion map and $\beta\circ \varphi^{-1}=\pi:X\oplus Z\to Z$ is the canonical map.
    \end{enumerate}
\end{pro}
\begin{proof}
    The logic chain of the proof is $1.\implies 2.\implies 3.\implies 4.\implies 1.$.

    $[1.\implies 2.]$. Suppose that the SES splits, so let $Y'\subseteq Y$ be such that $Y=Y'\oplus \alpha(X)$. We define the map $\gamma:Z\to Y$ where $z\mapsto \gamma(z)$ such that $\gamma(z)$ is defined via the following procedure
    \begin{itemize}
        \item Since $\im \beta = \ker 0 = Z$, so there exists $y\in Y$ such that $\beta(y)=z$.
        \item Since $Y=Y'\oplus \alpha(X)$, we can write $y=a+b$ by some $a\in Y'$ and $b\in \alpha(X)$.
        \item Then we define $\gamma(z):=a$.
    \end{itemize}
    We first claim that $\gamma$ is well-defined. Suppose that $\beta(y') = z = \beta(y)$ for some other $y'\in Y$ where $y'= a'+b'$ where $a'\in Y'$ and $b'\in \alpha(X)$. Note $\beta(y-y')=0$, so $y-y'\in \ker \beta = \im \alpha = \alpha(X)$. We can write
    \[y-y' = (a-a')+(b-b')\]
    Since $y-y'\in \alpha(X)$, so $a-a'$ must be the zero element, which shows that $a=a'$. This implies that $\gamma(z) = a = a'$, so $\gamma$ is well-defined. Next, we show that $\gamma$ is an $R$-module homomorphism, i.e. we show that $\gamma(rz) = r\gamma(z)$. Let $\beta(y)=z$. Note that 
    \[\beta(ry)=r\beta(y)=rz\]
    On the other hand, we have $ry=r(a+b) = ra+rb$. Since $Y'$ and $\alpha(X)$ are $R$-modules, so $ra\in Y'$ and $rb\in \alpha(X)$, implying that $ry\in Y'\oplus \alpha(X)$. Therefore
    \[\gamma(rz) = ra = r\gamma(z)\]
    This shows that $\gamma$ is an $R$-module homomorphism. Lastly, we check the requirement: for any $z\in Z$, let $\gamma(z)=a$, then 
    \[\beta(\gamma(z)) = \beta(a) \stackrel{(*)}= \beta(y)=z\]
    where the stared equality is achieved as followed: since $y\in Y=Y'\oplus\alpha(X)$, so we can write $y=a+b$ such that $a\in Y'$ and $b\in \alpha(X)$. Let $b=\alpha(x)$. Together, we see
    \[\beta(y) = \beta(a+b) = \beta(a) + \beta(b) = \beta(a) + \beta(\alpha(x)) = \beta(a)\]
    since $\beta\circ\alpha$ is zero map due to exactness. This shows that $\beta\circ \gamma = \id_Z$.
        
    $[2.\implies 3.]$. Suppose that we have an $R$-module homomorphism $\gamma:Z\to Y$ such that $\beta\circ\gamma = \id_Z$. We define $\delta:Y\to X$ such that $y\mapsto \delta(y)$ where $\delta(y)$ is defined as follow:
    \begin{itemize}
        \item Note that $\beta(y-\gamma(\beta(y))) = \beta(y) -\beta(\gamma(\beta(y))) = \beta(y) - \beta(y)=0$
        \item It implies that $y-\gamma(\beta(y))\in \ker \beta = \im \alpha$, so there exists $x\in X$ such that $\alpha(x) = y-\gamma(\beta(y))$.
        \item We then define $\delta(y):=x$.
    \end{itemize}
    We first show that $\delta$ is well-defined. Note that the map from $0$ to $X$ is a zero map, so $\ker \alpha =\im 0$ is trivial, meaning that $\alpha$ is injective. Since $\delta$ is defined via $\alpha$, so consequently $\delta$ must be injective. Next we show that $\delta$ is an $R$-module homomorphism. To compute $\delta(ry)$, consider:
    \[ry-\func{\gamma}{\func{\beta}{ry}} = r(y-\func{\gamma}{\beta(y)}) = r\alpha(x) = \alpha(rx)\]
    So $\delta(ry)=rx = r\delta(y)$. This shows that $\delta$ is an $R$-module homomorphism. Lastly, we check the requirement: to compute $\delta(\alpha(x))$, consider:
    \[\alpha(x) - \gamma(\beta(\alpha(x))) = \alpha(x) - \gamma(0) = \alpha(x)\]
    due to the exactness. So $\delta(\alpha(x)) = x$.  This shows that $\delta\circ \alpha = \id_X$.
    
    $[3.\implies 4.]$. Suppose we have an $R$-module homomorphism $\delta:Y\to X$ such that $\delta\circ \alpha = \id_X$. Define $\varphi:Y\to X\oplus Z$ such that $y\mapsto (\delta(y),\ \beta(y))$. We have then have the following diagram: 
    \[\begin{tikzcd}[sep=small]
	0 && X && Y && Z && 0 \\
	\\
	0 && X && {X\oplus Z} && Z && 0
	\arrow["0", from=1-1, to=1-3]
	\arrow["\id", tail reversed, from=1-1, to=3-1]
	\arrow["\alpha"', from=1-3, to=1-5]
	\arrow["\id", tail reversed, from=1-3, to=3-3]
	\arrow["\delta"'{pos=0.4}, shift right, curve={height=12pt}, from=1-5, to=1-3]
	\arrow["\beta"', from=1-5, to=1-7]
	\arrow["\varphi", from=1-5, to=3-5]
	\arrow["0", from=1-7, to=1-9]
	\arrow["\id", tail reversed, from=1-7, to=3-7]
	\arrow["\id", tail reversed, from=1-9, to=3-9]
	\arrow["0", from=3-1, to=3-3]
	\arrow["\iota", from=3-3, to=3-5]
	\arrow["\pi", from=3-5, to=3-7]
	\arrow["0", from=3-7, to=3-9]
\end{tikzcd}\]
    where $\iota:X\to X\oplus Z$ is the inclusion map $x\mapsto (x,0)$ and $\pi$ is the canonical map onto $Z$. We examine the two requirements. For the first one:
    \[\varphi(\alpha(x)) =(\delta(\alpha(x)), \beta(\alpha(x))) = (x, 0) = \iota(x)\]
    For the second one:
    \[\pi(\varphi(y)) = \pi(\delta(y), \beta(y)) = \beta(y)\]
    This shows $\pi\circ \varphi = \beta$, and thus $\beta\circ \varphi^{-1} = \pi$.
    
    $[4.\implies 1.]$. Suppose as assumed in the given condition. We have the following diagram:
    \[\begin{tikzcd}[sep=small]
	&&&& y && {\beta(y)} \\
	0 && X && Y && Z && 0 \\
	\\
	0 && X && {X\oplus Z} && Z && 0 \\
	&&&& {(0, \beta(y))} && {\beta(y)}
	\arrow[maps to, from=1-5, to=1-7]
	\arrow["\in"{marking, allow upside down}, draw=none, from=1-5, to=2-5]
	\arrow["\in"{marking, allow upside down}, draw=none, from=1-7, to=2-7]
	\arrow[curve={height=-24pt}, maps to, from=1-7, to=5-7]
	\arrow["0", from=2-1, to=2-3]
	\arrow["\id", tail reversed, from=2-1, to=4-1]
	\arrow["\alpha"', from=2-3, to=2-5]
	\arrow["\id", tail reversed, from=2-3, to=4-3]
	\arrow["\beta"', from=2-5, to=2-7]
	\arrow["\varphi", from=2-5, to=4-5]
	\arrow["0", from=2-7, to=2-9]
	\arrow["\id", tail reversed, from=2-7, to=4-7]
	\arrow["\id", tail reversed, from=2-9, to=4-9]
	\arrow["0", from=4-1, to=4-3]
	\arrow["\iota", from=4-3, to=4-5]
	\arrow["\pi", from=4-5, to=4-7]
	\arrow["0", from=4-7, to=4-9]
	\arrow["\in"{marking, allow upside down}, draw=none, from=5-5, to=4-5]
	\arrow[maps to, from=5-5, to=5-7]
	\arrow["\in"{marking, allow upside down}, draw=none, from=5-7, to=4-7]
\end{tikzcd}\]
    To show that the SES splits, define $Y' = \varphi^{-1}(0\oplus Z)$. Since $0$ and $Z$ are modules, and $\varphi$ is module homomorphism, so $Y'$ is a module, and is thus a submodule of $Y$. We claim that $Y=Y' \oplus \im \alpha$. By assumption $\varphi\circ \alpha = \iota$, so $\alpha = \varphi^{-1} \circ \iota$. Next, note that $\varphi^{-1}:X\oplus Z \to Y$, so
    \[Y' =\varphi^{-1}(X\oplus 0) \oplus \varphi^{-1}(0\oplus Z)  = \varphi^{-1}(\iota(X)) \oplus Y'= \alpha(X) \oplus Y'\]
    This completes the proof.
\end{proof}


\begin{pro} \label{pro: inj implies inj}
    Let $X, Y$ and $V$ be $R$-modules. Let $\beta:X\to Y$ be an $R$-module homomorphism. The map $\beta_*: \h_R(V,X)\to \h_R(V,Y)$ where $f\mapsto \beta\circ f$ is as abelian group homomorphism. Furthermore, if $\beta$ is injective, then so is $\beta_*$. In other words, the SES
    \[0\to X \xrightarrow{\beta} Y\]
    implies that we have the SES
    \[0 \to \h_R(V,X)\xrightarrow{\beta_*} \h_R(V,Y)\]
\end{pro}

\begin{proof}
    Suppose $\beta$ is injective. Let $f\in \h_R(V,X)$ such that $\beta\circ f = 0$ for every $v\in V$. Then for any $v\in V$ we have
    \[(\beta\circ f)(v)=0 \implies \beta(f(v)) = 0 \implies f(v)=0\]
    since $\beta$ is injective. This shows that $f$ is a zero map, so $\beta_*$ is injective.
\end{proof}
\begin{re}
    In general $\beta_*$ is not surjective even if $\beta$ is surjective.
\end{re}

\medskip

\begin{thm} \label{thm: all exact}
    Let $V, X, Y, Z$ be $R$-modules and
    \begin{equation}\label{eqn: short seq}
        0\to X \xrightarrow{\alpha} Y\xrightarrow{\beta} Z
    \end{equation}
    be a short sequence where $\alpha$ and $\beta$ are $R$-module homomorphisms. 
    \begin{enumerate}
        \item If the above short sequence \ref{eqn: short seq} is exact, then the following is exact:
    \[0 \to \h_R(V,X)\xrightarrow{\alpha_*}\h_R(V,Y)\xrightarrow{\beta_*}\h_R(V,Z)\]
        \item $0 \to \h_R(V,X)\xrightarrow{\alpha_*}\h_R(V,Y)\xrightarrow{\beta_*}\h_R(V,Z)$ is exact for all $V$ if and only if $0\to X \xrightarrow{\alpha} Y\xrightarrow{\beta}Z$ is exact.
    \end{enumerate}
\end{thm}
\begin{proof}
    For the first statement, suppose that the short sequence \ref{eqn: short seq} given exists. Then $\alpha$ is injective, and thus $\alpha_*$ is injective by Proposition \ref{pro: inj implies inj}. We now prove that $\im \alpha_* = \ker \beta_*$. Firstly, let $f\in \h_R(V,X)$. Since 
    \[(\beta_*\circ \alpha_*)(f) = \beta \circ \alpha \circ f = 0 \circ f = 0\]
    This implies that $\im\alpha_*\subseteq \ker\beta_*$. Next, to show $\ker\beta_*\subseteq \im \alpha_*$, let $g\in \ker \beta_*$, so for any $v\in V$ we have
    \[(\beta_*(g))(v) = \beta(g(v))=0\]
    Note it implies that $g(v) \in \ker \beta = \im\alpha$ due to the exactness of SES \ref{eqn: short seq}. So there exists $x_v\in X$ such that $\alpha(x_v) = g(v)$. Next, define the map $f:V\to X$ such that $v\mapsto x_v$. The map $f$ is well-defined since $\alpha$ is injective by assumption. We claim that $f$ is a $R$-module homomorphism. Indeed, for any $v,v'\in V$, let $\alpha(x_v) = g(v)$ and $\alpha(x_{v'}) = g(v')$. Then 
    \[\alpha(x_v + x_{v'}) = \alpha(x_v) + \alpha(x_{v'}) = g(x_v) + g(x_{v'}) = g(x_v + x_{v'}) \]
    This proves that $f(v+v) = x_{v+v'} = x_v + x_{v'} = f(v) + f(v')$. Next, let $r\in R$, then 
    \[\alpha(rx_v) = r\alpha(x_v) = rg(v) = g(rv)\]
    This proves that $f(rv) = x_{rv} = rx_v = rf(v)$, and so $f$ is really an $R$-module homomorphism. Finally, note that
    \[(\alpha_*(f))(v) = (\alpha\circ f) (v) = \alpha(f(v)) = \alpha(x_v) = g(v)\]
    Since $v\in V$ is arbitrary, we conclude that $\alpha_*(f)=g$. This proves that $\ker\beta_*\subseteq \im \alpha_*$, which also proved the first statement.

    For the second statement, note that the backward direction is equivalent to the first statement, which we have proven it to be true. For the forward direction, suppose that 
    \[0 \to \h_R(V,X)\xrightarrow{\alpha_*}\h_R(V,Y)\xrightarrow{\beta_*}\h_R(V,Z)\]
    is exact for all $V$. It suffices to take $V=R$, since we have that $\h_R(R, X) \cong X$ where the isomorphism is given by $f\mapsto f(1)$ This is similar for $Y$ and $Z$. Thus we have the following diagram:
    \[\begin{tikzcd}[sep=small]
	0 && {\h_R(R,X)} && {\h_R(R,Y)} && {\h_R(R,Z)} && 0 \\
	\\
	0 && X && Y && Z && 0
	\arrow[from=1-1, to=1-3]
	\arrow["{\alpha_*}", from=1-3, to=1-5]
	\arrow[tail reversed, from=1-3, to=3-3]
	\arrow["{\beta_*}", from=1-5, to=1-7]
	\arrow[tail reversed, from=1-5, to=3-5]
	\arrow[from=1-7, to=1-9]
	\arrow[tail reversed, from=1-7, to=3-7]
	\arrow[from=3-1, to=3-3]
	\arrow["\alpha"', from=3-3, to=3-5]
	\arrow["\beta"', from=3-5, to=3-7]
	\arrow[from=3-7, to=3-9]
    \end{tikzcd}\]
    where the linking between both sequences is the isomorphism defined above. We claim that the diagram is commutative. We only prove the commutativity between $\h_R(R,X)- \h_R(R,Y)-Y-X$, i.e. the first square. Let $f\in \h_R(R,X)$. Sending $f$ along the upper path we get $(\alpha_* (f))(1) = (\alpha\circ f)(1) = \alpha(f(1))$. On the other hand, sending $f$ along the lower path we get $\alpha(f(1))$. This shows commutativity in the first square. Same argument can be applied to show commutativity in the second square.
    
    We need to show that $\im \alpha = \ker \beta$. We first show $\im\alpha \subseteq
     \ker \beta$. Let $x\in X$, so there exists $f_x\in \h_R(R,X)$ such that $f_x(1) = x$. Sending $f_x$ along the first row, and due to exactness we have that 
    \[(\beta_*\circ \alpha_*)(f_x) = 0 \implies \beta\circ \alpha \circ f_x = 0\]
    Therefore $(\beta\circ \alpha \circ f_x)(1) = \beta(\alpha(f_x(1))) = \beta(\alpha(x)) = 0$. This proves that $\alpha(x)\in \ker\beta$, implying that $\im\alpha \subseteq \ker\beta$.
    
    Next, to show $\ker \beta \subseteq \im \alpha$, let $y\in \ker \beta$, i.e. $\beta(y) = 0$. Then there exists $f_y\in \h_R(R,Y)$ such that $f_y(1)= y$. Note that 
    \[(\beta_*(f_y))(1) = \beta(f_y(1)) = \beta(y) = 0 \in Z\]
    But $\beta_*(f_y)\in \h_R(R,Z)$, and there is an isomorphism between $\h_R(R,Z)$ and $Z$. Since $(\beta_*(f_y))(1) = 0 = 0(1)$, we see that $\beta_*(f_y)$ must be the zero map due to the injectivity of the isomorphism. This further implies that $f_y\in \ker \beta_*$, and by exactness in first row we get $f_y\in \im \alpha_*$. So, there exists $g_x\in \h_R(R,X)$ where $g_x(1):= x$ such that $\alpha_*(g_x) = f_y$. Thus 
    \[(\alpha_*(g_x))(1) = f_y(1)\implies (\alpha\circ g_x)(1) = y \implies \alpha(g_x(1)) = \alpha(x) = y\]
    Therefore $y\in \im \alpha$, which proves that $\ker \beta\subseteq \im \alpha$. This completes the proof.
\end{proof}

\begin{pro}
    Let $X, Y, Z$ be $R$-module. Then
    \begin{enumerate}
        \item $\h_R(X, Y\oplus Z)\cong \h_R(X,Y)\oplus \h_R(X,Z)$
        \item $\h_R(X\oplus Y, Z)\cong \h_R(X,Z)\oplus \h_R(Y,Z)$
    \end{enumerate}
\end{pro}
\begin{proof}
    We only prove the first statement. Consider a map from $\h_R(X, Y\oplus Z)$ to $\h_R(X,Y)\oplus \h_R(X,Z)$ such that 
    \[f\mapsto (\pi_Y\circ f, \pi_Z\circ f)\]
    This is an isomorphism of abelian groups.
\end{proof}

\begin{re}
    The above can be generalized to infinite direct sum, where we have that
    \[\h_R\br{\bigoplus_{i\in I}X_i, Y}\cong \prod_{i\in I} \h_R(X_i, Y)\]
\end{re}

\newpage


\subsection{Projective Modules and Introduction to Categories}

\begin{pro} [Equivalent Condition of Projective Module] \label{pro: Proj Mod}
    Let $P$ be an $R$-module. TFAE:
    \begin{enumerate}
        \item \textbf{Any} SES $0 \to X\xrightarrow{\alpha} Y\xrightarrow{\beta} Z\to 0$ gives rise to a SES 
        \[0 \to \h_R(P,X)\xrightarrow{\alpha_*}\h_R(P,Y)\xrightarrow{\beta_*}\h_R(P,Z)\to 0\]
        \item For any surjective $R$-module homomorphism $\beta:Y\to Z$ and any $R$-module homomorphism $f:P\to Z$, there exists $R$-module homomorphism $g: P\to Y$, which is called a lift, such that $\beta \circ g = f$, i.e.
        \[\begin{tikzcd}[sep=small]
	&& P \\
	\\
	Y && Z && 0
	\arrow["{\exists g}"', dashed, from=1-3, to=3-1]
	\arrow["f", from=1-3, to=3-3]
	\arrow["\beta"', from=3-1, to=3-3]
	\arrow[from=3-3, to=3-5]
\end{tikzcd}\]
        \item Every SES $0\to X\to Y\to P\to 0$ splits. In this case $P$ is a direct summand of $Y$, that is there exists an $R$-module $Y'$ such that $Y\cong Y'\oplus P$. We write $P\mid Y$.
        \item $P$ is a direct summand of a free $R$-module.
    \end{enumerate}
    If $P$ satisfies any of these equivalent conditions, we call $P$ a projective module.
\end{pro}
\begin{proof}
    $[1.\implies 2.]$ Consider the SES 
    \[0 \xto{} \ker \beta \xto{} Y \xto{\beta} Z\xto{}0\]
    By assumption this gives rise the folliowing SES:
    \[0 \xto{} \h_R(P,\ker \beta)\xto{} \h_R(P,Y) \xto{\beta_*} \h_R(P,Z) \xto{} 0\]
    where $\beta_*: g\mapsto\beta\circ g$. Note by second statement of Theorem \ref{thm: all exact} says that $\beta_*$ is surjective. Thus for any $f\in \h_R(P,Z)$ there exists a $g_f\in \h_R(P, Y)$ such that $\beta_*(g) = \beta\circ g = f$.

    $[2. \implies 3.]$ Suppose as stated by the statement. Consider the following diagram:
    \[\begin{tikzcd}[sep=small]
	&&&&&& P \\
	\\
	0 && X && Y && P && 0
	\arrow["{\exists g}"', dashed, from=1-7, to=3-5]
	\arrow["{\id_P}", tail reversed, from=1-7, to=3-7]
	\arrow[from=3-1, to=3-3]
	\arrow["\iota", from=3-3, to=3-5]
	\arrow["\pi", from=3-5, to=3-7]
	\arrow[from=3-7, to=3-9]
\end{tikzcd}\]
    where the existence of $g$ is ensure by the assumption and that $\pi\circ g = \id_P$. By the second statement of Proposition \ref{pro: SES splits}, the existence of $g$ implies that the SES splits.
    
    $[3. \implies 4.]$ Suppose as stated in the statement. Since every $R$-module is a quotient of a free module, define $F(S)$ be a free module such that $F(S)$ surjects to $P$ via map $\pi$. Then consider the SES
    \[0 \to \ker \pi \to F(S) \xto{\pi} P \to 0\]
    By the assumption, the above SES splits, and thus $F(S) = \ker\pi \oplus P$, showing that $P\mid F(S)$. 

    $[4.\implies 1.]$. Suppose as stated in the statement. Assume that we have an SES $0 \to X \xto{\alpha} Y \xto{\beta} Z\to 0$, and consider the short sequence
    \[0 \to \h_R(P,X)\xto{\alpha_*} \h_R(P,Y) \xto{\beta} \h_R(P,Z)\to 0\]
    Immediately by Theorem \ref{thm: all exact} we have that $\alpha_*$ is injective, so it suffices to show that $\beta_*$ is surjective. Take $f\in \h_R(P,Z)$. By assumption $F(S) \cong P \oplus P'$ for some free module $F(S)$ and $P'\subseteq F(S)$ is an $R$-module. Consider the following diagram:
    \[\begin{tikzcd}[sep=small]
	s && s \\
	S && {F(S)} \\
	&& P \\
	Y && Z
	\arrow[maps to, from=1-1, to=1-3]
	\arrow["\in"{marking, allow upside down}, draw=none, from=1-1, to=2-1]
    \arrow["\in"{marking, allow upside down}, draw=none, from=1-3, to=2-3]
	\arrow[hook, from=2-1, to=2-3]
	\arrow["\pi", two heads, from=2-3, to=3-3]
	\arrow["\iota"', shift right=2, curve={height=12pt}, from=3-3, to=2-3]
	\arrow["f", from=3-3, to=4-3]
	\arrow["\beta"', from=4-1, to=4-3]
    \end{tikzcd}\]
    where $\pi$ is the canonical map from $F(S)$ to $P$ and $\iota$ is the inclusion map from $P$ to $F(S)$. Note that $\pi\circ \iota = \id_P$. Next, define the map $\varphi:S \to Y$ where $s\mapsto m_s$ if $\beta(m_S) = (f\circ \pi)(s)$. The map $\varphi$ is inded well-defined since $\beta$ is surjective. Thus we now have the following diagram:
    \[\begin{tikzcd}[sep=small]
	& s && s \\
	& S && {F(S)} \\
	&&& P \\
	{m_s} & Y && Z & {(f\circ \pi)(s) = \beta(m_s)}
	\arrow[maps to, from=1-2, to=1-4]
	\arrow["\in"{marking, allow upside down}, draw=none, from=1-2, to=2-2]
	\arrow[curve={height=12pt}, maps to, from=1-2, to=4-1]
	\arrow["\in"{marking, allow upside down}, draw=none, from=1-4, to=2-4]
	\arrow[curve={height=-12pt}, maps to, from=1-4, to=4-5]
	\arrow[hook, from=2-2, to=2-4]
	\arrow["\varphi"', from=2-2, to=4-2]
	\arrow["\pi", two heads, from=2-4, to=3-4]
	\arrow["{\exists g}", dashed, from=2-4, to=4-2]
	\arrow["\iota"', shift right=2, curve={height=12pt}, from=3-4, to=2-4]
	\arrow["f", from=3-4, to=4-4]
	\arrow["\in"{marking, allow upside down}, draw=none, from=4-1, to=4-2]
	\arrow[curve={height=18pt}, maps to, from=4-1, to=4-5]
	\arrow["\beta"', from=4-2, to=4-4]
	\arrow["\in"{marking, allow upside down}, draw=none, from=4-5, to=4-4]
    \end{tikzcd}\]
    where the existence of $g$ follows from the universal property of free module. In the diagram, the upper-triangular part is commutative, and we claim that the lower-triangular part is also commutative, i.e. $\beta\circ g = f\circ \pi$. 

    We first show that $\beta\circ g = f\circ \pi$ when restricted to $S$, or more precisely, the image of $S$ in $F(S)$. This is easy, since for any $s\in S$ we have
    \[(\beta\circ g)(s) = \beta(g(s)) = \beta(\varphi(s))=\beta(m_s) = (f\circ \pi)(s)\]
    Next, consider the following commutative diagram:
    \[\begin{tikzcd}[sep=small]
	s && s \\
	S && {F(S)} \\
	\\
	&& Z
	\arrow[maps to, from=1-1, to=1-3]
	\arrow["\in"{marking, allow upside down}, draw=none, from=1-1, to=2-1]
	\arrow["\in"{marking, allow upside down}, draw=none, from=1-3, to=2-3]
	\arrow[hook, from=2-1, to=2-3]
	\arrow["{f\circ \pi = \beta \circ g}"', shift right=2, from=2-1, to=4-3]
	\arrow["{f\circ \pi}"', shift right=2, from=2-3, to=4-3]
	\arrow["{\beta\circ g}", shift left, from=2-3, to=4-3]
    \end{tikzcd}\]
    where the commutativity follows from the proven statement that $\beta \circ g = f \circ \pi$ when restricted on $S$. Then, by the uniqueness of the universal property of free module $F(S)$, we must have that $\beta\circ g = f \circ \pi$ on $F(S)$. This proves our claim. 

    Recall that we need to prove that $\beta_*$ is surjective, in particular we have been given $f\in \h_R(P,Z)$ and we want to look for its pre-image under $\beta_*$. Consider $g\circ \iota:P\to Y$, so $g\circ \iota \in \h_R(P,Y)$. Then
    \[\beta_*(g\circ \iota) = \beta\circ g \circ \iota \overset{(*)}{=} f\circ \pi \circ \iota \overset{(**)}{=} f\circ \id_P = f\]
    where in $(*)$ we use the fact that $\beta\circ g = f\circ \pi$ and in $(**)$ we use the fact that $\pi\circ \iota = \id_P$. We have shown that $g\circ \iota$ is the pre-image of $f$ under $\beta_*$, thus showing that $\beta_*$ is surjective. The proof is then completed. 
\end{proof}

\begin{re}
    As shown and stated previously in Proposition \ref{pro: inj implies inj}, if $\beta: X \to Y$ is exact, then $\beta_*: \h_R(V,X) \to \h_R(V,Y)$ is exact for any $V$, but this statement need not holds when we replace injectivity with surjectivity.  In particular, we have the statement: Let $V$ be an $R$-module. Then TFAE
    \begin{itemize}
        \item $V$ is projective.
        \item the SES $Y \xto{\beta} Z \to 0$ gives rise to the SES $\h_R (V,Y) \xto{\beta_*} \h_R(V,Z)\to 0$.
    \end{itemize}
    which is a direct consequence of the first statement of Proposition \ref{pro: Proj Mod}.
\end{re}

\medskip

\begin{cor}
    \hfill

    \begin{enumerate}
        \item Free modules are projective.
        \item A (finitely generated) $R$-module is projective if and only if it is a direct summand of a (finitely generated) free module.
        \item Direct sum of projective module is projective.
        \item Every module is a quotient of projective module. 
    \end{enumerate}
\end{cor}
\begin{proof}
    For the first statement, if $F$ is a free module, since $F\cong F\oplus 0$, so $F$ is projective.

    For the second statement, the statement is true by Proposition \ref{pro: Proj Mod}(4.). We check the finitely generated part. Suppose that $P$ is finitely generated, then there exists $S$ finite cardinality such that $F(S)$ surjects onto $P$, so $P\mid F(S)$. For the converse, if $P\mid F(S)$ where $S$ has finite cardinality, then $F(S)$ surjects to $P$ by the canonical map, $\pi$, so $P$ is finitely generated by the image of $\pi(S)$.

    Third statement is a tutorial question.

    For fourth statement, every module is a quotient of free module, and is thus a quotient of projective module.
\end{proof}

\begin{re}
    Projective module is nice. One of the reasons is that, according to Proposition \ref{pro: Proj Mod}, for a projective module $P$, it suffices to only discuss on its $\h$ set, i.e. we only have to talk about maps. It might seems complicated, but this provides us a huge space to carry out abstraction. In the light of this, we introduce some basic categorical notation.
\end{re}

\medskip

\begin{defn} [Category]
    A category $\cat$ consists of the following:
    \begin{enumerate}
        \item A class of objects $\obj(\mathcal C)$
        \item For any two objects $X$ and $Y$, we have a class of morphisms (i.e. maps) $\mor_\cat(X,Y)$
        \item For any objects $X$, $Y$, and $Z$, we have a binary operation $\mor_\cat(X,Y)\times \mor_\cat (Y,Z)\to \mor_\cat(X,Z)$ such that $(f,g)\mapsto g\circ f$, such that
        \begin{itemize}
            \item the operation is associative
            \item $\mor_\cat(X,X)$ contains an identity $1_X$ such that for any $g\in \mor_\cat(X,Y)$ and $h\in \mor_\cat(Z,X)$, we have $g\circ 1_X = g$ and $1_X\circ h = h$
        \end{itemize}
    \end{enumerate}
\end{defn}

\medskip 

\begin{defn} [Covariant functor]
    Let $\cat$ and $\mathcal D$ be a categories. A covariant functor $\f:\cat \to \mathcal D$ consists of the following things:
    \begin{enumerate}
        \item For any object $X\in \obj(\cat)$, we have an object $\f(X)\in \mathcal D$
        \item For any morphism $\alpha:X\to Y$ of $\cat$, we have a morphism $\f(\alpha):\f(X)\to \f(Y)$ of $\mathcal D$ such that the following holds:
        \begin{itemize}
            \item $\f(1_X) = 1_{\f(X)}$
            \item If we have the commutative diagram
            \[\begin{tikzcd}[sep=small]
	        X && Y \\
	        \\
	        && Z
	        \arrow["\alpha", from=1-1, to=1-3]
	        \arrow["{\beta\circ \alpha}"', from=1-1, to=3-3]
	        \arrow["\beta", from=1-3, to=3-3]
            \end{tikzcd}\]
            then we have the commutative diagram:
            \[\begin{tikzcd}[sep=small]
	        {\f(X)} && {\f(Y)} \\
	        \\
	        && {\f(Z)}
	        \arrow["{\f(\alpha)}", from=1-1, to=1-3]
	        \arrow["{\f(\beta\circ \alpha) = \f(\beta) \circ \f(\alpha)}"', shift right=2, from=1-1, to=3-3]
	        \arrow["{\f(\beta)}", from=1-3, to=3-3]
            \end{tikzcd}\]
        \end{itemize}
    \end{enumerate}
\end{defn}

\medskip 

\begin{defn} [Contravariant functor]
    Let $\cat$ and $\mathcal D$ be a categories. A contravariant functor $\f:\cat \to \mathcal D$ consists of the following things:
    \begin{enumerate}
        \item For any object $X\in \obj(\cat)$, we have an object $\f(X)\in \mathcal D$
        
        \item For any morphism $\alpha:X\to Y$ of $\cat$, we have a morphism $\f(\alpha):\f(Y)\to \f(X)$ of $\mathcal D$ such that the following holds:
        \begin{itemize}
            \item $\f(1_X) = 1_{\f(X)}$
            \item If we have the commutative diagram
            \[\begin{tikzcd}[sep=small]
	        X && Y \\
	        \\
	        && Z
	        \arrow["\alpha", from=1-1, to=1-3]
	        \arrow["{\beta\circ \alpha}"', from=1-1, to=3-3]
	        \arrow["\beta", from=1-3, to=3-3]
            \end{tikzcd}\]
            then we have the commutative diagram:
            \[\begin{tikzcd}[sep=small]
	        {\f(X)} && {\f(Y)} \\
	        \\
	        && {\f(Z)}
	        \arrow["{\f(\alpha)}"', from=1-3, to=1-1]
	        \arrow["{\f(\beta\circ \alpha) = \f(\alpha) \circ \f(\beta)}", shift left=2, from=3-3, to=1-1]
	        \arrow["{\f(\beta)}"', from=3-3, to=1-3]
            \end{tikzcd}\]
        \end{itemize}
    \end{enumerate}
\end{defn}

\medskip

\begin{cor}
    For any $R$-module $V$ the following
    \[\f:=\h_R(V,-):R\text{-}mod \to Ab\]
    is a left exact covariant functor. Moreover, the functor is exact if and only if $V$ is projective.
\end{cor}

\begin{proof}
    Let $X$ be an $R$-module and consider $\h_R(V,X)$. Let $\alpha:X\to Y$ be an $R$-module homomorphism and denote $\f(\alpha)= \alpha_*: \h_R(V,X)\to \h_R(V,Y)$. We prove the axiom for a covariant functor. Suppose we have $X\xto\alpha Y \xto\beta Z$, and so we have $\beta\circ \alpha: X\to Z$. As shown previously, this gives the following sequence:
    \[\h_R(V,X)\xto {\alpha_*} \h_R (V,Y)\xto {\beta_*} \h_R(V,Z)\]
    where clearly $\beta_*\circ \alpha_* = (\beta\circ \alpha)_*$. This proves the second axiom. Next, for the first axiom, define $\id_X:X\to X$ be the identity map of $X$. Then $(\id_X)_*: f\mapsto \id\circ f = f$, which shows that $(\id_X)_* = \id_{\h_R(X,X)}$. Therefore $\f$ is a covariant functor.

    For the second part of the statement, it follows directly from the definition of projective modules. This completes the proof.
\end{proof}
\begin{ex}
    \hfill

    \begin{enumerate}
        \item Let $F$ be a field, an $F$-module $V$ is a vector space over $F$ and hence $V$ has a basis, i.e. $B\subseteq V$ such that $V$ is free on $B$. So $V$ is then projective, since free implies projective. In particular, all $F$-module are free and projective.
        \item Let $V$ be a $\Z$-module. Suppose that $V$ consists an non-zero element $x$ of finite order $n$. We claim that $V$ is not free. Suppose not, then $V$ is free on a set $B\subseteq V$, then 
        \[x = r_1 b_1 + \dots, + r_mb_m\]
        where $r_i\in \Z$ and $b_i\in B$. But since order of $x$ is $n$, so we have
        \[x = (n+1)x = x = r_1(n+1)b_1 + \dots + r_m(n+1)b_m\]
        Note $(n+1)r_i \neq r_i$ in $\Z$, so this gives non-unique representation of $x$. Since projective $\Z$-module are direct summand of free $\Z$-modules, any projective $\Z$-module does not contain non-zero element of finite order.
        \item The previous example shows that, in general, finite abelian groups are not projective.
        \item The $\Z$-module $\Q/\Z$ is torsion. I.e for any $x\in \Q/\Z$, there exists $n\in \Z$ such that $nx=0$. In particular, if $x = r/s + \Z$, take $n=s$ and we have
        \[s\br{\frac{r}{s} + \Z} = r + \Z = \Z\]
        So $\Q/\Z$ is not projective. The SES
        \[0 \to \Z\to \Q \to \Q/\Z \to 0\]
        does not split. If not, then $\Q/\Z$ is a direct summand of $\Q$. Since $\Q/\Z$ contains non-zero element of finite order, it implies that so is $\Q$, which is clearly contradiction.
        \item $\Q$ is not a projective $\Z$-module.
        \item A finitely generated module over a $\Z$ is projective if and only if it is free. Free implies projective is clear. Assuming that it is projective. By the Classification of Finitely Generated Module over PID, an finitely generated $\Z$-module $M$ is isomorphic with 
        \[\Z^n \bigoplus \text{ direct sum of finite copies of finite cyclic groups}\]
        The direct sum of finite cyclic groups part contains elements of finite order. By assumption $M$ is projective, so there must be no non-zero elements of finite order in $M$, implying that the direct sum of finite cyclic groups must be $0$. This shows that 
        \[M \cong \Z^n\]
        So $M$ is free.
    \end{enumerate}
\end{ex}

\subsection{Injective Modules}

\begin{thm}
    Let $V$ be an $R$-module and consider the sequence
    \[X \xto{\alpha} Y\xto{\beta} Z \to 0\]
    If the above sequence is exact, then the following sequence is also exact:
    \[0 \to \h_R(Z,V) \xto{\beta^*} \h_R(Y,V) \xto{\alpha^*} \h_R(X,V)\]
    where $\beta^*: f\mapsto f\circ \beta$ and $\alpha^*: f\mapsto f\circ \alpha$. Furthermore, the sequance $0 \to X \xto{\alpha} Y\xto{\beta} Z \to 0$ is exact if and only if the following
    \[0 \to \h_R(Z,V) \xto{\beta^*} \h_R(Y,V) \xto{\alpha^*} \h_R(X,V)\]
    is exact for every $V$.
\end{thm}
\begin{proof}
    Suppose that $X\xto{\alpha} Y \xto{\beta} Z\to 0$ is exact. We first show that $\beta^*$ is injective. Let $f\in \ker \beta^*$, so $\beta^*(f) = f\circ \beta = 0$ is the zero map. In other words, for all $y\in Y$ we have $f(\beta(y))=0$. By assumption $\beta$ is surjective, thus let for all $z\in Z$, let $y_z\in Y$ be such that $\beta(y_z) = z$, and so $f(z) = f(\beta(y)) = 0$. This shows that $f$ must be a zero map.

    Next, we show that $\im \beta^* \subseteq \ker \alpha^*$. This is simple, simply follow:
    \[\alpha^*(\beta^*(f)) = \alpha^* (f\circ \beta) = f\circ \beta \circ \alpha = f\circ (\beta \circ \alpha) \overset{(*)}{=} f\circ (0) = 0\]
    where at $(*)$ we apply the assumption that $\beta\circ \alpha=0$ given by exactness.

    Then, we show that $\ker \alpha^* \subseteq \im \beta^*$. Let $g\in \ker \alpha^*$, i.e. $\alpha^* (g) = g \circ \alpha = 0$ is the zero map. Note that $\beta$ is surjective, so for every $z\in Z$ we let $y_z\in Y$ be such that $\beta(y_z)=z$. Then, we define $f:Z\to V$ such that $z\mapsto g(y_z)$. We claim that $\beta^*(f) = g$. 
    \begin{itemize}
        \item First we show $f$ is well-defined. Suppose given $z$, let $y_z, y_z'\in Y$ be such that $\beta(y_z')=z = \beta(y_z)$. This implies $\beta(y_z-y_z')=0$ and so $y_z - y_z'\in \ker \beta = \im \alpha$. So let $x\in X$ such that $\alpha(x) = y_z - y_z'$. Recall that $g\circ \alpha$ is the zero map by assumption, thus $g(y_z-y_z') = g(\alpha(x)) = 0$. This shows that $g(y_z) = g(y_z')$, showing that $f$ is indeed well-defined.
        \item Next we show that $f$ is an $R$-module homomorphism. Let $z, z'\in Z$ and let $y_z, y_{z'}\in Z$ such that $\beta(y_z) = z$ and $\beta(y_{z'}) = z'$, implying that $\beta(y_z+y_{z'}) = z+z'$. Therefore, by definition of the map $f$, we have $f(z) = g(y)$ and $f(z') = g(y_{z'})$. To show additivity:
        \[f(z+z') = g(y_z+y_{z'}) \overset{(**)}{=} g(y_z) + g(y_{z'}) = f(z) + f(z')\]
        where at $(**)$ it is valid to split since by assumption $g$ is $R$-modole homomorphism by assumption. Next, for the action, let $r\in R$. Note $\beta(ry_z) = r\beta(y_z) = rz$, so
        \[ f(rz) = g(ry_z) = rg(y_z) = f(z)\]
        This shows that $f$ is a $R$-module homomorphism.
        \item Lastly, suppose again given $z$, let $y_z\in Y$ be such that $\beta(y_z) = z$. So $(\beta^*(f))(y_z) = f(\beta(y_z)) = f(z) = g(y)$. This shows that $\beta^*(f) = g$.
    \end{itemize}
    This proves the first part of the statement.

    For the second statement, the forward direction is a immediate result of the first statement, so it suffices to show the backward direction is true. Suppose as assumed in the statement. First we show that $\beta$ is surjective. Consider 
    \[V := Z/\im \beta\]
    In other words, take $V= \text{coker}\ \beta$. Let $\pi:Z\to V$ be the canonical surjection. Then $\beta^*(\pi) = \pi\circ \beta$, so $(\beta^*(\pi))(y) = (\pi\circ \beta)(y) = \pi(\beta(y))$. Since $\beta(y)\in \im \beta$. so $(\beta^*(\pi))(y) = \overline{0}$. This means that $\beta^*(\pi)=0$. Since $\beta^*$ is injective, so $\pi=0$ which means that $V=0$, and so $Z = \im \beta$. This shows that $\beta$ is surjective.
    
    Next, we show that $\im\alpha \subseteq \ker \beta$. Take $V=Z$, and let $\id_Z$ be the identical map of $Z$. By the assumption of exactness we have $\alpha^*\circ \beta^*$ is zero map. So
    \[(\alpha^* \circ \beta^*)(\id_Z) = 0 \implies \beta\circ \alpha \circ \id_Z = 0\implies \beta\circ \alpha = 0\]
    This shows that $\im \alpha\subseteq \ker \beta$.

    Lastly, we show that $\ker\beta\subseteq \im \alpha$. Let $V=\text{coker}\ \alpha = Y/\im\alpha$. Let $\pi:Y\to V$ be the canonical surjection. Similar to previous argument, we see that $\alpha^*(\pi) = \pi \circ \alpha = 0$ is the zero map, so $\pi \in \ker \alpha^* = \im \beta^*$ due to exactness. Let $g\in Z\to V$ such that $\beta^*(g) = g\circ \beta=\pi$. Then for every $y\in \ker \beta$, we see that
    \[(g\circ \beta)(y) = \pi(y) \implies g(\beta(y)) = \pi(y) \implies g(0) = \pi(y) \implies \pi(y) = 0\]
    By definition of canonical surjection $\pi$, we have that $y\in \im \alpha$. So $\ker \beta \subseteq \im \alpha$. The proof is completed.
\end{proof}

\begin{defn} [Injective modules]
    Let $Q$ be an $R$-module. We say that $Q$ is injective if for any injective $R$-homomorphism $\varphi:Z\to Y$ and $R$-homomorphism $g:Z\to Q$, there exists $f:Y\to Q$ such that $f\circ \varphi = g$, i.e. we have the following commutative diagram
    \[\begin{tikzcd} [sep = small]
	&& Q \\
	\\
	Y && Z
	\arrow["{\exists f}", dashed, from=3-1, to=1-3]
	\arrow["g"', from=3-3, to=1-3]
	\arrow["\varphi", hook', from=3-3, to=3-1]
    \end{tikzcd}\]
\end{defn}

\begin{pro} \label{pro: Baer}
    Let $Q$ be an $R$-module.
    \begin{enumerate}
        \item (Baer's Criterion) The module $Q$ is injective if and only if for every left ideal $I$ of $R$ and any $R$-module homomorphism $g:I\to Q$, there exists $R$-module homomorphism $f:R\to Q$ such that $g = f\circ \iota$, where $\iota:I\hookrightarrow R$ is the inclusion map.
        \item If $R$ is a PID, then $Q$ is injective if and only if $Q$ is divisible (i.e. for every $r\in R$ is non-zero, we have $rQ=Q$). When $R$ is a PID, quotients of injective $R$-modules are injective.
    \end{enumerate}
\end{pro}
\begin{proof}
    The forward direction simply follows from the definition of injective module, thus we are done. To prove the backward statement, suppose as stated in the condition, and we want to show that module $Q$ is injective.

    First, let $\alpha:Z\hookrightarrow Y$ be the inclusion map, and let $\beta:Z\to Q$ be a $R$-module homomorphism. Define 
    \[\Omega = \sbr{(f',Y'):\im \alpha \subseteq Y' \subseteq Y \text{ and } f':Y'\to Q \text{ s.t. } f'\circ \alpha = \beta, f' \text{ is $R$-module homomorphism}}\]
    Note that $\Omega$ is non-empty since we can check that $(\beta\circ \alpha^{-1}, \im \alpha)\in \Omega$. We now impose a partial order to $\Omega$, where we define the partial order
    \[(f',Y')\leq (f'',Y'') \iff Y'\subseteq Y'' \text{ and }f''\mid_{Y'} = f'\]
    We show that it is indeed a partial order:
    \begin{itemize}
        \item Firstly, it is clear that $(f',Y') = (f',Y')$.
        \item Next, suppose that $(f',Y')\leq (f'',Y'')$ and $(f'',Y'')\leq (f',Y')$. This says that $Y'\subseteq Y''\subseteq Y'$, so $Y' = Y''$. Also, we see that $f' = f''\mid _{Y'} = f''\mid _{Y''} = f''$. This concludes that $(f',Y') = (f'',Y'')$.
        \item Lastly, suppose that $(f',Y')\leq (f'',Y'') \leq (f''',Y''')$. Then we have $Y'\subseteq Y''\subseteq Y'''$, implying that $Y'\subseteq Y'''$. Also, note that $f' = f''\mid _{Y'} = \br{f'''\mid _{Y''}} \mid_{Y'} = f'''\mid_{Y'}$. This shows that $(f',Y')\le (f''', Y''')$.
    \end{itemize}
    Therefore the relation $\le$ is indeed a partial order. (As part of a tutorial question) we see $\Omega$ satisfies the hypothesis for applying Zorn's Lemma, and thus $\Omega$ has a maximum element, say $(f,Y')$. 
    
    We claim that $Y' = Y$. Suppose not, then $Y'\subsetneq Y$, and let $m\in Y\setminus Y'$. Define $I = \sbr{r\in R:rm\in Y'}$. This is clearly an ideal of $R$. Let $g:I\to Q$ such that $g(r) = f(rm)$. We show that it is an $R$-module homomorphism:
    \begin{itemize}
        \item First, note $g(r+ r') = f((r+r')m) = f(rm + r'm) = f(rm) + f(r'm) = g(r) + g(r')$.
        \item Next, let $s\in R$, we have $g(sr) = f((sr)m) = f(s(rm)) = sf(rm) = sg(r)$.
    \end{itemize}
    So $g$ is an $R$-module homomorphism. By assumption, the exists an $R$-module homomorphism $h:R\to Q$ such that $h\circ \iota = g$, where $\iota:I\to R$ is the inclusion map.

    Define the map $\gamma:Y' + Rm \to Q$ by $\gamma(m'+rm) = f(m') + h(r)$ where $m'\in Y'$ and $r\in R$. We show that $(\gamma, Y' + Rm)\in \Omega$
    \begin{itemize}
        \item We first show that $\gamma$ is well-defined. Let $m_1' + r_1 m = m_2' + r_2 m$. Then $(r_2-r_1) m = m_1' - m_2' \in Y'$, implying that $(r_2-r_1)\in I$. Recall that $h\circ \iota = g$, so we have
        \[h(r_2-r_1) = (h\circ \iota) (r_2 - r_1) = g(r_2-r_1) = f((r_2-r_1)m) = f(m_1'-m_2')\]
        and thus $h(r_2)-h(r_1) = h(r_2-r_1) = f(m_1'-m_2') = f(m_1')-f(m_2')$. By rearranging we see that $\gamma$ is well-defined.
        \item We show that $\gamma$ is $R$-module homomorphism. Note
        \begin{align*}
            \gamma((m_1' + r_1 m) + (m_2' + r_2m))
            &= \gamma((m_1' + m_2') + (r_1 + r_2)m)\\
            &= f(m_1' + m_2') + h(r_1 + r_2)\\
            &= f(m_1') + f(m_2') + h(r_1) + h(r_2)\\
            &= \gamma(m_1' + r_1 m) + \gamma(m_2' + r_2m)
        \end{align*}
        and also 
        \begin{align*}
            \gamma(s(m' + rm))
            &= \gamma(sm' + (sr)m)\\
            &= f(sm') + h(sr)\\
            &= s f(m') + s h(r) \\
            &= s(f(m') + h(r))\\
            &= s \gamma(m' + rm)
        \end{align*}
        This shows that $\gamma$ is an $R$-module homomorphism.
        \item Lastly, we show that $\gamma \circ \alpha = \beta$. Since $(f,Y')\in\Omega$, by definition it satisfies $\im\alpha\subseteq Y'\subseteq Y$ and $f\circ \alpha=\beta$. Note $\alpha:Z\to Y$, so for all $z\in Z$ we have $\alpha(z)\in \im \alpha \subseteq Y'$, thus we can express $\alpha(z)=m' + 0m$ for $m'\in Y'$ and $0\in R$. Therefore
        \[\gamma(\alpha(z)) = \gamma(m'+0m) = f(m')+h(0) = f(m') = f(\alpha(z)) = (f\circ\alpha)(z) = \beta(z)\]
    \end{itemize} 
    We claim that $(\gamma, Y' + Rm)$ is strictly larger than the maximal element $(f,Y')$ obtained from the Zorn's Lemma. Clearly $Y'\subsetneq Y'+Rm$. Also, note $\gamma \mid_{Y'} = f$. This contradicts to the maximality of $(f,Y')$, thus $Y' = Y$, and we have obtained an extended map $f:Y\to Q$. This completes the proof for the first statement.

    For the second statement, we first show the forward direction: let $Q$ be injective. Let $r\in R$ be non-zero. It is clear that $rQ\subseteq Q$, so we want to show that $rQ\subseteq Q$. For any $m\in Q$, define $g:(r)\to Q$ where $r\mapsto m$ and so $sr\mapsto sm$. By Baer's criterion, since $Q$ is injective, there exists $f:R\to Q$ such that $f\circ \iota = g$ where $\iota: (r) \hookrightarrow R$ is the inclusion map. In particular $f(r) = (f\circ \iota)(r) = g(r) = m$. Note $f$ is an $R$-module homomorphism, so $m = f(r) = rf(1)\in rQ$, implying that $m\in rQ$. This shows that $Q$ is divisible.

    For the backward direction, suppose that $Q$ is divisible, and we want to show that $Q$ is injective. Let $I\triangleleft R$ and $g:I\to Q$ be an $R$-module homomorphism. Since $R$ is PID, so $I = (r)$ for some $r\in I$. If $r=0$, then take $f:R\to Q$ is the zero map, and we have $f\circ \iota = 0 = g$. So the statement holds for when $r=0$. Next, assume $r\neq 0$, since $Q$ is divisible we have $rQ = Q$. We want to construct $f:R \to Q = rQ$ such that $f\circ \iota = g$. Note $g(r) \in Q = rQ$, so let $g(r) = rm$ for some $m\in Q$, and we define $f:R\to Q$ where $1\mapsto m$. This implicitly defines for other $s\in R$ where $s \mapsto sm$. Note $f$ is certainly well-defined, and we now show that $f$ is a $R$-module homomorphism:
    \begin{itemize}
        \item $f(s+s') = (s+s')m = sm + s'm = f(s) + f(s')$. 
        \item $f(s\cdot s') = f(ss') = (ss')m = s(s'm) = sf'(s)$. 
    \end{itemize}
    So $f$ is a $R$-module homomorphism. Lastly, see that $(f\circ \iota)(r) = f(r) = rm = g(r)$. Since $I= (r)$, so it implies $f\circ \iota = g$. By definition of injective modules, we have shown that $Q$ is injective.
    
    Finally, let $Q$ be an injective $R$-module, and $Q'\subseteq Q$. Let $r\in R$ is non-zero element, observe that 
    \[r\br{Q/Q'} = rQ/Q' = Q/Q'\]
    Since $R$ is PID and $Q/Q'$ is divisible, we have that $Q/Q'$ is injective. This completes the whole proof.
\end{proof}

%  Baer's criterion for projective module? Reversing all the arrows?

\begin{ex}
    \hfill

    \begin{enumerate}
        \item $Q$ is injective $\Z$-module because $\Q$ is divisible. However $\Z$ is not injective $\Z$-module because $2\Z \neq \Z$. But $\Z$ is a free module, so $\Q/\Z$ is injective module. Recall we have seen that $\Q/\Z$ is not projective.
        \item Let $F$ be a field. Then any $F$-module is injective.
        \item Over any ring $R$, any injective $R$-module is divisible. 
    \end{enumerate}
\end{ex}

\medskip

\begin{cor} \label{cor: Z-mod sub inj}
    Any $\Z$-module is a sub-module of an injective $\Z$-module.
\end{cor}
\begin{proof}
    Let $M$ be a $\Z$-module and let $F(A)$ surjects onto $M$ via $\pi$. This induces an isomorphism $\varphi$ such that
    \[F(A)/\ker \pi \overset{\varphi}{\cong} M\]
    Let $Q = \bigoplus_{a\in A} \Q$ be a free $Q$-module. Consider $\Q$ as a $\Z$-module. For any $n\in \Z$ is non-zero, and $\br{\frac{r_a}{s_a}}_{a\in A} \in Q$, we have
    \[\br{\frac{r_a}{s_a}}_{a\in A} = n\br{\frac{r_a}{ns_a}}_{a\in A}\]
    So $Q$ is injective $\Z$-module. 

    Next, observe that $\ker\pi \subseteq F(A) \cong \bigoplus_{a\in A} \Z$ and we can embeed $\bigoplus_{a\in A} \Z$ into $Q$ via the following inclusion map
    \[\iota:(n_a)_{a\in A} \mapsto \br{\frac{n_a}{1}}_{a\in A}\]
    Since $Q$ is injective, so $Q/\ker \pi$ is injective by the second statement of Proposition \ref{pro: Baer} (note $\Z$ is PID). Together, we see that 
    \[M\cong \frac{F(A)}{\ker\pi} \overset{\iota'}{\hookrightarrow} \frac{Q}{\ker\pi}\]
    where the inclusion $\iota'$ is induced by $\iota$. This proves that $M$, as a $\Z$-module, is a submodule of $Q/\ker \pi$, an injective $\Z$-module.
\end{proof}

\begin{thm}
    Any $R$-module is a sub-module of an injective $R$-module.
\end{thm}
\begin{proof}
    Let $M$ be an $R$-module. By treating $M$ as a $Z$-module, by Corollary \ref{cor: Z-mod sub inj}, it is a sub-module of an injective $\Z$-module, say $Q$. Note that $\h_Z(R,M) \subseteq \h_Z(R,Q)$ due to the following arguments:
    \begin{itemize}
        \item Since $M\subseteq Q$, we have the exact sequence $0\to M \xto{\iota} Q/M$.
        \item This gives rise to the exact sequence $0 \to \h_\Z(R,M) \xto{\iota_*} \h_\Z(R,Q) \to \h_\Z(R,Q/M)$. 
        \item This shows that $\h_\Z(R,M) \subseteq \h_\Z(R,Q)$.
    \end{itemize}
    On the other hand, recall that $\h_R(R,M) \cong M$, and since it is clear that $\h_R(R,M) \subseteq \h_\Z(R,M)$, we have the following:
    \[M\cong \h_R(R,M)\subseteq \h_\Z(R,M) \subseteq \h_\Z(R,Q) \implies M\subseteq \h_\Z(R,Q)\]
    And we will show that $\h_\Z(R,Q)$ is an injective $R$-module.
    
    Firstly , note we can view $\h_\Z(R,Q)$ as an $R$-module via the $R$-action $(r\cdot \varphi)(x) = \varphi(xr)$, which is valid since we can impose the $(\Z,R)$-bimodule struction to $R$. 
    
    Next, to show that $\h_\Z(R,Q)$ is an injective $R$-module, let $X$ and $Y$ be any $R$-modules, and let $\alpha:X\hookrightarrow Y$ be an injective $R$-module homomorphism, and let $g:X \to \h_\Z(R,Q)$ be an $R$-module homomorphism. We want to show that there exists a $R$-module homomorphism that commutes the following diagram:
    \[\begin{tikzcd} [sep = small]
	X && Y \\
	\\
	{\h_\Z(R,Q)}
	\arrow["\alpha", hook, from=1-1, to=1-3]
	\arrow["g"', from=1-1, to=3-1]
	\arrow[dashed, from=1-3, to=3-1]
    \end{tikzcd}\]
    Define $g':X\to Q$ where $x\mapsto (g(x))(1)$. We claim that $g'$ is a $\Z$-module homomorphism:
    \begin{itemize}
        \item It suffices to show that it is an abelian group homomorphism. By definition $X$ and $Q$ are abelian groups. Note then $g'(x+x') = (g(x+x'))(1) = (g(x) + g(x'))(1) = (g(x))(1) + (g(x'))(1) = g'(x) + g'(x')$. Thus $g'$ is a $\Z$-module homomorphism.
    \end{itemize}
    By assumption $Q$ is an injective $\Z$-homomorphism, so there exists $f'$ is a $\Z$-module homomorphism such that the following diagram commutes:
    \[\begin{tikzcd} [sep = small]
	X && Y \\
	\\
	{Q}
	\arrow["\alpha", hook, from=1-1, to=1-3]
	\arrow["g'"', from=1-1, to=3-1]
	\arrow["\exists f'",dashed, from=1-3, to=3-1]
    \end{tikzcd}\]
    In particular $f'\circ \alpha = g'$. Define $f:Y\to \h_Z(R,Q)$ by $y \mapsto f_y$ such that $f_y$ is defined by $f_y:r\mapsto f'(ry)$. We show that $f$ is an $R$-module homomorphism:
    \begin{itemize}
        \item We claim $f$ is well-defined, i.e. $f_y \in \h_\Z(R,Q)$. It suffices to show that $f_y$ is an abelian group homomorphism. Note $f_y(r+r') = f'((r+r')y) = f'(ry + r'y) = f'(ry) + f(r'y) = f_y(r) + f_y(r')$. Thus $f(y) = f_y$ is indeed a $\Z$-module homomorphism.
        \item To show additivity: $(f(y+y'))(r) = f_{y+y'}(r) = f'(r(y+y')) = f'(ry + ry') = f'(ry) + f'(ry') = f_y(r) + f_{y'}(r) = (f_y + f_{y'})(r) = (f(y) + f(y'))(r)$.
        \item To show it respect $R$-action: $(s\cdot f(y))(r) = (s\cdot f_y)(r) = f_y(rs) = f'(rsy) = f'(r(sy)) = f_{sy}(r)=(f(sy))(r)$.
    \end{itemize}
    Lastly, we show that $f\circ \alpha= g$, i.e. we want to show that $((f\circ \alpha)(x))(r) = (g(x))(r)$ where $x\in X$ and $r\in R$. Note 
    \begin{align*}
        ((f\circ \alpha)(x))(r) 
        &= (f(\alpha(x)))(r) \\
        &= f_{\alpha(x)}(r) \\
        &= f'(r\alpha(x))\\
        &= f'(\alpha(rx)) \\
        &=(f'\circ \alpha)(rx) \\
        &= g'(rx)\\
        &= (g(rx))(1)\\
        &= (r\cdot g(x))(1) \\
        &= (g(x))(1\cdot r)\\
        &= (g(x)) (r)
    \end{align*}
    In other words, we have establish the following commutative diagram:
    \[\begin{tikzcd} [sep = small]
	X && Y \\
	\\
	{\h_\Z(R,Q)}
	\arrow["\alpha", hook, from=1-1, to=1-3]
	\arrow["g"', from=1-1, to=3-1]
	\arrow["\exists f", dashed, from=1-3, to=3-1]
    \end{tikzcd}\]
    Therefore $\h_Z(R,Q)$ is an injective $R$-module. This completes the proof.
\end{proof}




\begin{pro}
    Let $I$ be an $R$-module. TFAE:
    \begin{enumerate}
        \item $I$ is injective.
        \item For any SES $0\to X\xto\alpha Y\xto \beta Z\to 0$, we have SES 
        \[0 \to \h_R(X,I) \xto{\beta^*} \h_R(Y,I) \xto{\alpha^*} \h_R(X,I) \to 0\]
        \item If $I$ is isomorphic with a submodule of $Y$, then the following SES splits:
        \[0 \to I \to Y\to Y/I \to 0\]
        And hence $I\mid Y$.
    \end{enumerate}
\end{pro}
\begin{proof}
    Tutorial questions.
\end{proof}

\begin{cor}
    Let $V$ be an $R$-module. Then
    \[\mathcal F:= \h_R(-, V): R\text{-mod} \to \text{Ab}\]
    is a left exact contravariant functor, i.e. the SES $0\to X\xto \alpha Y\xto \beta Z\to 0$ gives rise to the exact sequence
    \[0 \to \h_R(Z,V) \xto {\beta^*} \h_R(Y,V) \xto {\alpha^*} \h_R(X,V)\]
    Furthermore, the functor $\mathcal F$ is exact if and only if $V$ is injective.
\end{cor}

\subsection{Flat Modules}
Let $D$ be a right $R$-module. The operation
\[\mathcal F := D \ten_R -:R\text{-mod} \to \text{Ab}\]
where $_RX\mapsto D\ten_R X$ such that $(\alpha:X\to Y)\mapsto \br{(1\ten \alpha):D\ten_R X \to D\ten_R Y, d\ten x \mapsto d\ten \alpha(x)}$. The functor $\mathcal F$ is a covariant functor. 

To see this, we show all the axioms of a covariant functor hold:
\begin{itemize}
    \item For any $R$-module $X$, it is clear that $D\ten_R X$ is well-defined and is an abelian group, which lies in the category Ab of abelian group.
    \item  Define $\mathds{1}_X:X\to X$ be the identity map on $X$. By definition $\mathcal F (\mathds{1}_X) = 1\ten \mathds 1_X$ such that $1\ten \mathds{1}_X:D\ten_R X \to D\ten_R X$ defined by $d\ten x \mapsto d\ten x$. Clearly we see that $\mathcal F (\mathds{1}_X)$ is the identity map on $\mathcal F(X)$. This shows that $\mathcal F(\mathds 1_X) = \mathds 1_{\mathcal F(X)}$.
    \item Suppose we have commutative diagram
        \[\begin{tikzcd}[sep=small]
	        X && Y \\
	        \\
	        && Z
	        \arrow["\alpha", from=1-1, to=1-3]
	        \arrow["{\beta\circ \alpha}"', from=1-1, to=3-3]
	        \arrow["\beta", from=1-3, to=3-3]
        \end{tikzcd}\]
        Then we have that 
        \[\begin{tikzcd}[sep=small]
	        \mathcal F\br{X} && \mathcal F \br{Y} \\
	        \\
	        && \mathcal F \br{Z}
	        \arrow["\mathcal F (\alpha)", from=1-1, to=1-3]
	        \arrow["\mathcal F \br{\beta\circ \alpha}"', from=1-1, to=3-3]
	        \arrow["\mathcal F\br{\beta}", from=1-3, to=3-3]
        \end{tikzcd}\]
        and we examine that it is commutative. By following definition we see 
        \[\f(\beta\circ \alpha) = 1\ten(\beta\circ \alpha) = (1\circ 1)\ten (\beta\circ \alpha) = (1\ten \beta) \circ (1\ten \alpha) = \f(\beta)\f(\alpha)\]
        This shows that the diagram is commutative.
\end{itemize}

Moreover, if $D$ is a $(S,R)$-bimodule, then $\mathcal F:X\mapsto D\ten_R X$ is a functor that maps from category of $R$-mod to category of $S$-mod.

\medskip

\begin{thm}
    Let $D$ be an $(S,R)$-bimodule and $X$, $Y$, $Z$ be left $R$-module. If $ X\xto\alpha Y\xto\beta Z$ is exact, then 
    \begin{equation} \label{eqn: thm: flat mod exact}
        D\ten_R X\xto {1\ten \alpha} D\ten_R Y\xto {1\ten\beta} D\ten_R Z \to 0
    \end{equation}
    is exact. Moreover  $X\xto\alpha Y\xto\beta Z$ is exact if and only if (\ref{eqn: thm: flat mod exact}) is exact for all $D$.
\end{thm}
\begin{proof}
    Assume as supposed in the statement. For the first statement we show the following:
    \begin{enumerate}
        \item $(1\ten \beta)$ is surjective. Let $d\ten z\in D\ten_R Z$. By assumption $\beta$ is surjective, so there exists $y\in Y$ such that $\beta(y) = z$. Observe then that $(1\ten\beta)(d\ten y) = d\ten \beta(y) = d\ten z$.
        \item $\im (1\ten \alpha)\subseteq \ker(1\ten \beta)$. First observe that by definition $\beta\circ \alpha = 0$. Thsu $(1\ten \beta) (1\ten\alpha) = 1\ten (\beta\circ \alpha) = 1\ten 0 = 0$. This shows that $\im(1\ten \alpha) \subseteq \ker (1\ten \beta)$.
        \item $\ker (1\ten \beta)\subseteq \im(1\ten \alpha)$. To prove this, recall we have proved that $\im(1\ten \alpha) \subseteq \ker(1\ten \beta)$, this implies that we have the surjection: 
        \[\pi: (D\ten_R Y)/\im(1\ten \alpha) {\twoheadrightarrow} (D\ten_R Y)/\ker(1\ten \beta)\cong D\ten_R Z\]
        Our goal is to show that $\pi$ is injective. First, by assumption $\beta$ is surjective, so for each $z\in Z$ we define $y_z\in Z$ be such that $\beta(y)=z$. Next define the map $\gamma:D\times Z\to (D\ten_R Y)/\im(1\ten\alpha)$ where $(d,z)\mapsto (d\ten y_z) + \im(1\ten \alpha) =: \overline{d\ten y_z}$.
        \begin{itemize}
            \item We claim that $\pi$ is well-defined. Let $y'$ and $y$ be such that $\beta(y') = z = \beta(y)$. Note then $y-y'\in \ker\beta = \im\alpha$ due to exactness. Thus $d\ten y- d\ten y' = d\ten (y-y') \in \im(1\ten \alpha)$. This shows that regardless of the choice of $y_z$ is $y$ or $y'$, we always have that 
            \[ \overline{d\ten y}  = \gamma(d,z) = \overline{d\ten y'}\]
            \item Next we show that $\gamma$ is $R$-balanced. If $\beta(y_z)=z$, then $\beta(ry_z) = rz$, and so $y_{rz} = ry_z$. Thus
            \[\gamma(d,rz) = \overline{d\ten y_{rz}}= \overline{d\ten r y_z}= \overline{dr\ten y_z}= \gamma(dr, z)\]
            For the second axiom, simply prove that 
            \[\gamma(d+d', z) = \overline{(d+d')\ten y_z} = \overline{d\ten y_z + d'\ten y_z} = \overline{d\ten y_z} + \overline{d'\ten y_z} =\gamma(d,z) + \gamma(d',z) \]
            For the third axiom, if $\beta(y_z)=z$ and $\beta(y_{z'}) = z'$, then $\beta(y_z + y_{z'}) = z+z'$, so $y_{z+z'} = y_z + y_{z'}$. Thus
            \[\gamma(d, z+z') = \overline{d\ten y_{z+z'}} = \overline{d\ten (y_z + y_{z'})} = \overline{d\ten y_{z}} + \overline{d\ten y_{z'}} = \gamma(d,z) + \gamma(d,z')\]
            This shows that $\gamma$ is $R$-balanced.
        \end{itemize}
        Therefore, by the Universal Property of Tensor Product, there exists $\pi' :D\ten_R Z \to (D\ten_R Y)/\im(1\ten \alpha)$ where $d\ten z \mapsto \overline{d\ten y_z}$.

        Define $\varphi: (D\ten_R Y)/\im(1\ten \alpha): D\ten_R Z$ by $(d\ten y)\mapsto d\ten \beta(y)$. We show that $\pi'\circ \varphi$ and $\varphi \circ \pi'$ are identity maps (on respective domain). 
        \begin{itemize}
            \item $(\pi'\circ \varphi)(\overline{d\ten y}) = \pi' (d\ten \beta(y)) = \overline{d\ten y}$
            \item $(\varphi\circ \pi')(d\ten z)=\varphi(\overline{d\ten y_z}) = d\ten \beta(y_z) = d\ten z$
        \end{itemize}
        This shows that $\varphi$ and $\pi$ are inverses of each other, implying that they are isomorphisms. This shows that $\im(1\ten \alpha) = \ker(1\ten\beta)$.
    \end{enumerate}

    For the second statement, the forward direction is proved, so supposed that (\ref{eqn: thm: flat mod exact}) is exact for all $(S,R)$-bimodule $D$. Take $D=R$. Recall $R\ten_R X \cong X$ and this holds similarly for $Y$ and $Z$. We then have the following diagram: 
    \[\begin{tikzcd} [sep = small]
	{R\ten_R X} && {R\ten_R Y} && {R\ten_R Z} && 0 \\
	\\
	X && Y && Z && 0
	\arrow["{1\ten \alpha}", from=1-1, to=1-3]
	\arrow[tail reversed, from=1-1, to=3-1]
	\arrow["{1\ten \beta}", from=1-3, to=1-5]
	\arrow[tail reversed, from=1-3, to=3-3]
	\arrow[from=1-5, to=1-7]
	\arrow[tail reversed, from=1-5, to=3-5]
	\arrow["\alpha", from=3-1, to=3-3]
	\arrow["\beta", from=3-3, to=3-5]
	\arrow[from=3-5, to=3-7]
    \end{tikzcd}\]
    where the isomorphism between $R\ten_R X$ and $X$ is given by $1\ten x \mapsto x$ (similarly for $Y$ and $Z$). We show that we have commutativity in the first and second square. First, let $1\ten x\in R\ten_R X$, then sending along the upper route to $Y$ we obtain 
    \[1\ten x \overset{1\ten \alpha}{\mapsto} 1\ten\alpha(x) \mapsto \alpha(x) \]
    If sending along the lower route to $Y$ we obtain 
    \[1\ten x \mapsto x \overset{\alpha}{\mapsto} \alpha(x)\]
    This shows we have commutativity in the first square. Similarly we have comutativity in the second square. This gives commutativity in the whole diagram. Since the first row is commutative and we have isomorphisms map, we conclude that the second row is commutative. This completes the proof.
\end{proof}

\begin{ex}
    $\Z \overset{\alpha}{\hookrightarrow} \Q$. Take $D= \Z/2Z$. Then $D\ten_\Z \Z\cong D = \Z/2\Z$. But $D\ten_\Z\Q\cong 0$, since
    \[x\ten \frac{r}{s} = x\ten \frac{2r}{2s} = 2x \ten \frac{r}{2s} = 0\]
    since $2x= 0$ in $D$.
\end{ex}

\medskip

\begin{pro} \label{pro: defn of flat mod}
    Let $D$ be a right $R$-module. TFAE:
    \begin{enumerate}
        \item If we have SES $0 \to X \to Y \to Z \to 0$, then we have $0 \to D\ten_R X \to D\ten_R Y \to D\ten_R Z \to 0$ is exact.
        \item if $0\to X\to Y$ is exact, then $0 \to D\ten_RX \to D\ten_R Y$ is exact. 
    \end{enumerate}
    In any of these cases, we say that $D$ is a flat $R$-module.
\end{pro}

\medskip

\begin{cor}
    Let $D$ be a right $R$-module. The functor $\mathcal F:D\ten_R -: R\text{-mod} \to \text{Ab}$ is right exact covariant. Moreover $\mathcal F$ is exact if and only if $D$ is flat. If $D$ is a $(S,R)$-bimodule, then $\mathcal F: D\ten_R -$ is a functor that sends from $R$-module to $S$-module.
\end{cor}

\medskip

\begin{thm}
    Projection (and hence free) modules are flat.
\end{thm}
\begin{proof}
    We first prove the special case for free modules. Let $F$ be a free $R$-module and $\alpha:X\to Y$ be an injective $R$-module homomorphism. To show that $F$ is flat, by Proposition \ref{pro: defn of flat mod} we show that if $0 \to X \xto{\alpha} Y$ is exact, then $0\to D\ten_R X \xto{1\ten\alpha} D\ten_R Y$ is exact. This is equivalent to showing that the injection $\alpha:X\to Y$ implies $1\ten \alpha:F\ten_R X\to F\ten_R Y$ is injective.

    First, since $F$ is free, so we can write $F\cong \bigoplus_{a\in A}R$ where $F$ is free on a subset $A\subseteq F$. Next, in tutorial, we have seen that $(\bigoplus R) \ten_R X\cong \bigoplus(R\ten R X)$. We have also seen previously that $R\ten_R X\cong X$. Altogether we obtain the following commutative diagram:
    \[\begin{tikzcd}[sep=small]
	& {F\ten_R X} &&& {F\ten_R Y} \\
	\\
	{\sum (\mathds 1_a \ten x_a)} & {\br{\bigoplus R}\ten_R X} &&& {\br{\bigoplus R} \ten_R Y} & {\sum (\mathds 1_a \ten \alpha(x_a))} \\
	\\
	{\sum(1\ten x_a)} & {\bigoplus(R\ten_R X)} &&& {\bigoplus(R\ten_R Y)} & {\sum(1\ten \alpha(x_a))} \\
	\\
	{\sum x_a} & {\bigoplus_{a\in A} X} &&& {\bigoplus_{a\in A} Y} & {\sum \alpha(x_a)}
	\arrow["{1\ten \alpha}", from=1-2, to=1-5]
	\arrow[tail reversed, from=1-2, to=3-2]
	\arrow[tail reversed, from=1-5, to=3-5]
	\arrow["\in"{description}, draw=none, from=3-1, to=3-2]
	\arrow[curve={height=-24pt}, maps to, from=3-1, to=3-6]
	\arrow[from=3-2, to=3-5]
	\arrow[tail reversed, from=3-2, to=5-2]
	\arrow[tail reversed, from=3-5, to=5-5]
	\arrow["\ni"{description}, draw=none, from=3-6, to=3-5]
	\arrow[maps to, from=3-6, to=5-6]
	\arrow[maps to, from=5-1, to=3-1]
	\arrow["\in"{description}, draw=none, from=5-1, to=5-2]
	\arrow[from=5-2, to=5-5]
	\arrow[tail reversed, from=5-2, to=7-2]
	\arrow[tail reversed, from=5-5, to=7-5]
	\arrow["\ni"{description}, draw=none, from=5-6, to=5-5]
	\arrow[maps to, from=5-6, to=7-6]
	\arrow[maps to, from=7-1, to=5-1]
	\arrow["\in"{description}, draw=none, from=7-1, to=7-2]
	\arrow["\varphi", from=7-2, to=7-5]
	\arrow["\ni"{description}, draw=none, from=7-6, to=7-5]
    \end{tikzcd}\]
    where $\mathds 1_a$ denotes the direct sum indexed by $A$ which is only takes value $1$ at the $a$-th position and $0$ otherwise. If $\sum x_a \in \ker \varphi$, then $\sum \alpha(x_a) = 0$. Since this is a direct sum, we must have $\alpha(x_a) = 0$ for all $a$. By assumption $\alpha$ is injective, so $x_a=0$ for all $a$. Therefore $\sum x_a = 0$. This shows that $\varphi$ is injective, and thus $1\ten \alpha$ is injective, showing that any free $R$-module is flat.

    Next, let $P$ be a projective $R$-module. Then $P \oplus P' = F$ for some free $R$-module $F$. Note $(P\oplus P')\ten_R X\cong (P\ten_R X) \oplus (P' \ten_R X)$. We have the following commutative diagram:
    \[\begin{tikzcd}[sep=small]
	{a\ten x} & {F\ten_R X} &&& {F\ten_R Y} & {a\ten \alpha(x)} \\
	\\
	{(a+0) \ten x} & {(P\oplus P') \ten_R X} \\
	\\
	{(a\ten x, 0)} & {(P\ten_R X) \oplus (P' \ten_R X)} &&& {(P\ten_R Y) \oplus (P' \ten_R Y)} & {(a \ten \alpha(x), 0)}
	\arrow["\in"{description}, draw=none, from=1-1, to=1-2]
	\arrow[curve={height=-24pt}, maps to, from=1-1, to=1-6]
	\arrow["{1\ten \alpha}", from=1-2, to=1-5]
	\arrow[tail reversed, from=1-2, to=3-2]
	\arrow[tail reversed, from=1-5, to=5-5]
	\arrow["\ni"{description}, draw=none, from=1-6, to=1-5]
	\arrow[from=1-6, to=5-6]
	\arrow[maps to, from=3-1, to=1-1]
	\arrow["\in"{description}, draw=none, from=3-1, to=3-2]
	\arrow[tail reversed, from=3-2, to=5-2]
	\arrow[maps to, from=5-1, to=3-1]
	\arrow["\in"{description}, draw=none, from=5-1, to=5-2]
	\arrow[from=5-2, to=5-5]
	\arrow["\ni"{description}, draw=none, from=5-6, to=5-5]
    \end{tikzcd}\]
    By the previous settled special case, since $F$ is free, so $1\ten \alpha$ is injective. By restricting $1\ten\alpha$ to $P\ten_R X$ (and thus is mapped to $P\ten_R Y$) it is also injective.
\end{proof}

\begin{ex}
    \hfill

    \begin{enumerate}
        \item $\Z/2\Z$ is not flat, since $\Z$ injects to $\Q$ yet $\Z/2\Z \ten_\Z \Z \cong \Z/2\Z$ and $\Z/2\Z \ten_\Z \Q \cong 0$, so no injective map is possible after tensoring.
        \item $\Q$ is not projective but it is flat. Let $\alpha:X\hookrightarrow Y$ be a $\Z$-module homomorphism. Consider $1\ten\alpha:\Q\ten_\Z X \to \Q\ten_\Z Y$. Note for an element in $\Q\ten_\Z X$ takes the form and can be rewriten into
        \begin{align*}
            \frac{r_1}{s_1} \ten x_1 + \dots + \frac{r_m}{s_m} \ten x_m 
            &= \frac{r_1'}{s} \ten x_1 \pd \frac{r_m'}{s} \ten x_m \\
            &= \frac{1}{s} \ten r'_1x_1 \pd \frac{1}{s}\ten r'_mx_m \\
            &= \frac{1}{s} \ten x
        \end{align*}
        where $s= \operatorname{lcm}(s_1 \many s_m)$. So $1\ten\alpha$ is injective, and $\Q$ is flat.
        \item We have seen that $\Q/\Z$ is injective. We claim that it is not flat. Let $\varphi:\Z\hookrightarrow\Z$ be defined $n\mapsto 2n$. Recall that $\Q/\Z \ten_\Z \Z \cong \Q/\Z$. Then note
        \[(1\ten\varphi)\br{\overline{\frac{1}{2}}\ten 1} = \overline{\frac{1}{2}} \ten \varphi(1) = \overline{\frac{1}{2}} \ten 2 = \overline{\frac{1}{2}}\cdot 2 \ten 1 = \overline{1} \ten 1= 0\]
        since $\overline{1} \in \Z$, which is quotiened out in $\Q/\Z$.
    \end{enumerate}
\end{ex}

\medskip

\begin{defn} [Functor adjunction]
    Let $\mathcal C$ and $\mathcal D$ be categories. Let $X$ and $Z$ be an object of $\mathcal C$ and $\mathcal D$ respectively. Let $\mathcal L:\mathcal C\to \mathcal D$ and $\mathcal R: \mathcal D\to \mathcal C$ be two functors.  We say that $(\mathcal L, \mathcal R)$ is a pair of adjoint functor if $\mor_\mathcal D(\mathcal L(X),Z)\cong \mor_\mathcal C (X,\mathcal R(Z))$.
\end{defn}

\medskip

\begin{thm} [Tensor-Hom Adjunction] \label{thm: ten-hom adjunct}
    Let $X$ be a right $R$-module, $Y$ be a $(R,S)$-bimodule, and $Z$ be a right $S$-module. Then $(-\ten_R Y,\ \h_S(Y,-))$ is a pair of adjunct functors. In other words, we have an isomorphism of abelian group:
    \[\h_S(X\ten_R Y, Z)\cong \h_R(X, \h_S(Y,Z))\]
\end{thm}
\begin{proof}
    Define $f:\h_S(X\ten_R Y, Z) \to \h_R(X, \h_S(Y,Z))$ where 
    \[f: \varphi \mapsto \tilde\varphi:X\to \h_S(Y,Z)\]
    and $\tilde\varphi$ is defined by $x\mapsto \tilde\varphi_x(y) := \varphi(x\ten y)$. Our goal is to construct a map $g$ and show that $f\circ g$ and $g\circ f$ are identity map on their respective domain, i.e. they are isomorphisms.

    We first need to show that $f$ is well-defined, which will be broken down into several item. First, we show that $f(\varphi)$ is really an $R$-module homomorphism For any $r\in R$, $x\in X$, and $y\in Y$, we have
    \begin{align*}
            ((f(\varphi))(xr))(y) 
            &= (\tilde\varphi(xr))(y) \\
            &= \tilde\varphi_{xr}(y) \\
            &= \varphi(xr\ten y) \\
            &= \varphi(x\ten ry) \\
            &= \tilde\varphi_x(ry) \\
            &= (\tilde\varphi_x\cdot r)(y) \\
            &= ((f(\varphi))(x)\cdot r)(y)
        \end{align*}
    This shows $f(\varphi)$ respects $R$-action. Next show that $(f(\varphi))(x) = \tilde\varphi_x$ is really a $S$-module homomorphism. For any $s\in S$, we have
    \begin{align*}
            ((f(\varphi))(x))(ys) 
            &= \tilde\varphi_x(ys) \\
            &= \varphi(x\ten ys)\\
            &= \varphi(x\ten y) s \\
            &= \tilde\varphi_x(y) s\\
            &= ((f(\varphi))(x))(y) s
        \end{align*}
    This shows that $(f(\varphi))(x) = \tilde\varphi_x$ respect $S$-action.

    Suppose given $\psi\in \h_R(X, \h_S(Y,Z))$. We define $\beta: X\times Y \to Z$ by 
    \[\beta: (x,y)\mapsto (\psi(x))(y)\]
    We claim that $\beta$ is $R$-balanced, where one just have to verify all the axioms for $R$-balanced:
    \[\beta(xr,y) = (\psi(xr))(y) = (\psi(x)\cdot r)(y) = (\psi(x))(ry) = \beta(x,ry)\]
    and
    \[\beta(x+x',y) = (\psi(x+x'))(y) = (\psi(x) + \psi(x'))(y) = (\psi(x))(y) + (\psi(x'))(y) = \beta(x,y) + \beta(x',y)\]
    and 
    \[\beta(x,y+y') = (\psi(x))(y+y') = (\psi(x))(y) + (\psi(x))(y') = \beta(x,y) + \beta(x,y')\]
    Since $\beta:X\times Y\to Z$ is an $R$-balanced map, by the universal property of tensor product, there exists $R$-homomorphism $\psi':X\ten_R Y\to Z$ such that $x\ten y \mapsto (\psi(x))(y)$. Thus, for all $\psi\in \h_R(X, \h_S(Y,Z))$ we are able to get a respective $\psi'$.
    
    We then define $g:\h_R(X, \h_S(Y,Z))\to\h_S(X\ten_R Y, Z)$ where 
    \[g: \psi \mapsto \psi' :X\ten_R Y \to Z\]
    where $\psi':x\ten y \mapsto \psi'(x\ten y) = (\psi(x))(y)$. We claim that $f$ and $g$ are invserses of each other. To see this, we first show $(f\circ g)(\psi) = \psi$. For the sake of readability, we write $(f\circ g)(\psi) = f(g(\psi))= f(\psi') = \tilde{\psi'}$ 
    \[(((f\circ g)(\psi))(x))(y) = (\tilde{\psi'}(x))(y) = \psi'_x(y) = \psi' (x\ten y) = (\psi(x))(y)\]
    Next, we have to show $(g\circ f)(\varphi) = \varphi$.
    \[((g\circ f)(\varphi))(x\ten y) = (f(\varphi))'(x\ten y) = ((f(\varphi))(x))(y) = \tilde\varphi_x(y) = \varphi(x\ten y)\] 
    Thus we have shown that $f$ and $g$ are isomorphisms. This completes the proof.
\end{proof}

\begin{re}
    The intuition of the defined map $f$ is as follow: for $f$, a homomorphism $\varphi$ that initially maps $x\ten y$ to $\varphi(x\ten y)$, is separated into stages: first an element $x$ of $X$ determines the image map, then it takes all $y$ to $\varphi(x\ten y)$.
    \[f:\varphi\mapsto (x\mapsto (y\mapsto \varphi(x\ten y)))\]
    For $g$, a homomorphism $\psi$ works as follow: given an $x\in X$, it defines another homomorphism $\psi(x)$, and this homomorphism sends $y$ to $(\psi(x))(y)$. This is exactly the image of $x\ten y$ mapped by $g$.
    \[\psi \mapsto (x\ten y \mapsto (\psi(x))(y))\]
\end{re}

\medskip

\begin{cor}
    Let $R$ be a commutative ring. Then the tensor product of two projective $R$-module is projective.
\end{cor}
\begin{proof}
    Let $R$ be a commutative ring. Let $P$ and $P'$ be projective $R$-modules. By definition of projective modules, suppose we have $X$ and $Y$ are $R$-modules, let $\beta:X\twoheadrightarrow Y$ and $h:P\ten_R P' \to Y$ be $R$-module homomorphisms where $\beta$ is a surjective $R$-module homomorphism:
    \[\begin{tikzcd}[sep=small]
	&& {P\ten_R P'} \\
	\\
	X && Y
	\arrow["{?}"{description}, dashed, from=1-3, to=3-1]
	\arrow["h", from=1-3, to=3-3]
	\arrow["\beta"', two heads, from=3-1, to=3-3]
    \end{tikzcd}\]
    We want to construct a map from $P\ten_R P' \to X$. First, note that $\beta$ induces the following surjective map
    \[\beta_* : \h_R(P', X) \twoheadrightarrow \h_R(P',Y)\]
    where $\alpha\mapsto \beta\circ \alpha$. This further induces a surjective map
    \[(\beta_*)_*:\h_R(P,\h_R(P',X))\twoheadrightarrow \h_R(P,\h_R(P',Y))\]
    With this, due to tensor-hom adjunction, we have 
    \[\begin{tikzcd}[sep=small]
	& {f(\gamma)} && {(\beta_*)\circ (f(\gamma))} \\
	& {\h_R(P,\h_R(P',X))} && {\h_R(P,\h_R(P',Y))} \\
	\\
	\gamma & {\h_R(P\ten_R P', X)} && {\h_R(P\ten_R, P', Y)} & h
	\arrow[maps to, from=1-2, to=1-4]
	\arrow["\in"{marking, allow upside down}, draw=none, from=1-2, to=2-2]
	\arrow["\in"{marking, allow upside down}, draw=none, from=1-4, to=2-4]
	\arrow[curve={height=-24pt}, maps to, from=1-4, to=4-5]
	\arrow["{(\beta_*)_*}", two heads, from=2-2, to=2-4]
	\arrow["g", from=2-4, to=4-4]
	\arrow[curve={height=-24pt}, maps to, from=4-1, to=1-2]
	\arrow["\in"{description}, draw=none, from=4-1, to=4-2]
	\arrow[curve={height=30pt}, maps to, from=4-1, to=4-5]
	\arrow["f", from=4-2, to=2-2]
	\arrow["\ni"{description}, draw=none, from=4-5, to=4-4]
    \end{tikzcd}\]
    where $f$ and $g$ are as defined in Theorem \ref{thm: ten-hom adjunct}. Here the element $\gamma$ is explicit defined according to the following argument:
    \begin{itemize}
        \item Due to tensor-hom adjunction, there must be an isomorphic copy of $h$ in $\h_R(P,\h_R(P',Y))$.
        \item Since $(\beta_*)_*$ is surjective, there must be a pre-image of the isomorphic copy of $h$,
        \item Again, due to tensor-hom adjunction, there must be an isomorphic copy of the pre-image of isomorphic copy of $h$. We define it to be $\gamma$.
    \end{itemize}
    We claim that $\beta\circ \gamma = h$. We want to show that for any $s\in P$ and $t\in P'$, we must have $(\beta\circ\gamma)(s\ten t) = h(s\ten t)$. Note that 
    \begin{align*}
        h(s\ten t)
        &= (g(\beta_* \circ f(\gamma)))(s\ten t) \\
        &= (g(\beta_* \circ \tilde\gamma))(s\ten t)\\
        &= (\beta_* \circ \tilde\gamma)'(s\ten t)\\
        &= ((\beta_* \circ \tilde\gamma)(s)) (t) \\
        &= (\beta_* (\tilde\gamma_s))(t) \\
        &= (\beta\circ \tilde\gamma_s)(t) \\
        &= \beta(\tilde\gamma_s(t)) \\
        &= \beta(\gamma(s\ten t)) \\
        &= (\beta\circ \gamma)(s\ten t)
    \end{align*}
    This shows that $h= \beta\circ \gamma$, and thus we have shown that $P\ten_R P$ is projective.
\end{proof}

\newpage
\section{Chain Complex}
\subsection{Chain Complex and Homology}

\begin{defn}[Chain complex and cochain complex]
    A chain complex is a sequence
    \[\mathcal C := (X_{\bullet}, d_\bullet):\dots \to X_1 \xto {d_1} X_0 \xto{d_0} X_{-1} \xto{d_{-1}} X_{-2} \to \dots\]
    where $X_i$ are $R$-modules and $d_i$ are $R$-module homomorphism, such that $d_{n-1}\circ d_n = 0$. In other words $\im d_n \subseteq \ker d_{n-1}$.

    Similarly, a cochain complex is a sequence 
    \[\mathcal D:=(X^{\bullet}, d^\bullet):\dots \ot X^1 \xot {d^1} X^0 \xot{d^0} X^{-1} \xot{d^{-1}} X^{-2} \ot \dots\]
    where $X^i$ are $R$-modules and $d^i$ are $R$-module homomorphism, such that $d^{n}\circ d^{n-1} = 0$. In other words $\im d^{n-1} \subseteq \ker d^{n}$.
\end{defn}

\medskip

\begin{defn} [Boundedness]
    A chain complex is said to be bounded above if there exists $N\in \Z$ such that $X_m=0$ for all $m<N$. Similarly, a cochain complex is said to be bounded below if there exists $M\in \Z$ such that $X^m=0$ for all $m<M$
\end{defn}

\medskip

\begin{defn} [Homology]
    Suppose given a chain complex $\mathcal C:=(X_\bullet, d_\bullet)$. The $n$-th homology is defined to be the quotient group
    \[H_n(\mathcal C):= \frac{\ker d_n}{\im d_{n+1}}\]
\end{defn}

\begin{defn} [Cohomology]
    Suppose given a cochain complex $\mathcal D:=(X^\bullet, d^\bullet)$. The $n$-th cohomology is defined to be the quotient group
    \[H^n(\mathcal D):= \frac{\ker d^{n+1}}{\im d^{n}}\]
\end{defn}

\begin{re} [(Co)homology and exactness]
    Homology and cohomology detects failure of exactness. In particular, the $n$-th homology (cohomology) is $0$ if and only if the $n$-th position of the chain (cochain) complex is exact. A (co)chain complex is said to be exact if it is exact at every term. It should be clear that, thus, a (co)chain complex is exact if and only if the (co)homology is always exact.
\end{re}

\medskip

\begin{defn} [Homomorphism of (cochain) complexes]
    Let $(X^\bullet, d^\bullet)$ and $(Y^\bullet, \delta^\bullet)$ be two (cochain) complexes. The homomorphism from $X^\bullet$ to $Y^\bullet$ is a collection of module homomorphism $\varphi:X^\bullet \to Y^\bullet$ where for all $n$ we have $\varphi_n:X^n \to Y^n$ such that the following diagram commutes:
    \[\begin{tikzcd} [sep = small]
	\dots && {X^{n-1}} && {X^{n}} && \dots \\
	\\
	\dots && {Y^{n-1}} && {Y^n} && \dots
	\arrow["{d^{n-1}}", from=1-1, to=1-3]
	\arrow["{d^{n}}", from=1-3, to=1-5]
	\arrow["{\varphi_{n-1}}", from=1-3, to=3-3]
	\arrow["{d^{n+1}}", from=1-5, to=1-7]
	\arrow["{\varphi_{n}}", from=1-5, to=3-5]
	\arrow["{\delta^{n-1}}", from=3-1, to=3-3]
	\arrow["{\delta^{n}}", from=3-3, to=3-5]
	\arrow["{\delta^{n+1}}", from=3-5, to=3-7]
    \end{tikzcd}\]
\end{defn}

\medskip

\begin{pro}
    Let $\varphi:X^\bullet \to Y^\bullet$ be a homomorphism of (cochain) complex. Then it induces an $R$-module homomorphism $\alpha^*:H^n(X) \to H^n(Y)$, one for each $n$, given by $\overline x \mapsto \overline {\varphi(x)}$.
\end{pro}

\begin{proof}
    Suppose as stated in the statement. The argument splits into two parts: showing that the claimed map is well-defined, and showing that it is indeed an $R$-module homomorphism. Note by definition of complexes homomorphism have the following commutative diagram:
    \[\begin{tikzcd} [sep = small]
	\dots && {X^{n}} && {X^{n+1}} && \dots \\
	\\
	\dots && {Y^{n}} && {Y^{n+1}} && \dots
	\arrow["", from=1-1, to=1-3]
	\arrow["{d^{n+1}}", from=1-3, to=1-5]
	\arrow["{\varphi_{n}}", from=1-3, to=3-3]
	\arrow["", from=1-5, to=1-7]
	\arrow["{\varphi_{n+1}}", from=1-5, to=3-5]
	\arrow["", from=3-1, to=3-3]
	\arrow["{\delta^{n+1}}", from=3-3, to=3-5]
	\arrow["", from=3-5, to=3-7]
    \end{tikzcd}\]
    let $x\in \ker d_{n+1}$. Since the diagram is commutative, we must have
    \[\delta_{n+1}(\varphi_n(x)) = \varphi_{n+1}(d_{n+1}(x))=\varphi_{n+1}(0) = 0\]
    So $\varphi_n(x) \in \ker \delta_{n+1}$. Thus an element $\overline{x}\in H^n(X)$ means that $x\in \ker d_n$, implying that $\varphi_n(x) \in \ker \delta_{n+1}$, and thus $\overline{\varphi_n(x)}\in H^{n+1}(X)$. This shows that the claimed map is valid.

    To show that the map is well-defined, consider elements $\overline{x}$ and $\overline{x'}$ from $H^n(X)$. This means that $x,x'\in \ker\im d_n$, and so $x-x'\in \ker d_{n+1}$. It then follows from the definition of complexes that we have $x-x'\in \im d_n$, so let $x''\in X^{n-1}$ such that $d_n(x'') = x-x'$. Observe that
    \[\varphi_n(x)-\varphi_n(x') = \varphi_n(x-x') = \varphi_n(d_n(x'')) = \delta_{n}(\varphi_{n-1}(x''))\]
    Thus $\varphi(x)-\varphi(x')\in \im \delta_n$, i.e. $\overline{(\varphi(x))} = \overline{(\varphi(x'))}$.

    Lastly, to show that the map $\alpha^*$ is $R$-module homomorphism:
    \begin{align*}
        \alpha^n(\overline{x} + \overline{x'})
        &= \alpha^n(\overline{x+x'})\\
        &= \overline{\varphi(x+x')} \\
        &= \overline{\varphi(x)+\varphi(x')} \\
        &= \overline{\varphi(x)}+\overline{\varphi(x')} \\
        &= \alpha^n(\overline{x}) + \alpha^n(\overline{x'})
    \end{align*}
    and
    \[\alpha^n(r\overline{x}) = \alpha^n(\overline{rx}) = \overline{\varphi(rx)} = \overline{r\varphi(x)} = r\ \overline{\varphi(x)}\]
    This completes the proof.
\end{proof}

\begin{thm} [Long Exact Sequence in Cohomology]
    Let $(X^\bullet, d_X^\bullet), (Y^\bullet, d_Y^\bullet), (Z^\bullet, d_Z^\bullet)$ be cochain complexes. Let $0\to X^\bullet \xto\alpha Y^\bullet \xto\beta Z^\bullet \to 0$ be a SES of cochain complex bounded below by $0$ (i.e. $X^{-n} = 0$ for all $n> 0$), that is, to say that for every $n$ we have
    \[0 \to X^n \xto{\alpha_n} Y^n \xto{\beta_n} Z^n \to 0\]
    Then, we have a long exact sequence (LES) given by
    \[0 \to H^0(X) \xto{\alpha^*_0} H^0(Y) \xto{\beta^*_0} H^0(Z) \xto{\delta_0} H^1(X) \xto{\alpha^*_1} H^1(Y) \xto{\beta^*_1} H^1(Z) \xto{\delta_1} \cdots\]
    where for each $n$ 
    \begin{itemize}
        \item $\alpha^*_n$ sends $\overline {x}$ to $\overline{\alpha_{n}(x)}$
        \item $\beta^*_n$ sends $\overline{y}$ to $\overline{\beta_{n}(y)}$
        \item $\delta_n:H^n(Z)\to H^{n+1}(X)$ where $\overline z \mapsto \delta_n(z)$ is defined as follow
        \begin{enumerate}
            \item Let $y\in Y^n$ such that $\beta_n(y)=z$.
            \item Let $x\in X^{n+1}$ such that $\alpha_{n+1}(x) = d_Y^{n+1}(y)$.
            \item Let $\overline x \in H^{n+1}(X)$ be represented by $x$.
            \item We thus define $\delta_n(z)$ to be $\overline x$.
        \end{enumerate}
    \end{itemize}
    Here, each $\delta_n$ is called the connecting homomorphism.

    Furthermore, if any two of the complexes are exact, then the third is exact.
\end{thm}
\begin{proof}
    The well-definedness of connecting homomorphisms is left as a tutorial problem, thus is omitted here.

    We first check that exactness occurs at
    \[H^n(X)\xto{\alpha_n^*} H^n(Y)\xto{\beta_n^*} H^{n}(Z)\]
    that is, we show that $\im \alpha_n^* = \ker \beta_n^*$. First, to show $\im \alpha_n^* \subseteq \ker \beta_n^*$, let $\overline x\in H^n(X)$. By assumption $\beta_n\circ \alpha_n$ is zero map due to exactness. Thus 
    \[\beta_n^*(\alpha_n^*(\overline{x})) = \beta_n^*(\overline{\alpha_n(x)}) = \overline{\beta_n(\alpha_n(x))} = \overline{0}\]
    Thus $\im \alpha_n^*\subseteq \ker \beta_n^*$. Next, to show that $\ker \beta_n^* \subseteq \im \alpha_n^*$, let $\overline y\in H^n(Y)$ such that $\beta_n^*(\overline y) = \overline{\beta_n(y)}= \overline{0}\in H^n(Z)$, thus $\beta_n(y)\in \im d_Z^n$, so let $z\in Z^{n-1}$ such that $d_Z^n(z)=\beta_n(y)$. Note $\beta_{n-1}$ is surjective, so let $y'\in Y^{n-1}$ such that $\beta_{n-1}(y')=z$. Altogether we have
    \[\beta_n(y)=d_Z^n(z)=d_Z^n(\beta_{n-1}(y')) = \beta_n(d_Y^n(y'))\]
    This implies $\beta_n(y-d_Y^n(y'))=0$, so $y-d_Y^n(y')\in \ker \beta_n = \im \alpha_n$. Let $x\in X^n$ such that $\alpha_n(x) = y-d_Y^n(y')$, and thus
    \[d_Y^{n+1}(\alpha_n(x)) = d_Y^{n+1}(y-d_Y^n(y')) = d_Y^{n+1}(y) - d_Y^{n+1}(d_Y^n(y')) = d_Y^{n+1}(y)+0\]
    Note that by commutativity of the diagram, LHS can be written as $\alpha_{n+1}(d_X^{n+1}(x))$. Also, for RHS, recall that $\overline y \in H^n(Y)$, where by definition 
    \[H^n(Y) = \frac{\ker d_Y^{n+1}}{\im d_Y^n}\]
    and thus $y\in \ker d_Y^{n+1}$, which implies that $d_Y^{n+1}(y)=0$. Altogether we have $\alpha_{n+1}(d_X^{n+1}(x)) = 0$. By assumption on exactness we see $\alpha_{n+1}$ is exact, so $d_X^{n+1}(x) = 0$, i.e. $x\in \ker d_X^{n+1}$. Again, by definition of cohomology, we see that $\overline{x}\in H^n(X)$. We claim that $\overline{x}$ is the pre-image of $\overline{y}$ under $\alpha_n^*$:
    \[\alpha_n^*(\overline{x}) = \overline{\alpha_n(x)} = \overline{y - d_Y^n(y')}\in H^n(Y) = \frac{\ker d_Y^{n+1}}{\im d_Y^n}\]
    By definition of cohomology $H^n(Y)$ we see that $\overline{y-d_Y^n(y')} = \overline y$ since the image of $d_Y^n$ is quotiented away in $H^n(Y)$. This shows that $\alpha_n^*(\overline{x}) = \overline{y}$, thus $y\in \im \alpha_n^*$. 

    We now check exactness occurs at
    \[H^n(Y)\xto{\beta_n^*} H^n(Z)\xto{\delta_n} H^{n+1}(X)\]
    First, to show $\im \beta_n^*\subseteq \ker \delta_n$, let $\overline{y}\in H^n(Y)$. By definition $\beta_n^*(\overline y) = \overline{\beta_n(y)}$. For convenience let $z = \beta(y)$, so $\beta_n^*(\overline y) = \overline{z}$, and we want to show $\delta_n(\overline{z}) = 0$. By definition of $\delta_n$, if $\alpha_n(x) = d_Y^n(y)$, then $\delta_n(\overline(z)) = \overline x$. Note that by our assumption $\overline y\in H^n(Y)$ implies that $y\in \ker d_Y^{n}$, so $\alpha_n(x) = d_Y^{n}(y)=0$. But provided the SES, we note $\ker \alpha_n = 0$, so $\alpha_n$ is injective, and thus $x=0$. This implies that 
    \[\delta_n(\beta_n^*(\overline y)) = \delta_n(\overline{\beta(y)}) = \delta_n(\overline z) = \overline x = \overline 0\]
    Thus $\im \beta_n^* \subseteq \ker \delta_n$. Next to show that $\ker \delta_n\subseteq \im \beta_n^*$, let $\overline z\in H^n(Z)$ such that $\delta_n(\overline z) = \overline{x} = 0 \in H^{n+1}(X)$, i.e. $x\in \im d_X^{n+1}$, where by definition of the connecting homomorphisms we have some $y$ such that $\beta_n(y)=z$ and $\alpha_{n+1}(x) = d_{Y}^{n+1}(y)$. Let $x = d_X^{n+1}(x')$. Then 
    \[d_Y^{n+1}(y) = \alpha_{n+1}(x) = \alpha_{n+1}(d_X^{n+1}(x')) \overset{(*)}{=} d_Y^n ((\alpha_n)(x'))\]
    where $(*)$ is due to the commutativity of the diagram. Together, the above implies that $y-\alpha_n(x')\in \ker d_Y^{n+1}$, and we claim that this is the pre-image of $\overline{z}$ under $\beta_n^*$:
    \[\beta_n^*(\overline{y-\alpha_n(x')}) = \overline{\beta_n(y-\alpha_n(x'))} = \overline{\beta_n(y)} - \overline{\beta_n(\alpha_n(x'))} = \overline {z} - 0 = \overline{z}\]
    where note $\beta_n\circ \alpha_n$ is the zero map due to exactness in the assumption.
    
    For the second statement, recall from some previous remark that exactness of a complex is equivalent to that the cohomology is trivial. According to the first statement we have obtained the long exact sequence
    \[0 \to H^0(X) \xto{\alpha^*_0} H^0(Y) \xto{\beta^*_0} H^0(Z) \xto{\delta_0} H^1(X) \xto{\alpha^*_1} H^1(Y) \xto{\beta^*_1} H^1(Z) \xto{\delta_1} \cdots\]
    Case 1: if $X^\bullet$ and $Y^\bullet$ are trivial, we have 
    \[0 \to 0 \xto{\alpha^*_0} 0 \xto{\beta^*_0} H^0(Z) \xto{\delta_0} 0 \xto{\alpha^*_1} 0 \xto{\beta^*_1} H^1(Z) \xto{\delta_1} \cdots\]
    This forces $\alpha_n^*$, $\beta_n^*$, and $\delta_n$ to be zero maps. Specifically, note that $\ker \delta_n = H^n (Z)$. Due to exactness, we see that $0 = \im \beta_n^* = \ker \delta_n = H^n (Z)$, and thus $Z^\bullet$ must be exact.

    Case 2: If $Y^\bullet$ and $Z^\bullet$ are trivial, we have
    \[0 \to H^0(X) \xto{\alpha^*_0} 0 \xto{\beta^*_0} 0 \xto{\delta_0} H^1(X) \xto{\alpha^*_1} 0 \xto{\beta^*_1} 0 \xto{\delta_1} \cdots\]
    This forces $\alpha_n^*$, $\beta_n^*$, and $\delta_n$ to be zero maps. Specifically, note that $\ker \alpha_n^* = H^n (X)$. Due to exactness, we see that $0 = \im \delta_n^* = \ker \delta_{n+1} = H^{n+1} (X)$. Similarly $H^0(X)$ it is also $0$. This shows that $X^\bullet$ must be exact.

    Case 3: If $X^\bullet$ and $Z^\bullet$ are trivial, we have
    \[0 \to 0 \xto{\alpha^*_0} H^0(Y) \xto{\beta^*_0} 0 \xto{\delta_0} 0 \xto{\alpha^*_1} H^1(Y) \xto{\beta^*_1} 0 \xto{\delta_1} \cdots\]
    This forces $\alpha_n^*$, $\beta_n^*$, and $\delta_n$ to be zero maps. Specifically, note that $\ker \beta_n^* = H^n (Y)$. Due to exactness, we see that $0 = \im \alpha_n^* = \ker \beta_{n}^* = H^{n} (Y)$. This shows that $Y^\bullet$ must be exact.

    Thus the proof is completed.
\end{proof}


\newpage
\subsection{Ext Group}

\begin{defn} [Projective resolution]
    Let $V$ be an $R$-module. A projective resolution of $V$ is an exact complex 
    \[\cdots \to P_2 \xto{d_2} P_1 \xto{d_1} P_0 \xto{\ep} V \to 0\to 0 \to \dots\]
    such that each $P_i$ is projective $R$-module. In shorthand notation we write $P_\bullet \twoheadrightarrow V$ to denote a free resolution of $V$.
\end{defn}
 
\medskip

\begin{re}
    Similarly we can define a free resolution, which is omitted here.
\end{re}

\medskip

\begin{pro}
    Every $R$-module has a projective resolution.
\end{pro}
\begin{proof}
    Let $V$ be an $R$-module. By previous result there exists projective $R$-module $P_0$ such that $P_0 \overset{\ep}{\twoheadrightarrow} V$. Consider $\ker \ep$, and there exists a projective $R$-module $P_1$ such that $P_1$ surjects to $\ker \ep$ via $d_1$. Suppose we have, inductively, that 
    \[P_n \xto{d_n} \dots \xto{d_1} P_0 \xto{\ep} V\]
    Let $\ep$ be $d_0$, and by our conrstruction we observe that $P_n$ surjects to $\ker d_{n-1}$ via $d_n$, i.e.
    \[P_n \twoheadrightarrow \ker d_{n-1}\]
    and thus this shows that $\im d_{n} = \ker d_{n-1}$. This completes the proof.
\end{proof}

With slight modification, a similar statement on free resolution can be proven:

\medskip

\begin{pro}
    Every $R$-module has a free resolution.
\end{pro}

\medskip

\begin{re}
    If $V$ is a projective $R$-module, then we have a projective resolution
    \[\dots \to 0 \to V \xto{\id} V \to 0 \to 0 \dots\]
    Also, projective resolution is not unique, where the following
    \[0 \to V \xto{\alpha} V\oplus V \xto{\beta} V \to 0 \to \dots\]
    where $\alpha:v\mapsto (v,0$ and $\beta:(v,w)\mapsto w$, is also a projective module of $V$
\end{re}

\medskip

\begin{defn}[Ext group]
    Let $P_\bullet \twoheadrightarrow V$ be a projective resolution of $V$ and $W$ be an $R$-module. We get a complex (of abelian group)
    \[\mathcal C:=0 \to \h_R(P_0, W)\xto{d_1^*} \h_(P_1, W) \xto{d_2^*} \h_R(P_2, W)\xto{d^*_3}\dots\]
    where $V$ is forgetted. It is indeed a complex since
    \[d_{n+1}^*\circ d_n^* = (d_n \circ d_{n+1})^* = 0\]
    Note that this complex is usually not exact.

    The $n$-th cohomology group derived from the left exact contravariant functor $\h_R(-, W)$ is 
    \[\ext_R^n(V,W):= H^n(\mathcal C)=\frac{\ker d_{n+1}^*}{\im d_n^*}\]
    Clearly $\ext_R^0(V,W) = \ker d_1 ^*$.
\end{defn}

\medskip

\begin{pro}
    Let $V$ and $W$ be $R$-modules. Then
    \[\ext^0_R(V,W) \cong \h_R(V,W)\]
\end{pro}
\begin{proof}
    We extract the following exact sequence from the projective resolution $P_\bullet \twoheadrightarrow V$:
    \[P_1 \xto{d_1} P_0 \xto{\ep} V \to 0\]
    Recall that $\h_R(-,W)$ is a left contravariant functor, so we have the exact sequence
    \[0 \xto{0} \h_R(V,W) \xto{\ep^*} \h_R(P_0, W) \xto{d_1^*} \h_R(P_1,W)\]
    By 1st isomorphism theorem on $\ep^*$ we get 
    \[\frac{\h_R(V,W)}{\ker \ep^*}\cong \im\ep^* = \ker d_1^*\]
    Note that $\ker \ep^* = \im 0 = 0$ due to exactness. On the other hand, by exactness we have $\im \ep^* = \ker d_1^*$. By definition of Ext we have that $\ext^0_R(V,W) = \ker d_1^*$. Altogether, we see
    \[\h_R(V,W) \cong \ext_R^0(V,W)\]
    This completes the proof.
\end{proof}
\begin{ex}
    \hfill

    \begin{enumerate}
        \item We compute $\ext^n_\Z(\Z/m\Z, D)$ for any abelian group $D$, where $m\geq 2$. From previous proposition we know 
        \[\ext_\Z^0(\Z/m\Z, D)\cong \h_\Z(\Z/m\Z, D)\]
        Let $\varphi\in \h_\Z(\Z/m\Z, D)$. By definition 
        \[m(\varphi(\overline{1})) = \varphi(\overline{m}) = 0\]
        Thus we know 
        \[\ext_\Z^0(\Z/m\Z, D)\cong \h_\Z(\Z/m\Z, D) \cong \sbr{d\in D:m\cdot d = 0}\]
        where we will denote it as $_mD$. To investigate the general case, we have to come up with the projective resolution:
        \[\dots \to 0\to \Z\xto{\times m}\Z \xto{\bmod m} \Z/m\Z \to 0\]
        This is indeed a free resolution of $\Z/m\Z$ (one can verify the exactness easily). Thus taking $\hom$ we get
        \[0\to \h_\Z(\Z,D) \xto{(\times m)^*}\h_\Z(\Z,D)\to 0 \to \dots\]
        And it is clear that $\ext^2_\Z(\Z/m\Z, D)=0$ for all $n\geq 2$. We compute the $\ext^1$ as follow: note $\h_\Z(\Z, D)\cong D$ with isomorphism given by $\varphi\mapsto \varphi(1)$. Thus:
        \[\begin{tikzcd} [sep = small]
	&& \varphi && {m\varphi} \\
	0 && {\h_\Z(\Z, D)} && {\h_\Z(\Z, D)} && 0 && \dots \\
	\\
	0 && D && D && 0 && \dots \\
	&& {\varphi(1)} && {m\varphi(1) = \varphi(m)}
	\arrow[maps to, from=1-3, to=1-5]
	\arrow["\in"{marking, allow upside down}, draw=none, from=1-3, to=2-3]
	\arrow[curve={height=18pt}, maps to, from=1-3, to=5-3]
	\arrow["\in"{marking, allow upside down}, draw=none, from=1-5, to=2-5]
	\arrow[curve={height=-30pt}, maps to, from=1-5, to=5-5]
	\arrow[from=2-1, to=2-3]
	\arrow["{(\times m)^*}", from=2-3, to=2-5]
	\arrow[from=2-3, to=4-3]
	\arrow[from=2-5, to=2-7]
	\arrow[from=2-5, to=4-5]
	\arrow[from=2-7, to=2-9]
	\arrow[from=4-1, to=4-3]
	\arrow[from=4-3, to=4-5]
	\arrow[from=4-5, to=4-7]
	\arrow[from=4-7, to=4-9]
	\arrow["\in"{marking, allow upside down}, draw=none, from=5-3, to=4-3]
	\arrow[from=5-3, to=5-5]
	\arrow["\in"{marking, allow upside down}, draw=none, from=5-5, to=4-5]
    \end{tikzcd}\]
        and $\ext_\Z^1(\Z/m\Z, D)\cong D/mD$.
    \item \todo{explain on how to construct}
    \end{enumerate}
\end{ex}

\medskip

\begin{re}
    The above example is not rigorous enough, in the sense that, we cannot assume that we still get the same result if starting from another projective resolution. In fact, we have that the result obtained is independent of the projective resolution.
\end{re}

\medskip

\begin{pro} [Comparison Theorem]
    Let $f:V\to V'$ be an $R$-module homomorphism and $P_\bullet\twoheadrightarrow V$ be a projective resolution of $V$ and $P'_\bullet \twoheadrightarrow V'$ be an exact complex, where it need not to be a projective resolution of $V'$. Then there exists $f_n:P_n\to P_n'$ such that the following commute:
    \[\begin{tikzcd}[sep=small]
	{\dots } && {P_3} && {P_2} && {P_1} && {P_0} && V && 0 \\
	\\
	\dots && {P_3'} && {P'_2} && {P'_1} && {P'_0} && {V'} && 0
	\arrow[from=1-1, to=1-3]
	\arrow["{d_3}", from=1-3, to=1-5]
	\arrow["{f_3}"', from=1-3, to=3-3]
	\arrow["{d_2}", from=1-5, to=1-7]
	\arrow["{f_2}", from=1-5, to=3-5]
	\arrow["{d_1}", from=1-7, to=1-9]
	\arrow["{f_1}", from=1-7, to=3-7]
	\arrow["{d_0}", from=1-9, to=1-11]
	\arrow["{f_0}", from=1-9, to=3-9]
	\arrow[from=1-11, to=1-13]
	\arrow["f", from=1-11, to=3-11]
	\arrow[from=1-13, to=3-13]
	\arrow[from=3-1, to=3-3]
	\arrow["{\delta_3}"', from=3-3, to=3-5]
	\arrow["{\delta_2}"', from=3-5, to=3-7]
	\arrow["{\delta_1}"', from=3-7, to=3-9]
	\arrow["{\delta_0}"', from=3-9, to=3-11]
	\arrow[from=3-11, to=3-13]
\end{tikzcd}\]
    Futhermore, given two such maps $f_n:P_n \to P_n'$ and $g_n:P_n\to P_n'$, there exists $s_n:P_n \to P_{n+1}'$ such that $f_n-g_n=\delta_{n+1}s_n + s_{n-1}d_n$.
\end{pro}
\begin{proof}
    Idea: use projective to get a map, and do induction.

    Part 2: Changing between straight square and slanted square, use projective module's property to get a lifting map, and perform induction.

    \todo{proof}
\end{proof}

\begin{defn}[Homotopic and homotopy equivalence]
    \hfill

    \begin{enumerate}
        \item Let $f,g: X^\bullet \to Y^\bullet$ be morphisms of complexes. We say that $f$ and $g$ are homotopic, denoted by $f\simeq g$, if there exists $s_\bullet$ be a collection of map where $s_n: X^n \to Y^{n+1}$ such that $f-g = ds + sd$.
        \item The complexes $X^\bullet$ and $Y^\bullet$ are homotopy equivalent if there exists $f:X^\bullet \to Y^\bullet$ and $f':Y^\bullet \to X^\bullet$ such that $f\circ f' \simeq \id_{Y^\bullet}$ and $f'\circ f \simeq \id_{X^\bullet}$. 
    \end{enumerate}
\end{defn}

\medskip

\begin{pro}
    \hfill

    \begin{enumerate}
        \item Suppose that $f,g:X^\bullet \to Y^\bullet$ are homotopic. We have 
        \[f^* = g^* : H^n (X^\bullet) \to H^n(Y^\bullet)\]
        \item If $X^\bullet$ and $Y^\bullet$ are homotopy equivalent, then $H^n(X^\bullet)\cong H^n(Y^\bullet)$.
    \end{enumerate}
\end{pro}
\begin{proof}
    The map $f:X^\bullet \to Y^\bullet$ induces $f^*: H^n(X^\bullet)\to H^n(Y^\bullet)$ where $[x]\mapsto [f(x)]$. This is similar for $g$. Since $f$ and $g$ are homotopic, there exists some map $s_n:X^n \to Y^{n+1}$ such that $f-g = ds + sd$. Observe that
    \begin{align*}
        f^*([x])
        &= [f(x)] \\
        &= [g(x)+ds(x) + sd(x)]\\
        &= [g(x)] + [ds(x)]\\
        &= [g(x)] + 0\\
        &= g^*([x])
    \end{align*}
    This proves the first statement.

    For the second statement, if $X^\bullet$ and $Y^\bullet$ are homotopy equivalent, it means having $f:X^\bullet\to Y^\bullet$ and $f':Y^\bullet\to X^\bullet$ such that $f\circ f' \simeq \id_{Y^\bullet}$ and $f'\circ f \simeq \id_{X^\bullet}$. These maps induces $f^*:H^n(X^\bullet) \to H^n(Y^\bullet)$ and ${f'}^*:H^n(Y^\bullet) \to H^n(X^\bullet)$. Note
    \[({f'}^*\circ f^*)([x]) = (f'\circ f)^*([x]) = (\id_{X^\bullet})^*([x]) = [x]\]
    Thus ${f'}^*\circ f^* = \id_{H^n(X^\bullet)}$. Similarly one can show that ${f}^*\circ {f'}^* = \id_{H^n(Y^\bullet)}$. This shows that $f^*$ is an isomorphism, and so $H^n(X^\bullet)$ and $H^n(Y^\bullet)$ are isomorphic.
\end{proof}

\begin{thm}
    The $n$-th cohomology group $\ext^n_R(V,W)$ is independent, up to isomorphism, of the choice of the projective residue of $V$.
\end{thm}
\begin{proof}
    \todo{proof}
\end{proof}

\begin{thm} [Snake Lemma]
    Suppose we have a commutative diagram below with exact rows:
    \[\begin{tikzcd} [sep = small]
	0 && X && Y && Z && 0 \\
	\\
	0 && {X'} && {Y'} && {Z'} && 0
	\arrow[from=1-1, to=1-3]
	\arrow["\alpha", from=1-3, to=1-5]
	\arrow["f", from=1-3, to=3-3]
	\arrow["\beta", from=1-5, to=1-7]
	\arrow["g", from=1-5, to=3-5]
	\arrow[from=1-7, to=1-9]
	\arrow["h", from=1-7, to=3-7]
	\arrow[from=3-1, to=3-3]
	\arrow["{\alpha'}", from=3-3, to=3-5]
	\arrow["{\beta'}", from=3-5, to=3-7]
	\arrow[from=3-7, to=3-9]
    \end{tikzcd}\]
    We then have an exact sequence
    \[0 \to \ker f \xto{\alpha} \ker g \xto\beta \ker h \xto\delta \coker f \xto {\alpha'} \coker g \xto {\beta'} \coker h \to 0\]
    where cokernel of a map $\phi:A\to B$ is defined as the quotient $\coker \phi = B/\im \phi$.
\end{thm}
\begin{proof}
    Left as tutorial exercise.
\end{proof}

\begin{thm} [Horseshoe Lemma]
    Let $0\to X\to Y \to Z\to 0$ be a SES of $R$-modules and $P_\bullet \twoheadrightarrow X$, $Q_\bullet \twoheadrightarrow Z$ be projective resolutions of $X,Z$ respectively. Then we have an exact commutative diagram:
    \[\begin{tikzcd} [sep=small]
	&& 0 && 0 && 0 && 0 \\
	\\
	\dots && {P_2} && {P_1} && {P_0} && X && 0 \\
	\\
	\dots && {P'_2} && {P'_1} && {P'_0} && Y && 0 \\
	\\
	\dots && {Q_2} && {Q_1} && {Q_0} && Z && 0 \\
	\\
	&& 0 && 0 && 0 && 0
	\arrow[from=1-3, to=3-3]
	\arrow[from=1-5, to=3-5]
	\arrow[from=1-7, to=3-7]
	\arrow[from=1-9, to=3-9]
	\arrow[from=3-1, to=3-3]
	\arrow["{d_2}", from=3-3, to=3-5]
	\arrow[from=3-3, to=5-3]
	\arrow["{d_1}", from=3-5, to=3-7]
	\arrow[from=3-5, to=5-5]
	\arrow["{d_0}", from=3-7, to=3-9]
	\arrow[from=3-7, to=5-7]
	\arrow[from=3-9, to=3-11]
	\arrow[from=3-9, to=5-9]
	\arrow[from=5-1, to=5-3]
	\arrow[from=5-3, to=5-5]
	\arrow["{\pi_2}"', from=5-3, to=7-3]
	\arrow[from=5-5, to=5-7]
	\arrow["{\pi_1}"', from=5-5, to=7-5]
	\arrow[from=5-7, to=5-9]
	\arrow["{\pi_0}"', from=5-7, to=7-7]
	\arrow[from=5-9, to=5-11]
	\arrow[from=5-9, to=7-9]
	\arrow[from=7-1, to=7-3]
	\arrow["{\delta_2}"', from=7-3, to=7-5]
	\arrow[from=7-3, to=9-3]
	\arrow["{\delta_1}"', from=7-5, to=7-7]
	\arrow[from=7-5, to=9-5]
	\arrow["{\delta_0}"', from=7-7, to=7-9]
	\arrow[from=7-7, to=9-7]
	\arrow[from=7-9, to=7-11]
	\arrow[from=7-9, to=9-9]
\end{tikzcd}\]
In particular, we denote the second non-zero row as $P_\bullet \oplus Q_\bullet$, and it is a projective resolution of $Y$.
\end{thm}
\begin{proof}
    Left as tutorial exercise.
\end{proof}

\begin{thm}
    Let $0\to X\to Y\to Z\to 0$ be a SES of $R$-modules. Then we have a LES of abelian groups 
    \[0 \to \h_R(Z,D) \to \h_R(Y,D) \to \h_R(X,D)\] 
    \[\to \ext^1_R(Z,D) \to \ext^1_R(Y,D) \to \ext^1_R(X,D) \to \ext^2_R(Z,D)\to \dots\]
\end{thm}
\begin{proof}
    Take projective resolution $P_\bullet \twoheadrightarrow X$ and $Q_\bullet \twoheadrightarrow Z$. By Horseshoe Lemma, we get the diagram 
    \[\begin{tikzcd} [sep=small]
	&& 0 && 0 && 0 && 0 \\
	\\
	\dots && {P_2} && {P_1} && {P_0} && X && 0 \\
	\\
	\dots && {P'_2} && {P'_1} && {P'_0} && Y && 0 \\
	\\
	\dots && {Q_2} && {Q_1} && {Q_0} && Z && 0 \\
	\\
	&& 0 && 0 && 0 && 0
	\arrow[from=1-3, to=3-3]
	\arrow[from=1-5, to=3-5]
	\arrow[from=1-7, to=3-7]
	\arrow[from=1-9, to=3-9]
	\arrow[from=3-1, to=3-3]
	\arrow["{d_2}", from=3-3, to=3-5]
	\arrow[from=3-3, to=5-3]
	\arrow["{d_1}", from=3-5, to=3-7]
	\arrow[from=3-5, to=5-5]
	\arrow["{d_0}", from=3-7, to=3-9]
	\arrow[from=3-7, to=5-7]
	\arrow[from=3-9, to=3-11]
	\arrow[from=3-9, to=5-9]
	\arrow[from=5-1, to=5-3]
	\arrow[from=5-3, to=5-5]
	\arrow["{\pi_2}"', from=5-3, to=7-3]
	\arrow[from=5-5, to=5-7]
	\arrow["{\pi_1}"', from=5-5, to=7-5]
	\arrow[from=5-7, to=5-9]
	\arrow["{\pi_0}"', from=5-7, to=7-7]
	\arrow[from=5-9, to=5-11]
	\arrow[from=5-9, to=7-9]
	\arrow[from=7-1, to=7-3]
	\arrow["{\delta_2}"', from=7-3, to=7-5]
	\arrow[from=7-3, to=9-3]
	\arrow["{\delta_1}"', from=7-5, to=7-7]
	\arrow[from=7-5, to=9-5]
	\arrow["{\delta_0}"', from=7-7, to=7-9]
	\arrow[from=7-7, to=9-7]
	\arrow[from=7-9, to=7-11]
	\arrow[from=7-9, to=9-9]
\end{tikzcd}\]
Taking $\h_R(-,D)$ we get 
\[\begin{tikzcd}[sep=small]
	&& 0 && 0 && 0 \\
	\\
	0 && {\h_R(P_0, D)} && {\h_R(P_1, D)} && {\h_R(P_2, D)} && \dots \\
	\\
	0 && {\h_R(P_0 \oplus Q_0, D)} && {\h_R(P_1 \oplus Q_1, D)} && {\h_R(P_2 \oplus Q_2, D)} && \dots \\
	\\
	0 && {\h_R(Q_0, D)} && {\h_R(Q_1, D)} && {\h_R(Q_2, D)} && \dots \\
	\\
	&& 0 && 0 && 0
	\arrow[from=3-1, to=3-3]
	\arrow[from=3-3, to=1-3]
	\arrow["{d_1^*}", from=3-3, to=3-5]
	\arrow[from=3-5, to=1-5]
	\arrow["{d_2^*}", from=3-5, to=3-7]
	\arrow[from=3-7, to=1-7]
	\arrow[from=3-7, to=3-9]
	\arrow[from=5-1, to=5-3]
	\arrow["{\iota_0^*}"', from=5-3, to=3-3]
	\arrow["{\gamma_1^*}", from=5-3, to=5-5]
	\arrow["{\iota_0^*}"', from=5-5, to=3-5]
	\arrow["{\gamma_2^*}", from=5-5, to=5-7]
	\arrow["{\iota_0^*}"', from=5-7, to=3-7]
	\arrow[from=5-7, to=5-9]
	\arrow[from=7-1, to=7-3]
	\arrow["{\pi_0^*}"', from=7-3, to=5-3]
	\arrow["{\delta_1^*}", from=7-3, to=7-5]
	\arrow["{\pi_0^*}"', from=7-5, to=5-5]
	\arrow["{\delta_2^*}", from=7-5, to=7-7]
	\arrow["{\pi_0^*}"', from=7-7, to=5-7]
	\arrow[from=7-7, to=7-9]
	\arrow[from=9-3, to=7-3]
	\arrow[from=9-5, to=7-5]
	\arrow[from=9-7, to=7-7]
\end{tikzcd}\]

Lastly, by theorem regarding LES on cohomology, we have 
\[0\to \h^0(A^\bullet)\]
\end{proof}

\begin{re}
    Note that since $\h_R(X,D)\cong \ext_R^0(X,D)$, thus the whole LES above is a LES of ext groups.
\end{re}

\medskip

\begin{thm}
    Let $Q$ be a $R$-module. TFAE:
    \begin{enumerate}
        \item $Q$ is injective.
        \item $\ext^1_R(A,Q)=0$ for all $R$-module $A$
        \item $\ext^n_R(A,Q)=0$ for all $R$-module $A$ and $n\in\Z^+$.
    \end{enumerate}
\end{thm}
\begin{proof}
    $3. \implies 2.$ is trivial.

    For $2. \implies 1.$, suppose given SES $0 \to X\to Y\to Z\to 0$, we have LES on the Ext group
    \[0 \to \h_R(Z,Q) \to \h_R(Y,Q) \to \h_R(X,Q) \to \ext_R^1(Z,D)=0\to \dots\]
    This reduces into a SES
    \[0 \to \h_R(Z,Q) \to \h_R(Y,Q) \to \h_R(X,Q) \to 0\]
    So $Q$ is injective.


\end{proof}



\begin{thm}
    Let $0\to U \xto{\alpha} V \xto{\beta} W \to 0$ be a SES of $R$-modules and $D$ be any $R$-module. Then we have a LES 
    \[0 \to \h_R(D,U) \to \h_R(0,V)\to \h_R(D,W)\]
    \[\to \ext_R^1 (D,U)\to \ext_R^1(D,V)\to \ext_R^1(D,W)\to \ext_R^2(D,U)\to \dots\]
\end{thm}
\begin{proof}
    \todo{proof}
\end{proof}

\begin{re}
    Similarly, note that since $\h_R(X,D)\cong \ext_R^0(X,D)$, thus the whole LES above is a LES of $\ext$ groups.
\end{re}

\medskip

\begin{re}
    Since $\h_R(D,-)$ is a left exact covariant functor, we obtained a right covariant derived functor $\ext_R^m(D,-):\text{R-mod} \to \text{Ab}$. This is 'dual' to that the relation between $\h_R(-,D)$ and $\ext_R^n(-,D)$.
\end{re}

\medskip

\begin{thm}
    Let $P$ be an $R$-module. TFAE:
    \begin{enumerate}
        \item $P$ is projective.
        \item $\ext_R^1(P,B)=0$ for all $R$-module $B$.
        \item $\ext_R^n(P,B)=0$ for all $n\geq 1$ and $R$-module $B$.
    \end{enumerate}
\end{thm}
\begin{proof}
    The proof is left as a tutorial question.
\end{proof}

\begin{ex}
    \hfill
    
    \begin{enumerate}
        \item For any free $R$-module $F$, we have $\ext_R^n(F,B)=0$ for any $n\geq 1$ and $R$-module $B$. In particular $\ext_\Z^n(\Z^m,B)=0$ for all $n\geq 1$ and $m\geq 1$.
        \item We compute $\ext_\Z^n(A,B)$ when $A$ is finitely generated $\Z$-module. Since $\Z$ is a PID, by the classification of finitely generated module over PID, we have 
        \[A\cong \Z^m \oplus (\Z/d_1\Z) \oplus \dots \oplus (\Z/d_\ell \Z) \]
        where $d_i\neq 0 $ for all $i$. In tutorial we will prove that
        \[\ext_R^n\br{\bigoplus V_i}, W\cong \prod \ext_R^n(V_i, W)\]
        But we have only finite number of items (since finitely generated), so 
        \[\ext_\Z^n(A,B)\cong \br{\bigoplus_{i=1}^m \ext_\Z^n(\Z,B)} \oplus \br{\bigoplus_{i=1}^\ell\ext_\Z^n(\Z/d_i\Z, B)}\]
    \end{enumerate}
    So when $n=0$ we have
    \[\ext_\Z^0(A,B) \cong \br{\bigoplus_{i=1}^m \h_\Z(\Z,B)} \oplus \br{\bigoplus_{i=1}^\ell\h_\Z(\Z/d_i\Z, B)}\cong B^m \oplus  \bigoplus_{i=1}^\ell\ _{d_i}B\]
    where $_{d_i}B:= \sbr{b\in B : d_i b = 0}$.

    When $n=1$ we have
    \[\ext_\Z^1(A,B) \cong 0 \oplus \bigoplus_{i=1}^\ell \ext_\Z^1 (\Z/d_i\Z, B) = \bigoplus B/d_i B\]
    When $n\geq 2$ we have 
    \[\ext_\Z^n(A,B)\cong 0 \oplus \bigoplus_{i=1}^\ell \ext_\Z^n(\Z/d_i\Z, B) = 0\]
\end{ex}

\medskip

\begin{defn} [Injective resolution]
    An injective resolution of an $R$-module $W$ is an exact sequence 
    \[0 \to W \to Q_0 \to Q_1 \to \dots\]
    such that each $Q_i$ is injective $R$-module.
\end{defn}

\medskip

\begin{pro}
    Every $R$-module has an injective resolution.
\end{pro}
\begin{proof}
    \todo{proof}
    Similar to the proof in the case of projective resolution, except that we will be taking the cokernel.
\end{proof}

\begin{defn} [Alternative definition of $\ext$]
    Take an injective resolution $W\hookrightarrow Q_\bullet$ of $W$ and take $\h_R(V,-)$. which we get 
    \[0 \xto{d_0^*} \h_R(V,Q_0) \xto{d_0^*} \h_R(V,Q_0) \xto{d_0^*} \h_R(V,Q_0) \to \dots \]
    The $n$-th cohomology group of this complex is defined as 
    \[\ext_R^n(V,W) := \frac{\ker d_{n+1}^*}{\im d_n^*}\]
\end{defn}

\medskip

\begin{re}
    Despite we are unable to prove that, but we have the following fact: the Ext group constructed from injective resolution is independent of the choice of the starting injective resolution. Also, the Ext group constructed from injective resolution is isomorphic to if it is constructed from a projective resolution.
\end{re}

\medskip

\begin{ex}
    We verify that the $\ext^0_R(V,W)$ constructed from projective resolution and injective resolution is isomorphic. Starting from $0$-th Ext group constructed from an injective resolution:
    \begin{align*}
        \ext_R^0(V,W) &= \ker d_1^* \\
        &= \sbr{f:V\to Q_0:d_1\circ f = 0}\\
        &= \sbr{f:V\to Q_0:\im f\subseteq \ker d_1}\\
        &= \sbr{f:V\to Q_0 : \im f\subseteq \im \iota, \iota: W\hookrightarrow Q_0}\\
        &= \sbr{f:V\to \iota(W)}\\
        &\cong \h_R(V,W)
    \end{align*}
    We have shown that the $0$-th Ext group constructed from projective resolution is also isomorphic to $\h_R(V,W)$, this shows the $0$-th Ext group is independet of the method of construction.
\end{ex}

\medskip

\begin{re} [Enough projective and enough injective]
    In fact, all the mentioned theory can be generalized to any category.
    
    A category $\mathcal C$ has enough projective if for any object $X$ in $\mathcal C$ there is a projective object $P$ such that $P\to X$ is an epimorphism. Since the definition of projective module is nothing except of lifting of maps, we need not 'module-like' object to define a projective object in the category $\mathcal C$, assuming that we can 'lift' the map.

    Similarly, the category has enough injective if for any object $X$ in $\mathcal C$ there is an injective object $I$ such that $X\to I$ is a monomorphism. Construction of injective object shares the same philosophy, as it just requires construction of maps.

    In the category R-mod, it is nice in the sense that it has both enough projective and enough injective. However, there might be some category where it is only enough projective, or vice versa. In this case, we might be restricted to construct the Ext group only from the projective resolution, or vice versa. The above says that they are equivalent.
\end{re}

\medskip

\begin{ex}
    \hfill

    \begin{enumerate}
        \item Let $A$ and $B$ be abelian groups. We compute $\ext_\Z^n(A,B)$ (again) using the injective resolution of $B$. Let $Q_0$ be an injective $\Z$-module such that $B\subseteq Q_0$. So
        \[0 \to B\to Q_0 \to Q_0/B \to 0\]
        Recall that the quotient of injective module is injective, so $Q_0/B$ is injective, and the above is an injective resolution of $B$. Taking hom we have 
        \[0 \to \h_\Z(A,Q_0) \to \h_\Z(A, Q_0/B) \to 0 \to 0 \to \dots\]
        We see that $\ext_\Z^n(A,B)=0$ for all $n\geq 2$, even without the assumption that $A$ is finitely genererated.
        \item Let $A$ be a torsion abelian group, i.e. any $a\in A$ there exists some $n\neq 0$ such that $n\cdot a = 0$. We compute $\ext_\Z^0(A,\Z) = \h_\Z(A,\Z)$, which is $0$, since for any group homomorphism $\varphi:A\to \Z$ and any $a\in A$, let $n\neq 0$ such that $n\cdot a = 0$, and so 
        \[0 = \varphi(0) = \varphi(na) = n\varphi(a)\]
        implying that $\varphi(a)=0$. Since $a$ is arbitrary, so $\varphi$ is the zero map, indicating that $\ext_\Z^0(A,\Z)$ is trivial. Next, we consider $\ext_\Z^1(A,\Z)$. Take 
        \[0 \to \Z \xto{\iota} \Q \xto{\pi} \Q/\Z \to 0\]
        where $\pi$ is the canonical surjectiion, and $\iota$ is the inclusion map. The above is an injective resolution of $\Z$. Taking hom, we have 
        \[0 \to \h_\Z(A,\Q)\to \h_\Z(A, \Q/\Z) \to 0 \to 0 \to \dots\]
        We claim that $\h_\Z(A,\Q)=0$. To see this, take $\varphi:A\to \Q$ be a group homomorphism and let $a\in A$ and $n\neq 0$ such that $n\cdot a = 0$. Note
        \[0 = \varphi(0) = \varphi(na) = n\varphi(a)\]
        Similarly, since $a$ is arbitrary, we have that $\varphi$ is trivial, so $\h_\Z(A,\Q)=0$. Lastly, simply take the first cohomology group and we obtained
        \[\ext_\Z^1 (A,\Z) = \h_\Z(A,\Q/\Z)\]
        where $\h_\Z(A,\Q/\Z)$ is called the Pontryagin dual group of $A$.
    \end{enumerate}
\end{ex}

\medskip

\begin{re}
    The reason why Ext group has its name is because of the following theorem:

    \medskip

    \begin{thm}
        $\ext^n_R(V,W)$ is the equivalence classes of $n$-fold extensions of exact sequences that takes the following form 
        \[0 \to W \to V_{n-1}\to \dots \to V_0 \to V \to 0\]
    \end{thm}

    When $n=1$, we are considering the equivalence classes of the exact sequences of the form $0\to W \to V_0 \to V \to 0$, which is just SES.

    As an example, take $R=\Z$ and $V=W = \Z/p\Z$. Based on the previous computed example we have that 
    \[\ext_\Z^1(\Z/p\Z, \Z/p\Z) \cong \frac{\Z/p\Z}{p\cdot (\Z/p\Z)} = \Z/p\Z\]
    That is to mean that there are $p$ equivalent classes of SES in the form of 
    \[0 \to \Z/p\Z \to - \to \Z/p\Z \to 0\]
    In particular, they are either equivalent to 
    \[0 \to \Z/p\Z \xto{\iota} (\Z/p\Z)\oplus (\Z/p\Z) \xto{\pi} \Z/p\Z \to 0\]
    or, for any $j=1,2, \dots, p-1$, that 
    \[0 \to \Z/p\Z \to \Z/p^2\Z \xto{\cdot j} \Z/p\Z \to 0\]
    where the formal one splits and the latter one doesn't.
\end{re}


\newpage
\subsection{Tor group}
\begin{defn} [Tor group]
    Let $D$ be a right $R$-module, $B$ be a left $R$-module and $P_\bullet \twoheadrightarrow B$ be a projective resolution of $B$:
    \[\dots \to P_2\xto{d_2} P_1 \xto{d_1} P_0 \xto{\varepsilon} B \to 0 \to \dots\]
    We then construct the complex by taking tensor product over $R$:
    \[\dots \xto{1\ten d_3} D\ten_R P_2 \xto{1\ten d_2} D\ten_R P_1 \xto{1\ten d_1} D\ten_R P_0 \to 0\]
    which is indeed a complex since $(1\ten d_n)\circ (1\ten d_{n+1}) = 1\ten (d_n\circ d_{n+1}) = 0$, but it is not exact. In fact $D\ten_R -$ is a right exact functor. The $n$-th Tor group is defined to be the $n$-th homology group of this complex, i.e.
    \[\tor_n^R(D,B) := \frac{\ker \br{1\ten d_n}}{\im \br{1\ten d_{n+1}}}\]
    The functor $\tor_n^R(D, -)$ is the left covariant derived functor of the right covariant functor $D\ten_R -$.
\end{defn}

\medskip

\begin{pro}
    $\tor_0^R(D,B)\cong D\ten_R B$.
\end{pro}
\begin{proof}
    \todo{proof}
\end{proof}
%\medskip

\begin{re}
    Given group $G$, recall the invariant subgroup $A^G$ of a $G$-module $A$ is defined to be
    \[A^G:= \sbr{a\in A: ga = a\ \forall g\in G}\]
    In fact, we can view it as a factor $-^G: \text{Ab} \to G\text{-mod}$. It is a left exact functor. Then, group cohomology is then the right-derived functor of the invariant subgroup functor $-^G$. Roughly speaking, group cohomology measures the failure of being fixed by $G$. For example, if a $G$-module $A$ is invariant under $G$, then it should have trivial group cohomology. 
\end{re}

\medskip

\begin{ex}
    We compute the cohomology $\ext_{\Z G}^n (\Z, A)$ of the finite cyclic group $G=C_n = \langle \sigma \rangle$. By definition
    \[\Z G = \sbr{\sum_{i=0}^{n-1} n_i \sigma^i:n_i\in \Z} \]
    Consider $N= 1 + \sigma + \sigma^2 \pd \sigma^{n-1}$. We have a free resolution of $\Z$ given by 
    \[\dots \xto{\cdot (\sigma-1)} \Z G \xto{\cdot N} \Z G \xto{\cdot (\sigma-1)} \Z G \xto{\ep} \Z \to 0\]
    where 
    \[\ep(g)=1 \quad \text{and} \quad \ep\br{\sum n_i \sigma^i} = \sum_{i=0}^{n-1} n_i\]
    Since $G$ is abelian, so $\Z G$ is commutative, thus we have
    \[(\sigma-1)N = N(\sigma-1) = (1+\sigma \pd \sigma^{n-1})(\sigma-1) = \sigma^n-1=0\]
    Thus $\im (\cdot (\sigma-1))\subseteq \ker (\cdot N)$, and $\im (\cdot N) \subseteq \ker(\cdot (\sigma-1))$.

    Next, to show that $\ker (\cdot (\sigma-1))\subseteq \im (\cdot N)$, let $\sum n_i \sigma^i\in \ker (\cdot (\sigma-1))$. By definition
    \[0=(\sigma-1)(\sum n_i \sigma^i) = \sum_{j=0}^{n-1} (n_{j-1} - n_j) \sigma^j\]
    This means that all coefficients are zero, so $n_{j-1} = n_j$ for all $j$, and thus we have $\sum n_i \sigma^i = \lambda N$ for some $\lambda\in \Z$.

    Also, we have to show that $\ker (\cdot N) \subseteq \im (\cdot (\sigma-1))$. First note that for every $g\in G$, we have 
    \[\ep((\sigma-1)\cdot g) = \ep (\sigma g - g) = 1-1 = 0\]
    suggesting that $\im(\cdot (\sigma-1))\subseteq\ker \ep$. Let $\sum n_i \sigma^i \in \ker \ep$, i.e. $\sum n_i = 0$. So
    \[\ker \ep = \operatorname{span} _\Z \sbr{1-\sigma, \sigma-\sigma^2 \many \sigma^{n-2} - \sigma^{n-1}}\]
    so $\ker \ep \subseteq \im (\cdot (\sigma-1))$.

    In short, this shows that the above complex is indeed a free resolution. 

    Next, we compute the Ext group. Taking hom and ignore the first term we have
    \[0 \to \h_{\Z G} (\Z G, A) \xto{(\cdot (\sigma-1))_*}  \h_{\Z G} (\Z G, A) \xto{(\cdot N)_*}  \h_{\Z G} (\Z G, A) \xto{(\cdot (\sigma-1))_*} \]
    This is equivalent to
    \[0 \to A \xto{\cdot (\sigma-1)} A \xto{\cdot N} A \xto{\cdot (\sigma-1)} \dots\]
    We can now compute the Ext group: when $n$ is $0$
    \[\ext_{\Z G}^0 (\Z, A) \cong A^{\Z G} = A^G\]
    When $n\neq 0$ is even:
    \[\ext_{\Z G}^n (\Z, A) = \frac{\ker (\cdot (\sigma -1))}{\im (\cdot N)} = \frac{_{\sigma-1} A}{NA} = \frac{A^G}{NA}\]
    and when $n$ is odd we have 
    \[\ext_{\Z G}^n (\Z, A) = \frac{\ker (\cdot N)}{\im (\cdot (\sigma-1))} = \frac{_N A}{(\sigma-1)A}\]
\end{ex}

\medskip

\begin{defn} [$n$-cochain of $G$ with coefficient in $A$]
    Let $A$ be a $G$-module. Define $C^0(G,A):= A$, and for all $n\geq 1$ define
    \[C^n(G,A) := \sbr{\text{maps from $G^n$ to $A$}}\]
    where maps refer to general maps. These are called the $n$-cochain of $G$ with coefficient in $A$.
\end{defn}

\medskip

\begin{pro}
    Let $A$ be a $G$-module. For $n\geq 1$, the $n$-cohcain $C^n(G,A)$ is an abelian group defined by the group operation 
    \[(\alpha+\beta)(g_1 \many g_n) = \alpha(g_1 \many g_n) + \beta(g_1 \many g_n)\]
    where $\alpha, \beta\in C^n(G,A)$. In particular, $C^n(G,A)$ is an abelian group for all $n$. 

    Furthermore, we have isomorphism
    \[\begin{tikzcd}[sep=small]
	& {(g_1 \many g_n) \mapsto \beta(1| g_1 |\dots | g_n)} && \beta \\
	{\Phi:} & {C^n(G,A)} && {\h_{\Z G} (F_n, A)} & {: \Psi} \\
	& \alpha && {\Phi_\alpha:1 |g_1| \dots | g_n \mapsto \alpha(g_1\many g_n)}
	\arrow["\in"{marking, allow upside down}, draw=none, from=1-2, to=2-2]
	\arrow["\Psi"{description}, maps to, from=1-4, to=1-2]
	\arrow["\in"{marking, allow upside down}, draw=none, from=1-4, to=2-4]
	\arrow[tail reversed, from=2-2, to=2-4]
	\arrow["\in"{marking, allow upside down}, draw=none, from=3-2, to=2-2]
	\arrow["\Phi"{description}, maps to, from=3-2, to=3-4]
	\arrow["\in"{marking, allow upside down}, draw=none, from=3-4, to=2-4]
    \end{tikzcd}\]
\end{pro}
\begin{proof}
    The proof of $C^n(G,A)$ being abelian group is clear, thus omitted.

    For the second part of the statement, we start by first showing that $\Phi$ is well-defined. Let $\alpha\in C^n(G,A)$, so $\alpha:G^n \to A$ is a map. Note $F_n$ has a $\Z G$-basis 
    \[B:=\sbr{1\mid g_1 \mid \dots \mid g_n : g_1 \many g_n \in G}\]
    This implies that $F_n$ is a free module. By the universal property of free module, we have the following commutative diagram:
    \[\begin{tikzcd}[sep=small]
	{(1\mid g_1 \mid \dots \mid g_n)} & B && {F_n} & {(1\mid g_1 \mid \dots \mid g_n)} \\
	\\
	&&& A \\
	&&& {\alpha(1\mid g_1 \mid \dots \mid g_n)}
	\arrow["\in"{marking, allow upside down}, draw=none, from=1-1, to=1-2]
	\arrow[curve={height=-18pt}, maps to, from=1-1, to=1-5]
	\arrow[maps to, from=1-1, to=4-4]
	\arrow[hook, from=1-2, to=1-4]
	\arrow["\alpha"', from=1-2, to=3-4]
	\arrow["{\exists \Phi_\alpha}", dashed, from=1-4, to=3-4]
	\arrow["\in"{marking, allow upside down}, draw=none, from=1-5, to=1-4]
	\arrow[dashed, maps to, from=1-5, to=4-4]
	\arrow["\in"{marking, allow upside down}, draw=none, from=4-4, to=3-4]
    \end{tikzcd}\]
    so we now obtain $\Phi_\alpha$ that maps from the $\Z G$-basis $B$ of $G^n$ to $A$. We can extend the map $\Phi_\alpha$ linearly such that the $\Phi_\alpha$ maps from $G^n$, since $B$ is a basis. It is ensured by the universal property that $\Phi_\alpha$ is indeed an $\Z G$-module homomorphism. This shows that $\Phi$ is indeed well-defined.

    Next we show that $\Phi$ is a group homomorphism. Let $\alpha,\alpha'\in C^n(G,A)$. Then
    \begin{align*}
        (\Phi(\alpha+\alpha'))(g_1 \many g_n) &= \Phi_{\alpha+\alpha'} (g_1 \many g_n) \\
        &= (\alpha+\alpha')(1 \mid g_1 \mid \dots \mid g_n) \\
        &= \alpha(1 \mid g_1 \mid \dots \mid g_n) + \alpha'(1 \mid g_1 \mid \dots \mid g_n) \\
        &= \Phi_\alpha(g_1 \many g_n) + \Phi_{\alpha'}(g_1 \many g_n) \\
        &= (\Phi_\alpha + \Phi_{\alpha'})(g_1 \many g_n) \\
        &= (\Phi(\alpha) + \Phi(\alpha'))(g_1 \many g_n)
    \end{align*}
    This shows that $\Phi(\alpha+\alpha') = \Phi(\alpha) + \Phi(\alpha')$.

    On the other hand, there is no ambiguity in the well-definedness of $\Psi$. We prove that $\Psi$ is indeed a group homomorphism: Let $\beta, \beta' \in \h_{\Z G}(F_n,A)$, then
    \begin{align*}
        (\Psi(\beta+\beta'))(g_1 \many g_n) &= (\beta+\beta')(1\mid g_1 \mid \dots \mid g_n) \\
        &= \beta(1\mid g_1 \mid \dots \mid g_n) + \beta'(1\mid g_1 \mid \dots \mid g_n) \\
        &= (\Psi(\beta))(g_1 \many g_n) + (\Psi(\beta'))(g_1 \many g_n)\\
        &= (\Psi(\beta) + \Psi(\beta'))(g_1 \many g_n)
    \end{align*}
    This shows that $\Psi(\beta + \beta') = \Psi(\beta) + \Psi(\beta')$.

    Lastly, we show that $\Phi$ and $\Psi$ is isomorphism pair. Firstly note $(\Psi \circ \Phi)(\alpha) = \Psi (\Phi(\alpha)) = \Psi \circ \Phi_\alpha$, so
    \[((\Psi \circ \Phi)(\alpha))(g_1 \many g_n) = (\Psi \circ \Phi_\alpha)(g_1 \many g_n) = \Phi_\alpha(1\mid g_1 \mid \dots \mid g_n) = \alpha(g_1 \many g_n)\]
    So $(\Psi \circ \Phi)(\alpha) = \alpha$. Conversely, see that 
    \[((\Phi \circ \Psi)(\beta))(1\mid g_1 \mid \dots \mid g_n) = (\Psi(\beta))(g_1 \many g_n) = \beta(1\mid g_1 \mid \dots \mid g_n)\]
    This shows that $(\Phi \circ \Psi)(\beta) = \beta$. In summary, this shows that $\Psi$ and $\Phi$ are indeed invserses of each other, so the proof is completed.
\end{proof}

\begin{pro}
    Under the isomorphism $\h_{\Z G}(F_n, A) \cong C^n(G,A)$, the differential maps $d_{n+1}^*:\h_{\Z G}(F_n, A) \to \h_{\Z G} (F_{n+1}, A)$ translate to, for $n\geq 1$, that
    \[\delta_{n+1}:C^n(G,A) \to C^{n+1}(G,A),\ f\mapsto \delta_{n+1}(f)\]
    where $f\mapsto \delta_{n+1}(f)$
    \begin{align*}
        &\ \delta_{n+1}(f)(x_1 \many x_{n+1})\\
        &= x_1 f(x_2 \many x_{n+1}) + \sum_{i=1}^n (-1)^n f(x_1 \many, x_{i-1}, x_i x_{i+1} \many x_{n+1}) + (-1)^{n+1}f(x_1 \many x_n)
    \end{align*}
    and when $n=0$, the map $\delta_1 : C^0(G,A) \to C^1(G,A)$ is defined to be $a\mapsto (g\mapsto ga-a)$.
\end{pro}
\begin{proof}
    Firstly, we examine the case for $n=0$.
    \[\begin{tikzcd} [sep=small]
	{(\Phi_a:1\mapsto a)} & {\h_{\Z G} (F_0, Z)} && {\h_{\Z G}(F_1,A)} & {\Phi_a \circ d_1} \\
	\\
	a & {A=C^0(G,A)} && {C^1(G,A)} & {\Psi(\Phi_a \circ d_1)(g)}
	\arrow["\in"{marking, allow upside down}, draw=none, from=1-1, to=1-2]
	\arrow[curve={height=-24pt}, maps to, from=1-1, to=1-5]
	\arrow["{d_1^*}", from=1-2, to=1-4]
	\arrow["\Phi"', tail reversed, no head, from=1-2, to=3-2]
	\arrow["\Psi", from=1-4, to=3-4]
	\arrow["\in"{marking, allow upside down}, draw=none, from=1-5, to=1-4]
	\arrow[maps to, from=1-5, to=3-5]
	\arrow[maps to, from=3-1, to=1-1]
	\arrow["\in"{marking, allow upside down}, draw=none, from=3-1, to=3-2]
    \arrow["\in"{marking, allow upside down}, draw=none, from=3-5, to=3-4]
	\arrow["{\delta_1}"', from=3-2, to=3-4]
    \end{tikzcd}\]
    We want that the above diagram is commutative, i.e. we want to show that $\Psi(\Phi_a \circ d_1)(g) = \delta_1(a)$. Simply note
    \begin{align*}
        \Psi(\Phi_a \circ d_1)(g) &= (\Phi_a \circ d_1)(1\mid g) \\
        &= \Phi_a(d_1(1\mid g))\\
        &= \Phi_a(g-1) \\
        &= (g-1)\cdot \Phi_a(1)\\
        &= (g-1)a\\
        &= ga-a \\
        &= \delta_1(g)
    \end{align*}
    This shows that the statement holds for $n=0$. For $n\geq 1$, we have
    \[\begin{tikzcd} [sep=small]
	{\Phi_\beta} & {\h_{\Z G} (F_n, Z)} && {\h_{\Z G}(F_{n+1},A)} & {\Phi_\beta \circ d_{n+1}} \\
	\\
	\beta & {C^n(G,A)} && {C^{n+1}(G,A)} & {\Psi(\Phi_\beta \circ d_{n+1})}
	\arrow["\in"{marking, allow upside down}, draw=none, from=1-1, to=1-2]
	\arrow[curve={height=-24pt}, maps to, from=1-1, to=1-5]
	\arrow["{d_{n+1}^*}", from=1-2, to=1-4]
	\arrow["\Phi"', tail reversed, no head, from=1-2, to=3-2]
	\arrow["\Psi", from=1-4, to=3-4]
	\arrow["\in"{marking, allow upside down}, draw=none, from=1-5, to=1-4]
	\arrow[maps to, from=1-5, to=3-5]
	\arrow[maps to, from=3-1, to=1-1]
	\arrow["\in"{marking, allow upside down}, draw=none, from=3-1, to=3-2]
	\arrow["{\delta_{n+1}}"', from=3-2, to=3-4]
	\arrow["\in"{marking, allow upside down}, draw=none, from=3-5, to=3-4]
    \end{tikzcd}\]
    \todo{what?}
\end{proof}

\begin{re}
    Altogether, the previous few proposition implies that we have isomorphissm of complex 
    \[\Phi: \h_{\Z G}(F_\bullet, A) \to C^\bullet (G,A)\]
    By previous result, we see that $H^n(G,A) := H^n(C^\bullet (G,A)) \cong \ext_{\Z G}(\Z, A)$
    where 
    \[H^n(C^\bullet (G,A)) = \frac{\ker \delta_{n+1}}{\im \delta_n}\]
    Here, we call $\ker \delta_{n+1}$ as the $n$-cocycles, and $\im \delta_n$ the $n$-coboundaries.
\end{re}

\medskip 

\begin{ex}
    \hfill

    \begin{enumerate}
        \item Note the $0$-th group cohomology
        \[H^0(G,A) = \ker \delta_1 = \sbr{a\in A:\delta_1(a) = 0} = \sbr{a\in A : \forall g\in G,\ ga-a = 0}=A^G\]
        is just the submodule of $\Z G$ fixed by $G$.
        \item When $G=\sbr{1}$ is the trivial group, from previous example we see that $H^0(G,A) = A$. For $n\geq 1$, note $H^n(G,A) = H^n(C^n(G,A)) = 0$, since $C^n(G,A)$ is the maps from $G^n=\sbr{1}^n$ to $A$, where there is only one possible ma.
    \end{enumerate}
\end{ex}

\medskip 

\begin{thm} [LES in group cohomology]
    Let $0\to A \to B\to C \to 0$ be a SES of $G$-module. Then we have a LES 
    \[0 \to A^G \to B^G \to C^G \to H^1(G,A) \to H^1(G,B) \to H^1(G,C) \to H^2(G,A) \to \dots\]
\end{thm}
\begin{proof}
    The proof is immediate follows from LES in Ext groups, since by definition group cohomology is just a special type of Ext group.
\end{proof}

\begin{cor}
    Suppose that $H^n(G,B)=0$ for all $n\geq 1$, then we have exact sequence
    \[0 \to A^G \to B^G \to C^G \to H^1(G,A)\to 0\]
    and isomorphism 
    \[H^{n+1}(G,A) \cong H^n(G,C)\]
\end{cor}
\begin{proof}
    Simply apply the assumption on the LES in group cohomology. The isomorphism comes from the exactness.
\end{proof}

\begin{defn} [Induced module]
    Let $H$ be a subgroup of $G$, then $\Z H$ is a subalgebra(subring) of $\Z G$. Let $A$ be an $H$-module. We define the induced $G$-module by
    \[M_H^G(A) := \h_{\Z H} (\Z G, A) = \sbr{\varphi: \Z G \to A: h\varphi(g) = \varphi(hg) \forall h\in H, \forall g\in G}\]
    where we consider $\Z G$ as the bimodule $_{\Z H} \Z G_{\Z G}$.
\end{defn}

\medskip

\begin{re}
    In the case where $G$ is finite, then 
    \[M^G_H(A) \cong \Z G \ten_{\Z H} A\]
\end{re}
%\input{week 12}
%\input{week 12}




\end{document}