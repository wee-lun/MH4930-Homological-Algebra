\begin{pro} \label{pro: inj implies inj}
    Let $X, Y$ and $V$ be $R$-modules. Let $\beta:X\to Y$ be an $R$-module homomorphism. The map $\beta_*: \h_R(V,X)\to \h_R(V,Y)$ where $f\mapsto \beta\circ f$ is as abelian group homomorphism. Furthermore, if $\beta$ is injective, then so is $\beta_*$. In other words, the SES
    \[0\to X \xrightarrow{\beta} Y\]
    implies that we have the SES
    \[0 \to \h_R(V,X)\xrightarrow{\beta_*} \h_R(V,Y)\]
\end{pro}

\begin{proof}
    Suppose $\beta$ is injective. Let $f\in \h_R(V,X)$ such that $\beta\circ f = 0$ for every $v\in V$. Then for any $v\in V$ we have
    \[(\beta\circ f)(v)=0 \implies \beta(f(v)) = 0 \implies f(v)=0\]
    since $\beta$ is injective. This shows that $f$ is a zero map, so $\beta_*$ is injective.
\end{proof}
\begin{re}
    In general $\beta_*$ is not surjective even if $\beta$ is surjective.
\end{re}

\medskip

\begin{thm} \label{thm: all exact}
    Let $V, X, Y, Z$ be $R$-modules and
    \begin{equation}\label{eqn: short seq}
        0\to X \xrightarrow{\alpha} Y\xrightarrow{\beta} Z
    \end{equation}
    be a short sequence where $\alpha$ and $\beta$ are $R$-module homomorphisms. 
    \begin{enumerate}
        \item If the above short sequence \ref{eqn: short seq} is exact, then the following is exact:
    \[0 \to \h_R(V,X)\xrightarrow{\alpha_*}\h_R(V,Y)\xrightarrow{\beta_*}\h_R(V,Z)\]
        \item $0 \to \h_R(V,X)\xrightarrow{\alpha_*}\h_R(V,Y)\xrightarrow{\beta_*}\h_R(V,Z)$ is exact for all $V$ if and only if $0\to X \xrightarrow{\alpha} Y\xrightarrow{\beta}Z$ is exact.
    \end{enumerate}
\end{thm}
\begin{proof}
    For the first statement, suppose that the short sequence \ref{eqn: short seq} given exists. Then $\alpha$ is injective, and thus $\alpha_*$ is injective by Proposition \ref{pro: inj implies inj}. We now prove that $\im \alpha_* = \ker \beta_*$. Firstly, let $f\in \h_R(V,X)$. Since 
    \[(\beta_*\circ \alpha_*)(f) = \beta \circ \alpha \circ f = 0 \circ f = 0\]
    This implies that $\im\alpha_*\subseteq \ker\beta_*$. Next, to show $\ker\beta_*\subseteq \im \alpha_*$, let $g\in \ker \beta_*$, so for any $v\in V$ we have
    \[(\beta_*(g))(v) = \beta(g(v))=0\]
    Note it implies that $g(v) \in \ker \beta = \im\alpha$ due to the exactness of SES \ref{eqn: short seq}. So there exists $x_v\in X$ such that $\alpha(x_v) = g(v)$. Next, define the map $f:V\to X$ such that $v\mapsto x_v$. The map $f$ is well-defined since $\alpha$ is injective by assumption. We claim that $f$ is a $R$-module homomorphism. Indeed, for any $v,v'\in V$, let $\alpha(x_v) = g(v)$ and $\alpha(x_{v'}) = g(v')$. Then 
    \[\alpha(x_v + x_{v'}) = \alpha(x_v) + \alpha(x_{v'}) = g(x_v) + g(x_{v'}) = g(x_v + x_{v'}) \]
    This proves that $f(v+v) = x_{v+v'} = x_v + x_{v'} = f(v) + f(v')$. Next, let $r\in R$, then 
    \[\alpha(rx_v) = r\alpha(x_v) = rg(v) = g(rv)\]
    This proves that $f(rv) = x_{rv} = rx_v = rf(v)$, and so $f$ is really an $R$-module homomorphism. Finally, note that
    \[(\alpha_*(f))(v) = (\alpha\circ f) (v) = \alpha(f(v)) = \alpha(x_v) = g(v)\]
    Since $v\in V$ is arbitrary, we conclude that $\alpha_*(f)=g$. This proves that $\ker\beta_*\subseteq \im \alpha_*$, which also proved the first statement.

    For the second statement, note that the backward direction is equivalent to the first statement, which we have proven it to be true. For the forward direction, suppose that 
    \[0 \to \h_R(V,X)\xrightarrow{\alpha_*}\h_R(V,Y)\xrightarrow{\beta_*}\h_R(V,Z)\]
    is exact for all $V$. It suffices to take $V=R$, since we have that $\h_R(R, X) \cong X$ where the isomorphism is given by $f\mapsto f(1)$ This is similar for $Y$ and $Z$. Thus we have the following diagram:
    \[\begin{tikzcd}[sep=small]
	0 && {\h_R(R,X)} && {\h_R(R,Y)} && {\h_R(R,Z)} && 0 \\
	\\
	0 && X && Y && Z && 0
	\arrow[from=1-1, to=1-3]
	\arrow["{\alpha_*}", from=1-3, to=1-5]
	\arrow[tail reversed, from=1-3, to=3-3]
	\arrow["{\beta_*}", from=1-5, to=1-7]
	\arrow[tail reversed, from=1-5, to=3-5]
	\arrow[from=1-7, to=1-9]
	\arrow[tail reversed, from=1-7, to=3-7]
	\arrow[from=3-1, to=3-3]
	\arrow["\alpha"', from=3-3, to=3-5]
	\arrow["\beta"', from=3-5, to=3-7]
	\arrow[from=3-7, to=3-9]
    \end{tikzcd}\]
    where the linking between both sequences is the isomorphism defined above. We claim that the diagram is commutative. We only prove the commutativity between $\h_R(R,X)- \h_R(R,Y)-Y-X$, i.e. the first square. Let $f\in \h_R(R,X)$. Sending $f$ along the upper path we get $(\alpha_* (f))(1) = (\alpha\circ f)(1) = \alpha(f(1))$. On the other hand, sending $f$ along the lower path we get $\alpha(f(1))$. This shows commutativity in the first square. Same argument can be applied to show commutativity in the second square.
    
    We need to show that $\im \alpha = \ker \beta$. We first show $\im\alpha \subseteq
     \ker \beta$. Let $x\in X$, so there exists $f_x\in \h_R(R,X)$ such that $f_x(1) = x$. Sending $f_x$ along the first row, and due to exactness we have that 
    \[(\beta_*\circ \alpha_*)(f_x) = 0 \implies \beta\circ \alpha \circ f_x = 0\]
    Therefore $(\beta\circ \alpha \circ f_x)(1) = \beta(\alpha(f_x(1))) = \beta(\alpha(x)) = 0$. This proves that $\alpha(x)\in \ker\beta$, implying that $\im\alpha \subseteq \ker\beta$.
    
    Next, to show $\ker \beta \subseteq \im \alpha$, let $y\in \ker \beta$, i.e. $\beta(y) = 0$. Then there exists $f_y\in \h_R(R,Y)$ such that $f_y(1)= y$. Note that 
    \[(\beta_*(f_y))(1) = \beta(f_y(1)) = \beta(y) = 0 \in Z\]
    But $\beta_*(f_y)\in \h_R(R,Z)$, and there is an isomorphism between $\h_R(R,Z)$ and $Z$. Since $(\beta_*(f_y))(1) = 0 = 0(1)$, we see that $\beta_*(f_y)$ must be the zero map due to the injectivity of the isomorphism. This further implies that $f_y\in \ker \beta_*$, and by exactness in first row we get $f_y\in \im \alpha_*$. So, there exists $g_x\in \h_R(R,X)$ where $g_x(1):= x$ such that $\alpha_*(g_x) = f_y$. Thus 
    \[(\alpha_*(g_x))(1) = f_y(1)\implies (\alpha\circ g_x)(1) = y \implies \alpha(g_x(1)) = \alpha(x) = y\]
    Therefore $y\in \im \alpha$, which proves that $\ker \beta\subseteq \im \alpha$. This completes the proof.
\end{proof}

\begin{pro}
    Let $X, Y, Z$ be $R$-module. Then
    \begin{enumerate}
        \item $\h_R(X, Y\oplus Z)\cong \h_R(X,Y)\oplus \h_R(X,Z)$
        \item $\h_R(X\oplus Y, Z)\cong \h_R(X,Z)\oplus \h_R(Y,Z)$
    \end{enumerate}
\end{pro}
\begin{proof}
    We only prove the first statement. Consider a map from $\h_R(X, Y\oplus Z)$ to $\h_R(X,Y)\oplus \h_R(X,Z)$ such that 
    \[f\mapsto (\pi_Y\circ f, \pi_Z\circ f)\]
    This is an isomorphism of abelian groups.
\end{proof}

\begin{re}
    The above can be generalized to infinite direct sum, where we have that
    \[\h_R\br{\bigoplus_{i\in I}X_i, Y}\cong \prod_{i\in I} \h_R(X_i, Y)\]
\end{re}

\newpage


\subsection{Projective Modules and Introduction to Categories}

\begin{pro} [Equivalent Condition of Projective Module] \label{pro: Proj Mod}
    Let $P$ be an $R$-module. TFAE:
    \begin{enumerate}
        \item \textbf{Any} SES $0 \to X\xrightarrow{\alpha} Y\xrightarrow{\beta} Z\to 0$ gives rise to a SES 
        \[0 \to \h_R(P,X)\xrightarrow{\alpha_*}\h_R(P,Y)\xrightarrow{\beta_*}\h_R(P,Z)\to 0\]
        \item For any surjective $R$-module homomorphism $\beta:Y\to Z$ and any $R$-module homomorphism $f:P\to Z$, there exists $R$-module homomorphism $g: P\to Y$, which is called a lift, such that $\beta \circ g = f$, i.e.
        \[\begin{tikzcd}[sep=small]
	&& P \\
	\\
	Y && Z && 0
	\arrow["{\exists g}"', dashed, from=1-3, to=3-1]
	\arrow["f", from=1-3, to=3-3]
	\arrow["\beta"', from=3-1, to=3-3]
	\arrow[from=3-3, to=3-5]
\end{tikzcd}\]
        \item Every SES $0\to X\to Y\to P\to 0$ splits. In this case $P$ is a direct summand of $Y$, that is there exists an $R$-module $Y'$ such that $Y\cong Y'\oplus P$. We write $P\mid Y$.
        \item $P$ is a direct summand of a free $R$-module.
    \end{enumerate}
    If $P$ satisfies any of these equivalent conditions, we call $P$ a projective module.
\end{pro}
\begin{proof}
    $[1.\implies 2.]$ Consider the SES 
    \[0 \xto{} \ker \beta \xto{} Y \xto{\beta} Z\xto{}0\]
    By assumption this gives rise the folliowing SES:
    \[0 \xto{} \h_R(P,\ker \beta)\xto{} \h_R(P,Y) \xto{\beta_*} \h_R(P,Z) \xto{} 0\]
    where $\beta_*: g\mapsto\beta\circ g$. Note by second statement of Theorem \ref{thm: all exact} says that $\beta_*$ is surjective. Thus for any $f\in \h_R(P,Z)$ there exists a $g_f\in \h_R(P, Y)$ such that $\beta_*(g) = \beta\circ g = f$.

    $[2. \implies 3.]$ Suppose as stated by the statement. Consider the following diagram:
    \[\begin{tikzcd}[sep=small]
	&&&&&& P \\
	\\
	0 && X && Y && P && 0
	\arrow["{\exists g}"', dashed, from=1-7, to=3-5]
	\arrow["{\id_P}", tail reversed, from=1-7, to=3-7]
	\arrow[from=3-1, to=3-3]
	\arrow["\iota", from=3-3, to=3-5]
	\arrow["\pi", from=3-5, to=3-7]
	\arrow[from=3-7, to=3-9]
\end{tikzcd}\]
    where the existence of $g$ is ensure by the assumption and that $\pi\circ g = \id_P$. By the second statement of Proposition \ref{pro: SES splits}, the existence of $g$ implies that the SES splits.
    
    $[3. \implies 4.]$ Suppose as stated in the statement. Since every $R$-module is a quotient of a free module, define $F(S)$ be a free module such that $F(S)$ surjects to $P$ via map $\pi$. Then consider the SES
    \[0 \to \ker \pi \to F(S) \xto{\pi} P \to 0\]
    By the assumption, the above SES splits, and thus $F(S) = \ker\pi \oplus P$, showing that $P\mid F(S)$. 

    $[4.\implies 1.]$. Suppose as stated in the statement. Assume that we have an SES $0 \to X \xto{\alpha} Y \xto{\beta} Z\to 0$, and consider the short sequence
    \[0 \to \h_R(P,X)\xto{\alpha_*} \h_R(P,Y) \xto{\beta} \h_R(P,Z)\to 0\]
    Immediately by Theorem \ref{thm: all exact} we have that $\alpha_*$ is injective, so it suffices to show that $\beta_*$ is surjective. Take $f\in \h_R(P,Z)$. By assumption $F(S) \cong P \oplus P'$ for some free module $F(S)$ and $P'\subseteq F(S)$ is an $R$-module. Consider the following diagram:
    \[\begin{tikzcd}[sep=small]
	s && s \\
	S && {F(S)} \\
	&& P \\
	Y && Z
	\arrow[maps to, from=1-1, to=1-3]
	\arrow["\in"{marking, allow upside down}, draw=none, from=1-1, to=2-1]
    \arrow["\in"{marking, allow upside down}, draw=none, from=1-3, to=2-3]
	\arrow[hook, from=2-1, to=2-3]
	\arrow["\pi", two heads, from=2-3, to=3-3]
	\arrow["\iota"', shift right=2, curve={height=12pt}, from=3-3, to=2-3]
	\arrow["f", from=3-3, to=4-3]
	\arrow["\beta"', from=4-1, to=4-3]
    \end{tikzcd}\]
    where $\pi$ is the canonical map from $F(S)$ to $P$ and $\iota$ is the inclusion map from $P$ to $F(S)$. Note that $\pi\circ \iota = \id_P$. Next, define the map $\varphi:S \to Y$ where $s\mapsto m_s$ if $\beta(m_S) = (f\circ \pi)(s)$. The map $\varphi$ is inded well-defined since $\beta$ is surjective. Thus we now have the following diagram:
    \[\begin{tikzcd}[sep=small]
	& s && s \\
	& S && {F(S)} \\
	&&& P \\
	{m_s} & Y && Z & {(f\circ \pi)(s) = \beta(m_s)}
	\arrow[maps to, from=1-2, to=1-4]
	\arrow["\in"{marking, allow upside down}, draw=none, from=1-2, to=2-2]
	\arrow[curve={height=12pt}, maps to, from=1-2, to=4-1]
	\arrow["\in"{marking, allow upside down}, draw=none, from=1-4, to=2-4]
	\arrow[curve={height=-12pt}, maps to, from=1-4, to=4-5]
	\arrow[hook, from=2-2, to=2-4]
	\arrow["\varphi"', from=2-2, to=4-2]
	\arrow["\pi", two heads, from=2-4, to=3-4]
	\arrow["{\exists g}", dashed, from=2-4, to=4-2]
	\arrow["\iota"', shift right=2, curve={height=12pt}, from=3-4, to=2-4]
	\arrow["f", from=3-4, to=4-4]
	\arrow["\in"{marking, allow upside down}, draw=none, from=4-1, to=4-2]
	\arrow[curve={height=18pt}, maps to, from=4-1, to=4-5]
	\arrow["\beta"', from=4-2, to=4-4]
	\arrow["\in"{marking, allow upside down}, draw=none, from=4-5, to=4-4]
    \end{tikzcd}\]
    where the existence of $g$ follows from the universal property of free module. In the diagram, the upper-triangular part is commutative, and we claim that the lower-triangular part is also commutative, i.e. $\beta\circ g = f\circ \pi$. 

    We first show that $\beta\circ g = f\circ \pi$ when restricted to $S$, or more precisely, the image of $S$ in $F(S)$. This is easy, since for any $s\in S$ we have
    \[(\beta\circ g)(s) = \beta(g(s)) = \beta(\varphi(s))=\beta(m_s) = (f\circ \pi)(s)\]
    Next, consider the following commutative diagram:
    \[\begin{tikzcd}[sep=small]
	s && s \\
	S && {F(S)} \\
	\\
	&& Z
	\arrow[maps to, from=1-1, to=1-3]
	\arrow["\in"{marking, allow upside down}, draw=none, from=1-1, to=2-1]
	\arrow["\in"{marking, allow upside down}, draw=none, from=1-3, to=2-3]
	\arrow[hook, from=2-1, to=2-3]
	\arrow["{f\circ \pi = \beta \circ g}"', shift right=2, from=2-1, to=4-3]
	\arrow["{f\circ \pi}"', shift right=2, from=2-3, to=4-3]
	\arrow["{\beta\circ g}", shift left, from=2-3, to=4-3]
    \end{tikzcd}\]
    where the commutativity follows from the proven statement that $\beta \circ g = f \circ \pi$ when restricted on $S$. Then, by the uniqueness of the universal property of free module $F(S)$, we must have that $\beta\circ g = f \circ \pi$ on $F(S)$. This proves our claim. 

    Recall that we need to prove that $\beta_*$ is surjective, in particular we have been given $f\in \h_R(P,Z)$ and we want to look for its pre-image under $\beta_*$. Consider $g\circ \iota:P\to Y$, so $g\circ \iota \in \h_R(P,Y)$. Then
    \[\beta_*(g\circ \iota) = \beta\circ g \circ \iota \overset{(*)}{=} f\circ \pi \circ \iota \overset{(**)}{=} f\circ \id_P = f\]
    where in $(*)$ we use the fact that $\beta\circ g = f\circ \pi$ and in $(**)$ we use the fact that $\pi\circ \iota = \id_P$. We have shown that $g\circ \iota$ is the pre-image of $f$ under $\beta_*$, thus showing that $\beta_*$ is surjective. The proof is then completed. 
\end{proof}

\begin{re}
    As shown and stated previously in Proposition \ref{pro: inj implies inj}, if $\beta: X \to Y$ is exact, then $\beta_*: \h_R(V,X) \to \h_R(V,Y)$ is exact for any $V$, but this statement need not holds when we replace injectivity with surjectivity.  In particular, we have the statement: Let $V$ be an $R$-module. Then TFAE
    \begin{itemize}
        \item $V$ is projective.
        \item the SES $Y \xto{\beta} Z \to 0$ gives rise to the SES $\h_R (V,Y) \xto{\beta_*} \h_R(V,Z)\to 0$.
    \end{itemize}
    which is a direct consequence of the first statement of Proposition \ref{pro: Proj Mod}.
\end{re}

\medskip

\begin{cor}
    \hfill

    \begin{enumerate}
        \item Free modules are projective.
        \item A (finitely generated) $R$-module is projective if and only if it is a direct summand of a (finitely generated) free module.
        \item Direct sum of projective module is projective.
        \item Every module is a quotient of projective module. 
    \end{enumerate}
\end{cor}
\begin{proof}
    For the first statement, if $F$ is a free module, since $F\cong F\oplus 0$, so $F$ is projective.

    For the second statement, the statement is true by Proposition \ref{pro: Proj Mod}(4.). We check the finitely generated part. Suppose that $P$ is finitely generated, then there exists $S$ finite cardinality such that $F(S)$ surjects onto $P$, so $P\mid F(S)$. For the converse, if $P\mid F(S)$ where $S$ has finite cardinality, then $F(S)$ surjects to $P$ by the canonical map, $\pi$, so $P$ is finitely generated by the image of $\pi(S)$.

    Third statement is a tutorial question.

    For fourth statement, every module is a quotient of free module, and is thus a quotient of projective module.
\end{proof}

\begin{re}
    Projective module is nice. One of the reasons is that, according to Proposition \ref{pro: Proj Mod}, for a projective module $P$, it suffices to only discuss on its $\h$ set, i.e. we only have to talk about maps. It might seems complicated, but this provides us a huge space to carry out abstraction. In the light of this, we introduce some basic categorical notation.
\end{re}

\medskip

\begin{defn} [Category]
    A category $\cat$ consists of the following:
    \begin{enumerate}
        \item A class of objects $\obj(\mathcal C)$
        \item For any two objects $X$ and $Y$, we have a class of morphisms (i.e. maps) $\mor_\cat(X,Y)$
        \item For any objects $X$, $Y$, and $Z$, we have a binary operation $\mor_\cat(X,Y)\times \mor_\cat (Y,Z)\to \mor_\cat(X,Z)$ such that $(f,g)\mapsto g\circ f$, such that
        \begin{itemize}
            \item the operation is associative
            \item $\mor_\cat(X,X)$ contains an identity $1_X$ such that for any $g\in \mor_\cat(X,Y)$ and $h\in \mor_\cat(Z,X)$, we have $g\circ 1_X = g$ and $1_X\circ h = h$
        \end{itemize}
    \end{enumerate}
\end{defn}

\medskip 

\begin{defn} [Covariant functor]
    Let $\cat$ and $\mathcal D$ be a categories. A covariant functor $\f:\cat \to \mathcal D$ consists of the following things:
    \begin{enumerate}
        \item For any object $X\in \obj(\cat)$, we have an object $\f(X)\in \mathcal D$
        \item For any morphism $\alpha:X\to Y$ of $\cat$, we have a morphism $\f(\alpha):\f(X)\to \f(Y)$ of $\mathcal D$ such that the following holds:
        \begin{itemize}
            \item $\f(1_X) = 1_{\f(X)}$
            \item If we have the commutative diagram
            \[\begin{tikzcd}[sep=small]
	        X && Y \\
	        \\
	        && Z
	        \arrow["\alpha", from=1-1, to=1-3]
	        \arrow["{\beta\circ \alpha}"', from=1-1, to=3-3]
	        \arrow["\beta", from=1-3, to=3-3]
            \end{tikzcd}\]
            then we have the commutative diagram:
            \[\begin{tikzcd}[sep=small]
	        {\f(X)} && {\f(Y)} \\
	        \\
	        && {\f(Z)}
	        \arrow["{\f(\alpha)}", from=1-1, to=1-3]
	        \arrow["{\f(\beta\circ \alpha) = \f(\beta) \circ \f(\alpha)}"', shift right=2, from=1-1, to=3-3]
	        \arrow["{\f(\beta)}", from=1-3, to=3-3]
            \end{tikzcd}\]
        \end{itemize}
    \end{enumerate}
\end{defn}

\medskip 

\begin{defn} [Contravariant functor]
    Let $\cat$ and $\mathcal D$ be a categories. A contravariant functor $\f:\cat \to \mathcal D$ consists of the following things:
    \begin{enumerate}
        \item For any object $X\in \obj(\cat)$, we have an object $\f(X)\in \mathcal D$
        
        \item For any morphism $\alpha:X\to Y$ of $\cat$, we have a morphism $\f(\alpha):\f(Y)\to \f(X)$ of $\mathcal D$ such that the following holds:
        \begin{itemize}
            \item $\f(1_X) = 1_{\f(X)}$
            \item If we have the commutative diagram
            \[\begin{tikzcd}[sep=small]
	        X && Y \\
	        \\
	        && Z
	        \arrow["\alpha", from=1-1, to=1-3]
	        \arrow["{\beta\circ \alpha}"', from=1-1, to=3-3]
	        \arrow["\beta", from=1-3, to=3-3]
            \end{tikzcd}\]
            then we have the commutative diagram:
            \[\begin{tikzcd}[sep=small]
	        {\f(X)} && {\f(Y)} \\
	        \\
	        && {\f(Z)}
	        \arrow["{\f(\alpha)}"', from=1-3, to=1-1]
	        \arrow["{\f(\beta\circ \alpha) = \f(\alpha) \circ \f(\beta)}", shift left=2, from=3-3, to=1-1]
	        \arrow["{\f(\beta)}"', from=3-3, to=1-3]
            \end{tikzcd}\]
        \end{itemize}
    \end{enumerate}
\end{defn}

\medskip

\begin{cor}
    For any $R$-module $V$ the following
    \[\f:=\h_R(V,-):R\text{-}mod \to Ab\]
    is a left exact covariant functor. Moreover, the functor is exact if and only if $V$ is projective.
\end{cor}

\begin{proof}
    Let $X$ be an $R$-module and consider $\h_R(V,X)$. Let $\alpha:X\to Y$ be an $R$-module homomorphism and denote $\f(\alpha)= \alpha_*: \h_R(V,X)\to \h_R(V,Y)$. We prove the axiom for a covariant functor. Suppose we have $X\xto\alpha Y \xto\beta Z$, and so we have $\beta\circ \alpha: X\to Z$. As shown previously, this gives the following sequence:
    \[\h_R(V,X)\xto {\alpha_*} \h_R (V,Y)\xto {\beta_*} \h_R(V,Z)\]
    where clearly $\beta_*\circ \alpha_* = (\beta\circ \alpha)_*$. This proves the second axiom. Next, for the first axiom, define $\id_X:X\to X$ be the identity map of $X$. Then $(\id_X)_*: f\mapsto \id\circ f = f$, which shows that $(\id_X)_* = \id_{\h_R(X,X)}$. Therefore $\f$ is a covariant functor.

    For the second part of the statement, it follows directly from the definition of projective modules. This completes the proof.
\end{proof}