\begin{ex}
    \hfill

    \begin{enumerate}
        \item We compute $\ext^n_\Z(\Z/m\Z, D)$ for any abelian group $D$, where $m\geq 2$. From previous proposition we know 
        \[\ext_\Z^0(\Z/m\Z, D)\cong \h_\Z(\Z/m\Z, D)\]
        Let $\varphi\in \h_\Z(\Z/m\Z, D)$. By definition 
        \[m(\varphi(\overline{1})) = \varphi(\overline{m}) = 0\]
        Thus we know 
        \[\ext_\Z^0(\Z/m\Z, D)\cong \h_\Z(\Z/m\Z, D) \cong \sbr{d\in D:m\cdot d = 0}\]
        where we will denote it as $_mD$. To investigate the general case, we have to come up with the projective resolution:
        \[\dots \to 0\to \Z\xto{\times m}\Z \xto{\bmod m} \Z/m\Z \to 0\]
        This is indeed a free resolution of $\Z/m\Z$ (one can verify the exactness easily). Thus taking $\hom$ we get
        \[0\to \h_\Z(\Z,D) \xto{(\times m)^*}\h_\Z(\Z,D)\to 0 \to \dots\]
        And it is clear that $\ext^2_\Z(\Z/m\Z, D)=0$ for all $n\geq 2$. We compute the $\ext^1$ as follow: note $\h_\Z(\Z, D)\cong D$ with isomorphism given by $\varphi\mapsto \varphi(1)$. Thus:
        \[\begin{tikzcd} [sep = small]
	&& \varphi && {m\varphi} \\
	0 && {\h_\Z(\Z, D)} && {\h_\Z(\Z, D)} && 0 && \dots \\
	\\
	0 && D && D && 0 && \dots \\
	&& {\varphi(1)} && {m\varphi(1) = \varphi(m)}
	\arrow[maps to, from=1-3, to=1-5]
	\arrow["\in"{marking, allow upside down}, draw=none, from=1-3, to=2-3]
	\arrow[curve={height=18pt}, maps to, from=1-3, to=5-3]
	\arrow["\in"{marking, allow upside down}, draw=none, from=1-5, to=2-5]
	\arrow[curve={height=-30pt}, maps to, from=1-5, to=5-5]
	\arrow[from=2-1, to=2-3]
	\arrow["{(\times m)^*}", from=2-3, to=2-5]
	\arrow[from=2-3, to=4-3]
	\arrow[from=2-5, to=2-7]
	\arrow[from=2-5, to=4-5]
	\arrow[from=2-7, to=2-9]
	\arrow[from=4-1, to=4-3]
	\arrow[from=4-3, to=4-5]
	\arrow[from=4-5, to=4-7]
	\arrow[from=4-7, to=4-9]
	\arrow["\in"{marking, allow upside down}, draw=none, from=5-3, to=4-3]
	\arrow[from=5-3, to=5-5]
	\arrow["\in"{marking, allow upside down}, draw=none, from=5-5, to=4-5]
    \end{tikzcd}\]
        and $\ext_\Z^1(\Z/m\Z, D)\cong D/mD$.
    \item \todo{explain on how to construct}
    \end{enumerate}
\end{ex}

\medskip

\begin{re}
    The above example is not rigorous enough, in the sense that, we cannot assume that we still get the same result if starting from another projective resolution. In fact, we have that the result obtained is independent of the projective resolution.
\end{re}

\medskip

\begin{pro} [Comparison Theorem]
    Let $f:V\to V'$ be an $R$-module homomorphism and $P_\bullet\twoheadrightarrow V$ be a projective resolution of $V$ and $P'_\bullet \twoheadrightarrow V'$ be an exact complex, where it need not to be a projective resolution of $V'$. Then there exists $f_n:P_n\to P_n'$ such that the following commute:
    \[\begin{tikzcd}[sep=small]
	{\dots } && {P_3} && {P_2} && {P_1} && {P_0} && V && 0 \\
	\\
	\dots && {P_3'} && {P'_2} && {P'_1} && {P'_0} && {V'} && 0
	\arrow[from=1-1, to=1-3]
	\arrow["{d_3}", from=1-3, to=1-5]
	\arrow["{f_3}"', from=1-3, to=3-3]
	\arrow["{d_2}", from=1-5, to=1-7]
	\arrow["{f_2}", from=1-5, to=3-5]
	\arrow["{d_1}", from=1-7, to=1-9]
	\arrow["{f_1}", from=1-7, to=3-7]
	\arrow["{d_0}", from=1-9, to=1-11]
	\arrow["{f_0}", from=1-9, to=3-9]
	\arrow[from=1-11, to=1-13]
	\arrow["f", from=1-11, to=3-11]
	\arrow[from=1-13, to=3-13]
	\arrow[from=3-1, to=3-3]
	\arrow["{\delta_3}"', from=3-3, to=3-5]
	\arrow["{\delta_2}"', from=3-5, to=3-7]
	\arrow["{\delta_1}"', from=3-7, to=3-9]
	\arrow["{\delta_0}"', from=3-9, to=3-11]
	\arrow[from=3-11, to=3-13]
\end{tikzcd}\]
    Futhermore, given two such maps $f_n:P_n \to P_n'$ and $g_n:P_n\to P_n'$, there exists $s_n:P_n \to P_{n+1}'$ such that $f_n-g_n=\delta_{n+1}s_n + s_{n-1}d_n$.
\end{pro}
\begin{proof}
    Idea: use projective to get a map, and do induction.

    Part 2: Changing between straight square and slanted square, use projective module's property to get a lifting map, and perform induction.

    \todo{proof}
\end{proof}

\begin{defn}[Homotopic and homotopy equivalence]
    \hfill

    \begin{enumerate}
        \item Let $f,g: X^\bullet \to Y^\bullet$ be morphisms of complexes. We say that $f$ and $g$ are homotopic, denoted by $f\simeq g$, if there exists $s_\bullet$ be a collection of map where $s_n: X^n \to Y^{n+1}$ such that $f-g = ds + sd$.
        \item The complexes $X^\bullet$ and $Y^\bullet$ are homotopy equivalent if there exists $f:X^\bullet \to Y^\bullet$ and $f':Y^\bullet \to X^\bullet$ such that $f\circ f' \simeq \id_{Y^\bullet}$ and $f'\circ f \simeq \id_{X^\bullet}$. 
    \end{enumerate}
\end{defn}

\medskip

\begin{pro}
    \hfill

    \begin{enumerate}
        \item Suppose that $f,g:X^\bullet \to Y^\bullet$ are homotopic. We have 
        \[f^* = g^* : H^n (X^\bullet) \to H^n(Y^\bullet)\]
        \item If $X^\bullet$ and $Y^\bullet$ are homotopy equivalent, then $H^n(X^\bullet)\cong H^n(Y^\bullet)$.
    \end{enumerate}
\end{pro}
\begin{proof}
    The map $f:X^\bullet \to Y^\bullet$ induces $f^*: H^n(X^\bullet)\to H^n(Y^\bullet)$ where $[x]\mapsto [f(x)]$. This is similar for $g$. Since $f$ and $g$ are homotopic, there exists some map $s_n:X^n \to Y^{n+1}$ such that $f-g = ds + sd$. Observe that
    \begin{align*}
        f^*([x])
        &= [f(x)] \\
        &= [g(x)+ds(x) + sd(x)]\\
        &= [g(x)] + [ds(x)]\\
        &= [g(x)] + 0\\
        &= g^*([x])
    \end{align*}
    This proves the first statement.

    For the second statement, if $X^\bullet$ and $Y^\bullet$ are homotopy equivalent, it means having $f:X^\bullet\to Y^\bullet$ and $f':Y^\bullet\to X^\bullet$ such that $f\circ f' \simeq \id_{Y^\bullet}$ and $f'\circ f \simeq \id_{X^\bullet}$. These maps induces $f^*:H^n(X^\bullet) \to H^n(Y^\bullet)$ and ${f'}^*:H^n(Y^\bullet) \to H^n(X^\bullet)$. Note
    \[({f'}^*\circ f^*)([x]) = (f'\circ f)^*([x]) = (\id_{X^\bullet})^*([x]) = [x]\]
    Thus ${f'}^*\circ f^* = \id_{H^n(X^\bullet)}$. Similarly one can show that ${f}^*\circ {f'}^* = \id_{H^n(Y^\bullet)}$. This shows that $f^*$ is an isomorphism, and so $H^n(X^\bullet)$ and $H^n(Y^\bullet)$ are isomorphic.
\end{proof}

\begin{thm}
    The $n$-th cohomology group $\ext^n_R(V,W)$ is independent, up to isomorphism, of the choice of the projective residue of $V$.
\end{thm}
\begin{proof}
    \todo{proof}
\end{proof}

\begin{thm} [Snake Lemma]
    Suppose we have a commutative diagram below with exact rows:
    \[\begin{tikzcd} [sep = small]
	0 && X && Y && Z && 0 \\
	\\
	0 && {X'} && {Y'} && {Z'} && 0
	\arrow[from=1-1, to=1-3]
	\arrow["\alpha", from=1-3, to=1-5]
	\arrow["f", from=1-3, to=3-3]
	\arrow["\beta", from=1-5, to=1-7]
	\arrow["g", from=1-5, to=3-5]
	\arrow[from=1-7, to=1-9]
	\arrow["h", from=1-7, to=3-7]
	\arrow[from=3-1, to=3-3]
	\arrow["{\alpha'}", from=3-3, to=3-5]
	\arrow["{\beta'}", from=3-5, to=3-7]
	\arrow[from=3-7, to=3-9]
    \end{tikzcd}\]
    We then have an exact sequence
    \[0 \to \ker f \xto{\alpha} \ker g \xto\beta \ker h \xto\delta \coker f \xto {\alpha'} \coker g \xto {\beta'} \coker h \to 0\]
    where cokernel of a map $\phi:A\to B$ is defined as the quotient $\coker \phi = B/\im \phi$.
\end{thm}
\begin{proof}
    Left as tutorial exercise.
\end{proof}

\begin{thm} [Horseshoe Lemma]
    Let $0\to X\to Y \to Z\to 0$ be a SES of $R$-modules and $P_\bullet \twoheadrightarrow X$, $Q_\bullet \twoheadrightarrow Z$ be projective resolutions of $X,Z$ respectively. Then we have an exact commutative diagram:
    \[\begin{tikzcd} [sep=small]
	&& 0 && 0 && 0 && 0 \\
	\\
	\dots && {P_2} && {P_1} && {P_0} && X && 0 \\
	\\
	\dots && {P'_2} && {P'_1} && {P'_0} && Y && 0 \\
	\\
	\dots && {Q_2} && {Q_1} && {Q_0} && Z && 0 \\
	\\
	&& 0 && 0 && 0 && 0
	\arrow[from=1-3, to=3-3]
	\arrow[from=1-5, to=3-5]
	\arrow[from=1-7, to=3-7]
	\arrow[from=1-9, to=3-9]
	\arrow[from=3-1, to=3-3]
	\arrow["{d_2}", from=3-3, to=3-5]
	\arrow[from=3-3, to=5-3]
	\arrow["{d_1}", from=3-5, to=3-7]
	\arrow[from=3-5, to=5-5]
	\arrow["{d_0}", from=3-7, to=3-9]
	\arrow[from=3-7, to=5-7]
	\arrow[from=3-9, to=3-11]
	\arrow[from=3-9, to=5-9]
	\arrow[from=5-1, to=5-3]
	\arrow[from=5-3, to=5-5]
	\arrow["{\pi_2}"', from=5-3, to=7-3]
	\arrow[from=5-5, to=5-7]
	\arrow["{\pi_1}"', from=5-5, to=7-5]
	\arrow[from=5-7, to=5-9]
	\arrow["{\pi_0}"', from=5-7, to=7-7]
	\arrow[from=5-9, to=5-11]
	\arrow[from=5-9, to=7-9]
	\arrow[from=7-1, to=7-3]
	\arrow["{\delta_2}"', from=7-3, to=7-5]
	\arrow[from=7-3, to=9-3]
	\arrow["{\delta_1}"', from=7-5, to=7-7]
	\arrow[from=7-5, to=9-5]
	\arrow["{\delta_0}"', from=7-7, to=7-9]
	\arrow[from=7-7, to=9-7]
	\arrow[from=7-9, to=7-11]
	\arrow[from=7-9, to=9-9]
\end{tikzcd}\]
In particular, we denote the second non-zero row as $P_\bullet \oplus Q_\bullet$, and it is a projective resolution of $Y$.
\end{thm}
\begin{proof}
    Left as tutorial exercise.
\end{proof}

\begin{thm}
    Let $0\to X\to Y\to Z\to 0$ be a SES of $R$-modules. Then we have a LES of abelian groups 
    \[0 \to \h_R(Z,D) \to \h_R(Y,D) \to \h_R(X,D)\] 
    \[\to \ext^1_R(Z,D) \to \ext^1_R(Y,D) \to \ext^1_R(X,D) \to \ext^2_R(Z,D)\to \dots\]
\end{thm}
\begin{proof}
    Take projective resolution $P_\bullet \twoheadrightarrow X$ and $Q_\bullet \twoheadrightarrow Z$. By Horseshoe Lemma, we get the diagram 
    \[\begin{tikzcd} [sep=small]
	&& 0 && 0 && 0 && 0 \\
	\\
	\dots && {P_2} && {P_1} && {P_0} && X && 0 \\
	\\
	\dots && {P'_2} && {P'_1} && {P'_0} && Y && 0 \\
	\\
	\dots && {Q_2} && {Q_1} && {Q_0} && Z && 0 \\
	\\
	&& 0 && 0 && 0 && 0
	\arrow[from=1-3, to=3-3]
	\arrow[from=1-5, to=3-5]
	\arrow[from=1-7, to=3-7]
	\arrow[from=1-9, to=3-9]
	\arrow[from=3-1, to=3-3]
	\arrow["{d_2}", from=3-3, to=3-5]
	\arrow[from=3-3, to=5-3]
	\arrow["{d_1}", from=3-5, to=3-7]
	\arrow[from=3-5, to=5-5]
	\arrow["{d_0}", from=3-7, to=3-9]
	\arrow[from=3-7, to=5-7]
	\arrow[from=3-9, to=3-11]
	\arrow[from=3-9, to=5-9]
	\arrow[from=5-1, to=5-3]
	\arrow[from=5-3, to=5-5]
	\arrow["{\pi_2}"', from=5-3, to=7-3]
	\arrow[from=5-5, to=5-7]
	\arrow["{\pi_1}"', from=5-5, to=7-5]
	\arrow[from=5-7, to=5-9]
	\arrow["{\pi_0}"', from=5-7, to=7-7]
	\arrow[from=5-9, to=5-11]
	\arrow[from=5-9, to=7-9]
	\arrow[from=7-1, to=7-3]
	\arrow["{\delta_2}"', from=7-3, to=7-5]
	\arrow[from=7-3, to=9-3]
	\arrow["{\delta_1}"', from=7-5, to=7-7]
	\arrow[from=7-5, to=9-5]
	\arrow["{\delta_0}"', from=7-7, to=7-9]
	\arrow[from=7-7, to=9-7]
	\arrow[from=7-9, to=7-11]
	\arrow[from=7-9, to=9-9]
\end{tikzcd}\]
Taking $\h_R(-,D)$ we get 
\[\begin{tikzcd}[sep=small]
	&& 0 && 0 && 0 \\
	\\
	0 && {\h_R(P_0, D)} && {\h_R(P_1, D)} && {\h_R(P_2, D)} && \dots \\
	\\
	0 && {\h_R(P_0 \oplus Q_0, D)} && {\h_R(P_1 \oplus Q_1, D)} && {\h_R(P_2 \oplus Q_2, D)} && \dots \\
	\\
	0 && {\h_R(Q_0, D)} && {\h_R(Q_1, D)} && {\h_R(Q_2, D)} && \dots \\
	\\
	&& 0 && 0 && 0
	\arrow[from=3-1, to=3-3]
	\arrow[from=3-3, to=1-3]
	\arrow["{d_1^*}", from=3-3, to=3-5]
	\arrow[from=3-5, to=1-5]
	\arrow["{d_2^*}", from=3-5, to=3-7]
	\arrow[from=3-7, to=1-7]
	\arrow[from=3-7, to=3-9]
	\arrow[from=5-1, to=5-3]
	\arrow["{\iota_0^*}"', from=5-3, to=3-3]
	\arrow["{\gamma_1^*}", from=5-3, to=5-5]
	\arrow["{\iota_0^*}"', from=5-5, to=3-5]
	\arrow["{\gamma_2^*}", from=5-5, to=5-7]
	\arrow["{\iota_0^*}"', from=5-7, to=3-7]
	\arrow[from=5-7, to=5-9]
	\arrow[from=7-1, to=7-3]
	\arrow["{\pi_0^*}"', from=7-3, to=5-3]
	\arrow["{\delta_1^*}", from=7-3, to=7-5]
	\arrow["{\pi_0^*}"', from=7-5, to=5-5]
	\arrow["{\delta_2^*}", from=7-5, to=7-7]
	\arrow["{\pi_0^*}"', from=7-7, to=5-7]
	\arrow[from=7-7, to=7-9]
	\arrow[from=9-3, to=7-3]
	\arrow[from=9-5, to=7-5]
	\arrow[from=9-7, to=7-7]
\end{tikzcd}\]

Lastly, by theorem regarding LES on cohomology, we have 
\[0\to \h^0(A^\bullet)\]
\end{proof}

\begin{re}
    Note that since $\h_R(X,D)\cong \ext_R^0(X,D)$, thus the whole LES above is a LES of ext groups.
\end{re}

\medskip

\begin{thm}
    Let $Q$ be a $R$-module. TFAE:
    \begin{enumerate}
        \item $Q$ is injective.
        \item $\ext^1_R(A,Q)=0$ for all $R$-module $A$
        \item $\ext^n_R(A,Q)=0$ for all $R$-module $A$ and $n\in\Z^+$.
    \end{enumerate}
\end{thm}
\begin{proof}
    $3. \implies 2.$ is trivial.

    For $2. \implies 1.$, suppose given SES $0 \to X\to Y\to Z\to 0$, we have LES on the Ext group
    \[0 \to \h_R(Z,Q) \to \h_R(Y,Q) \to \h_R(X,Q) \to \ext_R^1(Z,D)=0\to \dots\]
    This reduces into a SES
    \[0 \to \h_R(Z,Q) \to \h_R(Y,Q) \to \h_R(X,Q) \to 0\]
    So $Q$ is injective.


\end{proof}


